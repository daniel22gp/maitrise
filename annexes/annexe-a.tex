\chapter{Le titre}
% Trouve un titre plus spécifique

\section{Section un de l'annexe A}

La premi\`ere annexe du document.

Pour plus de renseignements, consultez le site \href{http://www.fesp.umontreal.ca}{web de la FESP}.
\begin{table}[htb]
	\renewcommand{\arraystretch}{1.25}
	\newcommand{\dotrule}[1]{\parbox[t]{#1}{\dotfill}}
	\centering
	\caption[Titre alternatif pour la table des mati\`eres]{Liste des parties}
	\label{tab:parties}
	\begin{tabular}{p{0.6\textwidth}@{\hspace{0.15\textwidth}}p{0.15\textwidth}}
		\hline\hline & \\[-3mm]
  		Les couvertures conformes 											& obligatoires			\\
		Les pages de garde 													& obligatoires			\\
		La page de titre 													& obligatoire			\\
		Le résumé en français et les mots clés français						& obligatoires			\\
		Le résumé en anglais et les mots clés anglais 						& obligatoires			\\
		Le résumé de vulgarisation											& facultatif			\\
		La table des mati\`eres, la liste des tableaux, la liste des figures 	& obligatoires			\\
		La liste des sigles, la liste des abréviations						& obligatoires			\\
		La dédicace															& facultative			\\
		Les remerciements 													& facultatifs			\\
		L'avant-propos 														& facultatif			\\
		Le corps de l'ouvrage												& obligatoire			\\
		L'index analytique													& facultatif			\\
		Les sources documentaires 											& obligatoires			\\
		Les appendices (annexes) 											& facultatifs			\\
		Le curriculum vit\ae{}												& facultatif			\\
		Les documents spéciaux 												& facultatifs			\\
		[3mm] \hline\hline
	\end{tabular}
\end{table}
Pour plus de renseignements, consultez le site \href{http://www.fesp.umontreal.ca}{web de la FESP}.
