\chapter*{Résumé} 	% La commande "\chapter*" crée un chapitre sans numéro, contrairement \`a la commande "\chapter" réguli\`ere.
% N.B. : La commande "\noindent" force LaTeX \`a ne pas indenter le nouveau paragraphe.
La \ac{GAT} a comme objectif de produire du texte compréhensible en langue naturelle à partir de données non-linguistiques. Les générateurs font essentiellement deux tâches: d'abord ils déterminent le contenu d'un message à communiquer, puis ils sélectionnent les mots et les constructions syntaxiques qui serviront à transmettre le message, aussi appellée la réalisation linguistique. Pour générer des textes aussi naturels que possible, un système de \ac{GAT} doit être doté de ressources lexicales riches. Si on veut avoir un maximum de flexibilité dans les réalisations, il nous faut avoir accès aux différentes propriétés de combinatoire des unités lexicales d'une langue donnée. Puisque les verbes sont au c\oe{}ur de chaque énoncé et qu'ils contrôlent généralement la structure de la phrase, il faudrait encoder leurs propriétés afin de produire du texte exploitant toute la richesse des langues. De plus, les verbes ont des propriétés de combinatoires imprévisibles, c'est pourquoi il faut les encoder dans un dictionnaire.

Ce mémoire porte sur l'intégration de VerbNet, un dictionnaire riche de verbes de l'anglais et de leurs comportements syntaxiques, à un réalisateur profond, GenDR. Pour procéder à cette implémentation, nous avons utilisé le langage de programmation Python pour extraire les données de VerbNet et les manipuler pour les adapter à GenDR, un réalisateur profond basé sur la théorie Sens-Texte. Nous avons ainsi intégré 274 cadres syntaxiques à GenDR ainsi que 6\,393 verbes de l'anglais.

\textbf{Mots-clés}: génération automatique de texte; réalisation linguistique; patrons de régime; cadres syntaxiques; verbes; Théorie Sens-Texte

\chapter*{Abstract}

Natural language generation's (NLG) goal is to produce understandable text from non-linguistic data. Generation essentially consists in two tasks: first, determine the content of a message to transmit and then, carefully select the words that will transmit the desired message. That second task is called linguistic realization. An NLG system requires access to a rich lexical ressource to generate natural-looking text. If we want a maximum of flexibility in the realization, we need access to the combinatory properties of a lexical unit. Because verbs are at the core of each utterance and they usually control its structure, we should encode their properties to generate text representing the true richness of any language. In addition to that, verbs are highly unpredictible in terms of syntactic behaviours, which is why we need to store them into a dictionary.

This work is about the integration of VerbNet, a rich lexical ressource on verbs and their syntactic behaviors, into a deep realizer called GenDR. To make this implementation possible, we have used the Python programming language to extract VerbNet's data and to adapt it to GenDR. We have imported 274 syntactic frames and 6\,393 verbs.

\textbf{Keywords}: natural language generation; linguistic realization; government patterns; syntactic frames; verbs; Meaning-Text theory