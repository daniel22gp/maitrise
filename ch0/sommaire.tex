\chapter*{Sommaire} 	% La commande "\chapter*" crée un chapitre sans numéro, contrairement \`a la commande "\chapter" réguli\`ere.
% N.B. : La commande "\noindent" force LaTeX \`a ne pas indenter le nouveau paragraphe.
La \ac{GAT} a comme objectif de produire du texte compréhensible en langue naturelle à partir de données non-linguistiques. Les générateurs font essentiellement deux tâches, d'abord ils déterminent le contenu d'un message à énoncer puis ils sélectionnent les mots qui serviront à transmettre le message, aussi appellée la réalisation linguistique. Pour réaliser du texte le plus naturel possible, un système de \ac{GAT} doit se doter de ressources lexicales riches. Si on veut couvrir un maximum de de formes d'énoncés, il nous faut avoir accès aux différents comportements des unités lexicales d'une langue donnée. Puisque les verbes sont au c\oe{}r de chaque énoncé et qu'ils contrôlent généralement la structure d'une phrase, il faudrait encoder leurs propriétés afin de produire du texte représentant la richesse réelle des langues. De plus, les verbes sont imprévisibles en termes de comportements syntaxiques, c'est pourquoi il faut les encoder dans un dictionnaire. Par exemple, on ne peut pas prévoir que le verbe \lex{parler} demandera la préposition \lex{de} pour la phrase \form{Paul parle d'un évènement à Martine}, tandis que \lex{relater} n'en a pas besoin \form{Paul relate un évènement à Martine}. Cet exemple illustre que deux verbes synonymiques engendrent des constructions très différentes, arbitrairement. Il faut donc encoder ce savoir dans un dictionnaire pour que les applications \ac{TAL} comme la \ac{GAT} reproduisent ces comportements correctement.

Ce mémoire porte sur l'intégration de VerbNet, une ressource lexicale riche sur les verbes et leurs comportements syntaxiques, à un réalisateur profond GenDR, dans le cadre de la \ac{GAT}. Pour procéder à cette implémentation, nous avons utilisé le langage de programmation Python et le module \emph{xml.etree.cElementTree} pour extraire les données de VerbNet et les manipuler pour les adapter à GenDR un réalisateur profond opérant dans le cadre de la théorie Sens-Texte. Nous avons ainsi intégré 274 cadres syntaxiques à GenDR ainsi que 6\,394 verbes.

\textbf{Mots-clés}: génération automatique de texte; réalisation linguistique; patrons de régime; cadres syntaxiques; verbes; Théorie Sens-Texte; traitement automatique des langues; linguistique

\chapter*{Summary}



\textbf{Keywords}: natural language generation; linguistic realisation; government patterns; syntactic frames; Meaning-Text theory; linguistics; natural language processing