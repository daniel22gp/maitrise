% TABLE DES MATIÈRES
\cleardoublepage
\pdfbookmark[chapter]{\contentsname}{toc}  %Crée un bouton sur la bar de navigation
\tableofcontents				% Table des mati\`eres.
% LISTE DES TABLEAUX
\cleardoublepage
\phantomsection
\listoftables
% LISTE DES FIGURES
\cleardoublepage
\phantomsection
\listoffigures	


%%%%%%%%%%%%%%%%%%%%%%%%%%%%%%%%%%%%%
%% LISTE DES SIGLES ET ABRÉVIATION %
%%%%%%%%%%%%%%%%%%%%%%%%%%%%%%%%%%%%%
%% Il est obligatoire, selon les directives de la FESP, 
%% pour une thèse ou un mémoire d'avoir une liste des sigles et 
%% des abréviations.  Si vous considérez que de telles listes ne seraient pas
%% pertinentes (si, par exemple, vous n'utilisez aucun sigle ou abré.), son
%% inclusion ou omission est laissé à votre discrétion.  En cas de doute,
%% parlez-en à votre directeur de recherche, le coadministrateur ou, ultimement, à
%% la FESP directement.
%%
%% Dans le cas où vous incluez une table des sigles et des abréviations,
%% vous pouvez enlever les % de la section suivante pour faire apparaître
%% un exemple d'une telle liste « fait à la main ».  Il existe des outils
%% plus sophistiqués si vous devez inclure une multitude de sigles et abréviations.
%% (Par exemple, le package <glossaries> peut faire des index élaborés.  Comme
%% son utilisation est technique, il n'y a pas d'exemple directement dans ce gabarit.
%% On invite les gens qui aurait à l'utiliser à consulter le wiki
%% du dms, le coadministrateur ou faire leur propre recherche.)

\chapter*{Liste des sigles et des abréviations}
\begingroup %Pour que le \renewcommand soit local
%Modifiez ce nombre (p.ex.remplacez 2 par 1.5) pour augmenter ou diminuer l'espace entre les lignes du tableau.
\renewcommand{\arraystretch}{2} 
\noindent\begin{tabular}{p{.2\textwidth} p{.7\textwidth}}
  GAT & Génération automatique de texte \\
  GP & Patron de régime, de l'anglais \textit{Government Pattern}\\
  DPOS  &  Partie du discours profonde, de l'anglais \textit{Deep Part of Speech}\\
  TST & Théorie Sens-Texte\\
  VN & \emph{VerbNet}\\
\end{tabular}
\endgroup  %Fin du groupe local pour \arraystretch


