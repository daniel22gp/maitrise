%!TEX root = ../memoire.tex

\chapter*{Introduction}

% \noindent c'est mieux de changer la définition de \chapter et autres titres pour que ça soit uniforme

Depuis le siècle dernier, la communication en intelligence artificielle a fasciné le commun des mortels. Ainsi, énormément de recherche s'oriente vers le développement de système pouvant communiquer avec les humains. Or, la communication implique deux processus: la compréhension du langage et la production de celle-ci. Ce sont deux concepts qui ont été approfondi par les scientifiques au cours des dernières décennies. Cela a donné naissance au \ac{TAL}. La \ac{GAT} est une branche du \ac{TAL} dont le but est de produire du texte compréhensible en langue naturelles à partir de données non-linguistiques. Les systèmes de \ac{GAT} se servent des connaissances linguistiques et des connaissances de domaine d'application pour générer automatiquement des documents. Ainsi, la \ac{GAT} est un domaine de recherche menant vers des applications concrètes et utile dans le monde réel. Les premiers systèmes de \ac{GAT} ayant vu le jour ont été créés pour produire des rapports automatiquement pour alléger la charge de travail des humains \citep{ReiterBuildingNaturalLanguage2000}. Autrement, la production de tels rapport est faite par un humain, ce qui entraîne des coûts importants souvent prohibitifs. La \ac{GAT} est aussi utilisée pour produire des documents résumant des informations complexes pour des gens n'ayant pas les connaissances de bases requises pour les (les données brutes) comprendre ou le temps requis. Comme des données numériques brutes sur la qualité de l'air (mettre les citations). Mais pas que pour des cas de traitement de données ilisibles, y'a aussi eu des cas de robo-journalsime \citep{W17-3513}. On est même rendu à avoir des générations à base de image-texte (description d'image) et même texte-texte pour la production de résumé. Bref, la \ac{GAT} se voit dans plusieurs milieux et prend de différents inputs. Pas uniforme comme science.

\draft{Un système de GAT classique se divise en deux parties. D'abord, ce que Danlos appelle le quoi-dire et Gatt le early process, équivaut à déterminer le contenu et le structurer. Autrement dit, il s'agit d'évaluer parmi les données s'offrant à nous, lesquelles voulons nous transmettre et dans quel ordre. Ensuite il y a le comment-le-dire que Gatt appelle le late process qui revient à choisir les mots qui transmettront le message qu'on souhaite exprimer. La réalisation linguistique est l'une}. La réalisation linguistique.

GenDR, héritier de MARQUIS (ce que faisais Marquis), TST (théorie qui a fait ses preuves).

Ce mémoire s'inscrit dans un contexte où on continue à chercher la meilleure manière de modéliser les connaissances linguistiques pour qu'un système de \ac{GAT} génère du texte le plus fluidement possible. Plusieurs chercheurs en pensent qu'un meilleur traitement du langage en \ac{TAL} passe par la connaissance des comportements des verbes. Puisque les verbes contrôlent la plupart des énoncés, en ayant des descriptions très précises des verbes, on devrait pouvoir modéliser toutes les constructions possibles par une langue. C'est pourquoi beaucoup de recherche s'est fait dans le but de développer des ressources lexicale computationnelle pour qu'elle soient utilisées dans tous les champs possibles du \ac{TAL}. Notamment les auteurs de VerbNet en s'inspirant de WordNet et FrameNet ont cru bon développer une ressource lexicala très riche ne focusant que sur les verbes. Nous pensons que la \ac{GAT} a beaucoup à gagner de ce type de ressource et nous tenterons d'implémenter VerbNet à notre système pour qu'il puisse réaliser le plus de phrases possibles. Cela agrandirait la couverture de GenDR et sa capacité de générer des phrases. Parler des comportements syntaxiques, Levin avait vu X, VerbNet a repris l'idée, et traite plus de 4000 vocables dont 6000 acceptions. Parler des patrons de régime, des verbes non prédictbles et pourquoi c'est important de les encoder.

Pour l'implémenter nous avons utilisé Python puisque les données de VerbNet sont encodées en XML et qu'il y existe un module Python pour lire ce type de fichier. Ensuite, nous avons manipulé les données pour qu'elles s'insèrent dans notre système, de 1 parce que notre logiciel demande un code x et parce que la théorie dans laquelle nous nous insérons demandait de faire quelques ajustement aux données que nous avions extrait. Puis nous avons implémenté le tout à notre système en l'adpatant.

Ce mémoire est divisé en six chapitres. Dans le premier, nous décrirons les six étapes du processus classique en \ac{GAT}, puis nous approfondirons l'une de ces étapes: la réalisation. Nous décrirons en conséquence des réalisateurs de texte pour faire un état de l'art sur la réalisation linguistique. Ensuite, le deuxième chapitre porte entièrement sur un réalisateur profond nommé GenDR (TST, multilingue, héritier de MARQUIS), capable de modéliser bcp de phénomènes langagiers.. On décrira en détails les modules qui le composent et ces forces et faiblesses. Ne fonctionne pas avec des dictionnaires verbaux.Donc on veut l'upgrader avec un dict de verbes pour qu'il soit encore plus puissant. Le troisième chapitre fait un survol des dictionnaires pour augmenter notre couverture. Et nous terminons par décrire VerbNet, la ressource que nous avons choisi. Par après, nous entrons dans le quatrième chapitre qui traite de l'extraction de VerbNet à l'aide de scripts Python. Nous avons extrait les verbes, l'architecture de VerbNet et les comportements syntaxiques des diverses classes verbales. Puis, le cinquième chapitre décrit comment nous avons adatpé GenDR pour qu'il incorpore les données que nous avons extraite de VerbNet. Nous avons du revoir certaines règles de grammaire et la composition de nos dictionnaires. Finalement, le sixième chapitre renferme l'évaluation du système. D'abord nous avons expliqué comment nous procédons à l'évaluation puis nos critères et finalement ce que les données disent. Puis pour conclure le tout, nous faisons une synthèse du travail en montrant la contribution de ce mémoire à l'état de l'art, puis nous évoquons quelques pistes à explorer.

\pagenumbering{arabic}

% Ici débute le corps de votre texte.

