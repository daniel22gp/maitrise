%!TEX root = ../memoire.tex

\chapter*{Introduction}

% \noindent c'est mieux de changer la définition de \chapter et autres titres pour que ça soit uniforme

Depuis le siècle dernier, la communication en intelligence artificielle continue de nous fasciner. Ainsi, énormément de recherche s'est orientée vers le développement de système informatique pouvant communiquer avec les humains. Or, la communication implique deux processus: la compréhension du langage et la production de celle-ci. Ce sont deux concepts qui ont été approfondi par les scientifiques au cours des dernières décennies dans le domaine du \ac{TAL}. La \ac{GAT} est une branche du \ac{TAL} dont le but est de produire du texte compréhensible en langue naturelles à partir de données non-linguistiques. Les systèmes de \ac{GAT} se servent des connaissances linguistiques et des connaissances de domaine d'application pour générer automatiquement des documents.  Les premiers systèmes de \ac{GAT} ayant vu le jour ont été créés pour produire des rapports automatiquement dans le but d'alléger la charge de travail des humains \citep{ReiterBuildingNaturalLanguage2000}. Sinon, l'écriture de ces documents est effectuée par un humain, ce qui entraîne des coûts importants. Ainsi, la \ac{GAT} est un domaine de recherche menant vers des applications concrètes et utiles dans le monde réel.

La \ac{GAT} est aussi utilisée pour produire des documents résumant des informations complexes pour des gens n'ayant pas les connaissances de bases requises pour les comprendre ou le temps requis. Par exemple, MARQUIS \citep{WannerMARQUISGENERATIONUSERTAILORED2010} est un système générant de tels rapports à partir de données numériques brutes sur la qualité de l'air. La \ac{GAT} a aussi fait une incursion dans le journalisme, ce qu'on appelle le robo-journalisme \citep{W17-3513}. Ainsi, des articles journalistiques peuvent être produits à partir de données brutes dans le but d'informer les gens, sans qu'un humain soit derrière le texte.

Traditionnellement un système de \ac{GAT} se divise en deux parties. D'abord, ce que \cite{DanlosPresentationmodelegeneration1983} appelle le \emph{quoi-dire} et que \cite{gatt18} appellent le \emph{early process} qui équivaut à déterminer le contenu qu'on génèrera et le structurer. Autrement dit, il s'agit d'évaluer parmi les données s'offrant à nous, lesquelles voulons-nous transmettre et dans quel ordre, c'est la génération profonde. Ensuite il y a ce que Danlos appelle le \emph{comment-le-dire} puis que Gatt et Kramer appellent le \emph{late process} qui revient à choisir les unités lexicales qui serviront à transmettre le message prévu lors de la génération profonde, on appelle cela la réalisation linguistique.

En soi, la réalisation linguistique a fait l'objet d'énormément de recherche, ainsi il existe sur le marché un nombre impressionnant de réalisateurs. Différentes méthodes: à base de patrons, à base de règles et statistiques. Nous on se sert d'à bases de règles donc notre système a un dictionnaire et une grammaire pour modéliser le langage. On divise les réalisateurs à base de règles: dictionnaires et règles de grammaires.. 

Contexte:Notre mémoire s'inscrit aussi dans un contexte où on continue à chercher la meilleure manière de modéliser les connaissances linguistiques pour qu'un système de \ac{GAT} génère du texte le plus fluidement possible.

Problématique1: Il faut avoir des ressources lexicales riches décrivant les différents comportements syntaxiques des verbes car les verbes contrôlent la structure de la plupart des énoncés et leurs comportements sont très imprévisibles. Donc, en détenant les propriétés syntaxiques des verbes d'une langue donnée, on peut couvrir une grande partie des constructions de phrases possibles pour celle-ci. De là vient notre objectif, qui est d'intégrer au réalisateur profond GenDR un dictionnaire de comportements syntaxiques renfermant une quantité exhaustive des constructions possibles de l'anglais. De plus, comme GenDR se veut multilingue, si l'intégration est réussie, alors cela nous amènerait vers l'acquisition d'autres ressources lexicales similaires dans d'autres langues afin d'exploiter l'aspect multilingue de GenDR à pleine capacité. Pour clarifier l'expression \scare{comportements syntaxiques}, il s'agit des cooccurences syntaxique d'une unité lexicale donnée avec les arguments que cette lexie sélectionne. Par exemple, la relation entre un verbe et son sujet, ou bien la relation entre un nom et le complément qu'il sélectionne. La raison pour laquelle nous encodons généralement ces comportements dans des dictionnaires est du à l'aspect imprévisible de ces comportemetns. En effet, on ne peut pas prédire le nombre d'actants qu'un prédicat gouverne, ou bien les prépositions qu'il régit. ce qui est illustré par \cite{MilicevicSchemaregimepont2009}: \form{on se souvient de $X$}, mais \form{on se rappelle $X$}.

Problématique2: De plus, une autre raison qui nous a poussé à acquérir une telle ressource provient du fait que l'architecture présente de GenDR, le réalisateur que nous utilisons présente des limites quant à l'encodage de divers comportements syntaxiques pour un lexème.Pour un verbe donné, on ne peut pas avoir deux parties du discours différentes qui compétitionnent pour la même position syntaxique. Autrement dit, si nous voulions exhaustivement représenter les comportements du verbe \lex{want}, il nous faudrait un \ac{GP} qui puisse tenir compte du fait que le second actant syntaxique de ce verbe peut avoir une \ac{DPOS} de type verbale ou nominale: \form{I want to eat} vs \form{I want a dog}. Cela nous était impossible à encoder dans MATE avec les paramètres que nous avions car le système ne nous laissait pas donner deux versions de l'actant syntaxique II de \lex{want}.Nous nous sommes alors tourné vers l'idée d'ajouter un dictionnaire supplémentaire à notre ressource, qui encoderait tous les régimes existants de l'anglais. Puis, nous n'aurions qu'à encoder l'identification des \ac{GP} dans les unités lexicales appropriées ce qui nous permettait de contourner le problème des \acp{GP} multiples. Notre ancienne méthode restreignait conséquemment le nombre de réalisations possibles pour un verbe donné.

Nous avons développé GenDR, un réalisateur profond multilingue qui reprend les rouages de la réalisation du générateur MARQUIS. GenDR a déjà fait l'objet d'un mémoire focusant sur sa capacité à rendre compte de phénomènes langagiers complexes: les collocations. Les deux fonctionnent sur la TST.

\draft{Élaborer sur les principes de bases de VerbNet} Nous pensons que GenDR sera amélioré grâce à cette implémentation. Plusieurs chercheurs dont \cite{SchulerVerbnetBroadcoverageComprehensive2005,Korhonenlargesubcategorizationlexicon2006} pensent qu'un meilleur traitement des langues naturelles passe par la connaissance des comportements des verbes. C'est pourquoi beaucoup de recherche s'est fait dans le but de développer des ressources lexicale computationnelle pour qu'elle soient utilisées dans tous les champs possibles du \ac{TAL}. Notamment les auteurs de VerbNet en s'inspirant de WordNet et FrameNet ont cru bon développer une ressource lexicala très riche ne focusant que sur les verbes. Nous pensons que la \ac{GAT} a beaucoup à gagner de ce type de ressource et nous tenterons d'implémenter VerbNet à notre système pour qu'il puisse réaliser le plus de phrases possibles. Cela agrandirait la couverture de GenDR et sa capacité de générer des phrases. Parler des comportements syntaxiques, Levin avait vu X, VerbNet a repris l'idée, et traite plus de 4000 vocables dont 6000 acceptions.

\draft{VerbNet a été créé dans un contexte où il y avait un réel besoin pour un dictionnaire décrivant la richesse et la complexité des verbes \citep{KipperClassBasedConstructionVerb2000}. \cite{SchulerVerbnetBroadcoverageComprehensive2005} trouvait qu'il y avait un manque de lignes directrices par rapport à l'organisation des verbes dans les dictionnaires destinés à des applications \ac{TAL}, et c'est pour remédier à cela qu'elle a construit VerbNet. Son dictionnaire est organisé en une hiérarchie de classes verbales héritées de \cite{verb-classes.levin.1993}. Nous allons d'abord présenter ce classement avant de voir en détail VerbNet.

\cite{verb-classes.levin.1993} proposait une méthode de classification des verbes qui a inspiré plusieurs dictionnaires, dont VerbNet \citep{SchulerVerbnetBroadcoverageComprehensive2005} et la LCS database \citep{AyanGeneratingParsingLexicon2002a,DorrUseLexicalSemantics1992}. Dans sa classification, les verbes de la langue anglaise sont placés dans un nombre fini de classes verbales. L'appartenance d'un verbe à l'une d'entre elles est motivée par le partage de comportements syntaxiques communs. Levin remarquait que tout locuteur natif est conscient des alternances de diathèses possibles pour un verbe, et ce sans avoir de connaissances méta-linguistiques préalables. Ainsi, en se basant sur son intuition, Levin a tenté de délimiter tous les patrons de régime possibles pour les verbes de la langue anglaise. Lorsque plusieurs présentaient des caractéristiques communes sur le plan syntaxique, elle rassemblait ces verbes dans une classe.}

L'objet de ce mémoire est de pourvoir GenDR d'une couverture linguistique beaucoup plus grande que celle qu'il a présentement en y intégrant VerbNet qui est une base de données lexicales décrivant les comportements syntaxiques de 6\,394 verbes. On devrait pouvoir modéliser toutes les constructions possibles par une langue en ayant des descriptions très précises des verbes puisque les verbes contrôlent la plupart des énoncés. Pour implémenter VerbNet à GenDR, nous avons utilisé le langage de programmation Python qui nous a permi de manipuler et d'extraire les données de VerbNet. Manipuler pour que les informations marchent avec la TST. Pour compléter l'implémentation de cette ressource, nous avons du revoir certaines règles de grammaire et l'architecture de nos dictionnaires pour tenir compte des nouvelles informations lexicales auxquelles nous avons accès. Finalement, nous avons testé le tout pour vérifier si l'intégration de VerbNet fonctionnait.

Nous avons séparé ce mémoire en 6 chapitres dont les trois premiers chapitre correspondent à la préparation du projet et les trois derniers correspondent à l'expérience comme telle. Ainsi, le premier chapitre décrit ce qu'est la \ac{GAT} en expliquant les six étapes du processus classique de génération de texte, puis dans ce même chapitre, nous approfondissons une de ces étapes: la réalisation. Nous décrivons en détails ce qu'est la réalisation puis nous présentons quelques réalisateurs pour montrer les différences qui existent entre ceux-ci. Ensuite, le deuxième chapitre porte entièrement sur un réalisateur profond nommé GenDR: un réalisateur multilingue fonctionnant dans le cadre de la \ac{TST}. Nous décrivons comment ce réalisateur fait pour modéliser les phénomènes langagiers en expliquant en détails les modules qui le composent. Le troisième chapitre est dédié à la description de ressources lexicales focusant sur les comportements syntaxiques des verbes. Comme GenDR ne possède pas de telles ressources, nous avons fait des recherches pour trouver quel sera le meilleur et nous sommes tombés sur VerbNet. Nous décrirons plus en détails cette ressource lexicale.

Dans le quatrième chapitre, nous expliquons comment nous avons procédé à l'extraction des verbes et des cadres syntaxiques pour enrichir notre dictionnaire. Nous expliquons aussi comment nous avons créer le dictionnaire de patron de régime qui servira à notre réalisateur.Le tout a été effectué en Python. Puis, le cinquième chapitre décrit comment nous avons adatpé le réalisateur profond GenDR pour qu'il incorpore les données que nous avons extraite de VerbNet. Nous décrivons comment les règles de grammaire ont été adpatées ainsi que les dictionnaires. Finalement, le sixième chapitre renferme l'évaluation du système, ce qui consiste à expliquer comment nous procédons à l'évaluation puis nos critères et finalement ce que les données disent. Puis, pour conclure le tout, nous faisons une synthèse du travail en montrant la contribution de ce mémoire à l'état de l'art, puis nous évoquons quelques pistes à explorer.

\pagenumbering{arabic}