%!TEX root = ../memoire.tex

\chapter*{Introduction}

% \noindent c'est mieux de changer la définition de \chapter et autres titres pour que ça soit uniforme

mémoire porte sur : implémentation de VN dans un générateur de texte
sous branche du TAl, qu'est-ce que la gat
la gat répond à quels besoins, quels domaines
Bref explicatoni de la structure de la gat
La réalisation linguistique

pour réaliser du texte le plus naturel possible, il faut se doter de ressource traitant les verbes
parler de la problématique
\draft{Parler de la problématique: Verbes sont les noyaux des énoncés, nous voulons implémenter une ressource de verbes dans un réalisateur profond pour élargir sa couverture. }

finalement objectif de ce mémoire, doter un réalisateur de ces données linguistiques grâce à VerbNet
TST, GenDR, héritier de MARQUIS. Nous commençons par l'anglais, mais comme c'est un réalisateur multilingue, on pourrait couvrir large pour d'autres langues comme il existe des variantes de VerbNet. utiliser python pour l'implémenter

Organisation du mémoire 6 chapitres:  parler de chacun





\pagenumbering{arabic}

% Ici débute le corps de votre texte.

