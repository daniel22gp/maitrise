%!TEX root = ../memoire.tex

\chapter*{Introduction}

% \noindent c'est mieux de changer la définition de \chapter et autres titres pour que ça soit uniforme

\draft{
\FL{utilise toujours un format particulier pour des commentaires, parce que c'est très facile de pas voir un petit commentaire dans un mémoire, et ça a l'air fou si tu te ramasses avec ça dans ton texte.}
Parler de la génération de texte automatique, de\emph{VerbNet}, de la problématique (pas de consensus quant à l'architecture de la classe verbale en \ac{TAL}\todo{utilise le package acronym}) de VerbNet, des dictionnaires. Pourquoi les verbes sont si importants ? (Kipper,2005, dissertation) Puisque les verbes sont porteurs du sens principal de la phrase, il faudrait donc créer une ressource qui puisse rendre compte du sens des verbes si on souhaite une bon fonctionnement des applications NLP.
}

\draft{ta problématique n'arrive qu'à la p.38, c'est très loin et le lecteur va se tanner avant. il faut que tu l'explicite tout de suite dans ton intro. tu devras en parler dans des termes un peu vagues puisque t'as pas encore tout expliqué, mais tu peux le faire ici. ensuite, dans ta section 2.5, tu élabores et précises ce que tu voulais dire ici}

\pagenumbering{arabic}

% Ici débute le corps de votre texte.

