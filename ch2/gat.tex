\chapter{Génération automatique de texte}

Branche du TAL
but est de produire du texte automatiquement à partir de contenu pas en langue naturelle comme des données brutes, représentations, informations entreposées dans des bases de donnés (voir les inputs des différents systèmes)

Dans notre quotidien on voit des exemples de NLG tout simple, mais que faire quand on veut pousser la chose (voir la thèse intéressante). Les faciles sont des template-based text generation, on s'en fout. Ensuite y'a les rule-based generation et les trainable NLG

la gat créée pour répondre à des besoins
traduction automatique (système de Dorr)
résumés automatiques
QUestion answering
dialogue systems
utilisé dans des domaines variés

mais variété entraîne un manque d'uniformité dans l'architecture, tâche, type de données (cité Lambrey).
Reiter et Dale se sont penchés sur la question. Ont tenté de donner une ligne directive

Danlos et Reiter/Dale sur le processus de génération automatique de texte

Ils ont découpé les étapes importantes de la GAT de cette manière :

blabla bla

là-dedans, dans une étape importante, la réalisation linguistique, y'a 2 sous-étapes : lexicalisation et arborisation

ensuite la problématique : les verbes, les dictionnaires verbaux, pourquoi les verbes, l'avantage

couvrir large en GAT grâce aux verbes

\section{Contexte}

utilité, manque d'uniformité, avenir,etc.

\section{Processus}

voir les thèses ou ouvrages qui parlent de NLG en général

Dale et Reiter pensent qu'il y a 6 tâches

Content determination
Discourse planning
sentence aggregation
lexicalization
referring expression generation
linguistic realization

\section{Réalisation linguistique}

\subsection{Réalisateur surface}

\subsection{Réalisateur profond}
FORGe : très similaire à ce qu'on fait, est-ce que c'est problématique ? En quoi on se démarque d'eux ?
\subsubsection{GenDr:héritier de MARQUIS}

article de françois
article sur MARQUIS


\section{Lexicalisation}

\section{Les verbes}

gross : constructing lexicon-grammar
melcuk : ECD
la richesse des verbes et pas facile à faire en NLG
VerbNet a tenté de pallier à ça aussi

\subsection{problématique}

On n'est pas les seuls à penser que pour faire un bon rule-based grammar, on a besoin de ressources lexicales riches, dont des dictionnaires de sous-catégorisation pour être capable de générer toutes les phrases possibles en anglais (FORGE et l'article de towards large-coverage detailed lexical resources)

\subsection{les patrons de régime}

finir avec : il nous fallait trouver un dictionnaire capable de pallier à notre objectif, puis chapitre suivant
