%!TEX root = ../memoire.tex

\chapter{GenDR}

Generic Deep Realizer \citep{lareau18}
théorie sens-texte : trouver la meilleure citation pour ça

Le chapitre au complet sera dedié à GenDR, résumé de l'article de françois.
Cette partie incluera tous les aspects théoriques que nous devons expliquer. (lexicalisation (collocations, FL), arborisation, patrons de régime,etc.)
C'est l'ancienne version de GenDR, expliquée avec les règles et dictionnaires du tronc, et non de la branche.
présenter les limites de ce système et pourquoi nous voulions aller chercher l'aide d'une ressource comme VerbNet pour pallier à ce problème.

basé sur la TST (dans une récente étude, elle est mentionnée comme étant importante dans le domaine{Vicentegeneracionlenguajenatural2015}, fait ses preuves avec MARQUIS

les différentes approches, méthodes, la mode présente, nous allons dans une position plus traditionnelle pour les raisons  X vue dans le premier chapitre, nous allons utiliser ça : GenDR qui fonctionne avec la plate-forme MATE.

Suivre l'article de françois pour détailler cette partie.
Faire des graphiques en PPT pour exemplifier la chose.

GenDR c'est un réalisateur profond : faire un court retour sur ce que c'est.

parler de pourquoi ils utilisent la TST : tous les travaux répertoriés dans Zotero qui parle de TST, pour justifier l'emploi de MTT.

\section{Architecture de MATE : le transducteur de graphe}

\section{Module interface sémantique-syntaxe dans GenDR}

\section{Les dictionnaires}
\section{Les règles de grammaire}
\subsection{domain-independant rules}
\subsection{domain dependent rule : per language}

\section{Les verbes}
mentionner les dictionnaires qui disent que les verbes sont les plus importants
Expliquer c'est quoi un patron de régime/cadre de sous-catégorisation, valence, etc.
permettent de couvrir large
patrons de régime des verbes y sont encodés
problématique liée aux verbes
