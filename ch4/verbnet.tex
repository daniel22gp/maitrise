%!TEX root = ../memoire.tex

\chapter{VerbNet et les dictionnaires de verbes}

Avant de parler de VerbNet, nous expliquerons pourquoi nous l'avons choisi parmi tant de candidats possibles. Dans la section qui suit, nous ferons un bref survol de ces candidats. Nous avons regardé les composantes de: WordNet, FrameNet, XTAG, LCS, Comlex, Valex,et le VDE. Parmi ces dictionnaires, il y en a qui traitent d'autres parties du discours tandis que certains ne traitent que des verbes.

%%%%%%%%%%%%%%%%%%%%%%%%%%%%%%%%%%%%%%%%%%%%%%%%
% --------- D I C T I O N N A I R E S  ---------
%%%%%%%%%%%%%%%%%%%%%%%%%%%%%%%%%%%%%%%%%%%%%%%%

\section{WordNet}
Wordnet \citep{Fellbaum1998} est une base de données lexicales traitant les verbes, noms, adjectifs et adverbes de la langue anglaise. C'est un large réseau lexical qui ressemble à un thésaurus car il regroupe les lexies en fonction de leur sens. WordNet propose une désambiguïsation exhaustive des vocables et contient 21\,000 acceptions verbales \citep{MillerWordNetonlinelexical1990}. On réfère généralement à cette base de données comme un modèle relationnel du lexique car le sens des lexies est représenté en termes de ses relations avec d'autres lexies, qui peuvent être de nature lexicale ou conceptuelle. La principale relation lexicale est la synonymie, WordNet regroupant les mots en ensembles de synonymes, ou \emph{synsets}. Les principales relations conceptuelles sont celles d'hyponymie et d'hyperonymie. Ces relations lient des \emph{synsets} généraux à des \emph{synsets} plus spécifiques, et vice-versa. Ces relations permettent de représenter le lexique à la fois verticalement (relations conceptuelles) et horizontalement (relations lexicales). Le tout crée un grande toile lexicale.

Concrètement, WordNet fournit des définitions et des relations sémantiques entre les lexies mais ne fournit pas explicitement de données syntaxiques. Toutefois, la base de données se justifie en soulignant que les gloses qui servent à exemplifier l'emploi d'une lexie peuvent aussi servir à déduire le comportement syntaxique de celle-ci. Bref, cela fournirait implicitement des données sur la syntaxe des lexies de WordNet \citep{FellbaumLargescaleLexicographyDigital2014}. Comme nous voulions acquérir un dictionnaire verbal pour les patrons de régime, il nous fallait un dictionnaire qui explicite ce genre d'information lexicale, c'est pourquoi nous avons laissé de côté WordNet.

\section{FrameNet}
FrameNet est une base de données construite par extraction d'information sémantique et syntaxique à partir de corpus électroniques manuellement et automatiquement annotés \citep{FillmoreBackgroundFramenet2003a}. Son nom provient de la théorie linguistique de la \emph{Frame Semantics} \citep{BakerBerkeleyFrameNetProject1998}. Ce projet couvre les domaines suivants: santé, chance, perception, communication, transactions, temps, espace, corps, motion, étapes de la vie, contextes sociaux, émotions et cognition. Pour mieux comprendre le fonctionnement de FrameNet, nous devons présenter les bases de la théorie des \emph{Frame Semantics}. L'idée centrale derrière celle-ci est que le sens des lexies doit être décrit en termes de cadres sémantiques qui décrivent les intéractions sémantiques entre la lexie décrite et les participants de la situation dénotée (appelés \emph{frame elements}). Autrement dit, FrameNet analyse le sens des lexies en faisant appel aux contextes (cadres) dans lesquels ces mots apparaissent et en explicitant les propriétés syntaxiques de ces lexies.

Les cadres sémantiques et les lexies forment les unités de base de ce système. Ainsi, FrameNet désambiguïse ses entrées en les traitant par paire de cadre sémantique-lexie. De cette manière, on pourra différencier les sens d'une lexie donnée en fonction des cadres sémantiques qui lui correspondent.

Dans FrameNet, les patrons de valence sont explicitement identifiés, contrairement à WordNet, et contiennent de l'information sémantique et syntaxique. La partie sémantique est rendue par les \emph{frame elements}. La valence sémantique est aussi notée en termes de structure argumentale (à la manière des structures logiques). Pour ce qui est de l'information syntaxique, on décrit les arguments de la lexie en leur attribuant une partie du discours (groupe nominal, groupe prépositionnel, etc.) et une fonction grammaticale (sujet, objet, etc.). De plus, FrameNet a annoté les citations extraites des corpus pour démontrer comment les patrons de valence sont instanciés dans de véritables phrases.

Les données de FrameNet sont stockées dans une base de données relationnelle qui reflète les bases théoriques du projet, composée de trois modules: un dictionnaire des unités lexicales traitées, un dictionnaire des cadres sémantiques, et des exemples annotées manuellement. Finalement, nous n'avons pas pris FrameNet bien qu'il s'agisse d'un excellent candidat. En effet, cette base de données explicite très bien les comportements sémantiques et syntaxiques du lexique, mais contrairement à VerbNet, elle n'est pas hiérarchisée et elle ne rassemble pas en groupe les verbes se comportant identiquement. Importer FrameNet dans notre système l'aurait saturé d'information, bien que celles-ci aurait été très pertinentes.

\section{XTAG}

XTAG est une \ac{TAG} de l'anglais développée chez Xerox \citep{XTAGResearchGroupLexicalizedTreeAdjoining2001, ParoubekXTAGGraphicalWorkbench1992} qui offre des descriptions syntaxiques riches de 9\,000 verbes. Dans ce dictionnaire, chaque unité lexicale se voit assigner un ensemble d'arbres qui décrivent les constructions syntaxiques permises pour celle-ci.

En \ac{TAG}, les phénomènes linguistiques sont représentés par des arbres élémentaires qui doivent se combiner par unification \citep{JoshiTreeAdjoiningGrammars1997}. Dans le cas de lexies prédicatives, ces arbres élémentaires sont pourvus de branches au bout desquelles il y a des n\oe{}uds non lexicalisés (mais contraints, notamment quant à leur partie du discours) destinés à acceuillir les arguments qu'elles sélectionnent. Ce sont les contraintes sur les n\oe{}uds qui contrôlent l'unification. Il n'existe que deux types d'unifications: substitution et adjonction. La substitution permet à un arbre de se substituer à un n\oe{}ud pour le combler (par exemple, un arbre représentant un groupe nominal pourra se substituer à un n\oe{}ud nominal dans un autre arbre, s'il en satisfait toutes les contraintes). L'adjonction permet à un arbre de se joinder à un autre arbre sans substituer de n\oe{}ud. Par exemple, un arbre élémentaire de type adverbial pourra s'adjoindre à un arbre représentant un groupe verbal. La substitution modélise la complémentation, et l'adjonction modélise la modification.

Le dictionnaire XTAG organise l'information syntaxique en regroupant les arbres \ac{TAG} en familles. Un arbre individuel représente une construction syntaxique donnée, mais une famille d'arbres comprend toutes les variantes syntaxiques possibles pour un arbre canonique. Ainsi, dans XTAG, chaque lexie se voit attribuer un nombre de familles d'arbres. Grâce à ce mécanisme, XTAG n'a pas à lister tous les arbres possibles permis pour une lexie verbale \citep{DoranXTAGSystemWide1994}.

Bien que c'était un candidat prometteur, le formalisme dans lequel s'insère XTAG est beaucoup trop attaché à une théorie linguistique (\ac{TAG}). Cela nous aurait forcé à effectuer de nombreuses manipulations de données pour les convertir en \ac{TST}. Il nous fallait donc une ressource plus détachée d'une théorie linguistique, comme VerbNet.

\section{La base de données \ac{LCS}}
La base de données \ac{LCS} de \cite{DorrUseLexicalSemantics1992} s'est construite à partir de la théorie de \cite{Jackendoff1972-JACSII-2,JackendoffSemanticStructures1992}, qui argumente en faveur d'une décomposition sémantique des verbes. Ceux-ci sont décrits en termes de leur structure conceptuelle lexicale. Une LCS est un graphe sémantique dont les structures syntaxiques de surface en découlent. Ces graphes sont des représentations hiérarchiques non-linéaires composées d'une tête logique (la racine du graphe), d'un sujet logique (un seul), d'arguments logiques et de modificateurs logiques. En ce qui concerne le traitement des verbes dans LCS, le verbe est la racine du graphe et les arguments du verbe (sujet et objets) sont les arguments logiques liés à la racine.

Chaque n\oe{}ud des LCS ont trois attributs: type, primitif sémantique et champ. Ceux-ci permettent de contraindre les n\oe{}uds des graphes sémantiques pour que les unités lexicales sélectionnées soient les bonnes.

Une décomposition sémantique des verbes en termes de LCS explique leur propriété syntaxiques. Tel que \cite{verb-classes.levin.1993} l'avait perçu, les propriétés sémantiques des verbes influencent leur comportement syntaxique. À l'intérieur de ce cadre théorique, on pense que les verbes avec des LCS similaires partagent aussi des comportement syntaxiques comme des alternances de diathèses. La base de données LCS de \cite{DorrUseLexicalSemantics1992,AyanGeneratingParsingLexicon2002a} s'inspire fortement des travaux de Levin. Les verbes y sont rassemblés en classes verbales par le partage d'une structure LCS commune. Ainsi, tous les membres d'une classe partagent la même structure sémantique \citep{TraumGenerationLexicalConceptual2000}.

Bien que cette ressource ressemble beaucoup à VerbNet en termes d'architecture (héritage de \cite{verb-classes.levin.1993}), elle possède des lacunes. \ac{LCS} ne couvre pas aussi large que VerbNet en termes de cadres syntaxiques et la base de données ne désambiguïse pas les différents sens des verbes.
\FL{et comment s'appelle le dictionnaire de Dorr et al? Daniel: LCS database}

\section{Comlex}\label{comlex}

Comlex est une base de données lexicales développée pour l'anglais par \cite{Grishman:1994:CSB:991886.991931}. C'est dictionnaire syntaxique des verbes créé à des fins computationnelles, riche mais pas libre d'accès. Ses auteurs ont opté pour un système qui se voulait le plus neutre possible d'un point de vue théorique afin d'être utilisé par un grand nombre de systèmes. Ce dictionnaire ne traite pas uniquement les verbes, mais ce sont les 6\,000 entrées verbales qui nous intéressent ici. Comlex décrit pour chaque verbe les compléments possibles qu'il peut sélectionner et il explicite les attributs propres à certaines constructions (comme le choix d'une préposition). Les entrées lexicales ont été manuellement décrites parce que les auteurs du sytème ne croyaient pas que les sytèmes automatiques d'acquisition étaient capables de traiter correctement les verbes à faibles fréquences.

Nous n'avons pas choisi cette ressource d'abord parce qu'elle est payante, mais aussi parce que dans son évaluation, \cite{SchulerVerbnetBroadcoverageComprehensive2005} soulignait que Comlex ne fait pas la distinction entre les sens des verbes.

\section{Valex}
 
Valex est un dictionnaire de \ac{SCF} et \cite{Korhonenlargesubcategorizationlexicon2006} qui contient 6\,397 verbes de l'anglais. Elle a bâti son dictionnaire par acquition automatique. Contrairement à Comlex, que nous avons présenté plus haut, Korhonen stipule que les dictionnaires bâtis manuellement comportent naturellement plus d'erreursque ceux construits automatiquement. Elle souligne aussi que la méthode automatique est moins coûteuse en termes de temps et de ressource. De plus, elle suggère que les dictionnaires manuellement construits comportent une faille cruciale: le manque d'information statistique. Grâce aux informations statistiques acquises via le traitement de corpus, on a de l'information quant à la fréquence d'utilisation d'un \ac{SCF} pour un verbe donné. Finalement, elle souligne qu'en raison du nombre d'applications \ac{TAL} fonctionnant avec des méthodes probabilistes, la présence d'information statistique ajoute à leur bon fonctionnement. En ce qui nous concerne, notre application \ac{TAL} fonctionne à base de règles, donc les statistiques n'ajoutent rien à notre système. Toutefois, si nous le souhaitions, MATE \citep{BohnetDevelopmentEnvironmentMTTbased2000a,BOHNET10,bohnet07} pourrait tenir compte des statistiques lors de la réalisation linguistique, c'est une fonctionnalité qu'il possède.

Dans son article, Korhonen explique avoir utilisé le système d'acquisition de \cite{BriscoeSecondReleaseRASP2006}, qui se basent sur la méthode RASP. En bref, les \ac{SCF} sont extraits grâce au système RASP à partir de textes non-annotés. Ce système tokenise, étiquette puis lemmatise les données brutes. Ensuite, les \ac{SCF} sont extraits des phrases analysées du corpus (qui est composé de 5 corpus différents). Finalement, un filtrage est effectué pour se débarrasser du bruit. Le dictionnaire en ainsi construit automatiquement en récupérant les verbes des corpus ainsi que les \ac{SCF} qui leurs sont associés.

Dans Valex, une entrée lexicale comprend, entre autres, la combinaison d'un verbe et d'un SCF, la syntaxe des arguments et la fréquence d'utilisation du SCF. Bien que cette ressource soit intéressante, nous avons préféré nous tourner vers VerbNet en raison de son architecture hiérarchisée, qui nous est très utile. L'architecture de Valex ne nous permet pas de tirer profit du mécanisme d'héritage des traits que nous avons vu à la section \ref{dictio}, ce qui fait en sorte qu'en utilisant Valex, notre dictionnaire serait très lourd et saturé d'informations redondantes.

\section{Valency dictionary of English}
Le \acf{VDE} est un dictionnaire de valences de l'anglais de \cite{HerbstValencyDictionaryEnglish2004} qui contient les patrons de régime de 511 verbes (il traite aussi les noms et les adjectifs). Dans ce dictionnaire, chaque entrée décrit une valence possible pour un verbe. Le tout est accompagné d'un exemple provenant de la \emph{Bank of English} \draft{\citep{REF}}.

Les 511 verbes du VDE ont été choisis sur la base de leur fréquence dans la langue anglaise, de leurs propriétés complexes et de leur utilité pour des apprenants de l'anglais. Les patrons de valence qu'on retrouve dans le VDE proviennent d'une étude de corpus faite sur le COBUILD. 

Lors de sa création, le VDE n'était pas destiné à des applications TAL, mais les auteurs se sont rapidement rendu compte que ce dictionnaire pourrait intéresser les linguistes computationnels. Cela a entraîné la création de l'\emph{Erlangen Valency Pattern Bank} \citep{faucris.1039365}, une ressource électronique squi liste les patrons de valence identifié par le VDE. Les patrons y sont décrits en termes de syntaxe de surface. Le dictionnaire est divisé en deux: une liste des 511 verbes désambiguïsés et les patrons de valence leur étant associés dans un premier dictionnaire, et les patrons de valence de la langue anglaise dans un dictionnaire séparé.

Ce dictionnaire ne couvre que les verbes les plus fréquents, et nous cherchions une ressources avec une bonne couverture. Toutefois, il s'agit d'un travail manuel, donc on s'attend à ce qu'il ne comporte pas beaucoup d'erreurs, et on pourrait ainsi en extraire une partie pour complémenter le dictionnaire de VerbNet si tel est le besoin.

%%%%%%%%%%%%%%%%%%%%%%%%%%%%%%%%%%
% --------- V E R B N E T   -----
%%%%%%%%%%%%%%%%%%%%%%%%%%%%%%%%%%

\section{VerbNet}

VerbNet a été créé dans un contexte où il y avait un réel besoin pour un dictionnaire décrivant la richesse et la complexité des verbes \citep{KipperClassBasedConstructionVerb2000}. \cite{SchulerVerbnetBroadcoverageComprehensive2005} trouvait qu'il y avait un manque de lignes directrices par rapport à l'organisation des verbes dans les dictionnaires destinés à des applications \ac{TAL}, et c'est pour remédier à cela qu'elle a construit VerbNet. Son dictionnaire est organisé en une hiérarchie de classes verbales héritées de \cite{verb-classes.levin.1993}. Nous allons d'abord présenter ce classement avant de voir en détail VerbNet.

\subsection{Classes verbales de Levin}

\cite{verb-classes.levin.1993} proposait une méthode de classification des verbes qui a inspiré plusieurs dictionnaires, dont VerbNet \citep{SchulerVerbnetBroadcoverageComprehensive2005} et la LCS database \citep{AyanGeneratingParsingLexicon2002a,DorrUseLexicalSemantics1992}. Dans sa classification, les verbes de la langue anglaise sont placés dans un nombre fini de classes verbales. L'appartenance d'un verbe à l'une d'entre elles est motivée par le partage de comportements syntaxiques communs. Levin remarquait que tout locuteur natif est conscient des alternances de diathèses possibles pour un verbe, et ce sans avoir de connaissances méta-linguistiques préalables. Ainsi, en se basant sur son intuition, Levin a tenté de délimiter tous les patrons de régime possibles pour les verbes de la langue anglaise. Lorsque plusieurs présentaient des caractéristiques communes sur le plan syntaxique, elle rassemblait ces verbes dans une classe.

Bien que son travail s'insère dans le cadre de la syntaxe, elle supposait que les verbes qui se comportent de la même façon syntaxiquement possèdent probablement des propriétés sémantiques sous-jacentes communes. Toutefois, elle souligne que deux verbes en apparence synonymiques peuvent très bien appartenir à deux classes différentes, tout comme deux verbes qui, en apparence, ne se ressemblent pas du tout, peuvent appartenir à une même classe. Bref, le classement des verbes permettait de prouver sa théorie en plus de faciliter la classification des verbes de l'anglais.

Voici un exemple tiré de la thèse de \cite[pp.~12--13]{SchulerVerbnetBroadcoverageComprehensive2005}. On prend les verbes \lex{break} et \lex{cut}, et on teste diverses configurations possibles pour décider s'ils appartiennent à la même classe. À prime abord, on pourrait penser que c'est le cas puisque leurs signifiés se ressemblent. \sem{Break} et \sem{cut} partagent évidemment des composantes sémantiques car le sens d'altérer quelque chose est présent dans ces deux verbes. Cependant, les faits suivants nous démontrent qu'ils appartiennent à deux classes distinctes:

\ex. \label{ex:transitive} \emph{Transitive construction}
	\a. John broke the window.
	\b. John cut the bread.
	
\ex. \label{ex:middle} \emph{Middle construction}
	\a. Glass breaks easily.
	\b. This loaf cuts easily.
	
\ex. \label{ex:intransitive} \emph{Intransitive construction}
	\a. The window broke.
	\b. \ungr{The bread cut.}

\ex. \label{ex:conative} \emph{Conative construction}
	\a.\ungr{John broke at the window.}
	\b. John valiantly cut at the frozen loaf, but his knife was too dull to make a dent in it.

% (FL) JE SUIS RENDU ICI mais j'ai restructuré les prochaines sections

On voit d'abord que les constructions en \ref{ex:transitive} et en \ref{ex:middle} sont possibles pour ces deux verbes. Toutefois, en \ref{ex:intransitive} et en \ref{ex:conative}, on remarque qu'ils ne partagent pas ces cadres syntaxiques. \lex{Break} prend seulement la construction intransitive et exclut la conative, tandis que \lex{cut} prend la construction conative et exclut l'intransitive. Selon la logique de Levin, cela est dû à des différences de composantes sémantiques. Le verbe \lex{cut} décrit une série d'actions entreprises dans le but de séparer un objet en morceaux. Toutefois, il est possible de commencer à découper un objet sans que l'objet ne soit séparé. Dans ce scénario, on peut tout de même percevoir que l'objet a été découpé. En ce qui concerne \lex{break}, le changement d'état (le fait d'être séparé en morceau) est au c\oe{}ur même de l'évènement. Si on n'arrive pas au résultat final, une tentavive de briser quelque chose ne peut être perçue. 

Toutefois, nous devrons critiquer cette approche de Levin, car elle a omis de considérer un aspect important dans un traitement comme celui-ci. Dans l'exemple qu'elle nous fournit en \ref{ex:intransitive} et en \ref{ex:transitive}, \lex{break} n'a pas le même sens. Dans l'exemple \ref{ex:intransitive}, on pourrait traduire le sens de \lex{break} par \sem{se briser} tandis que le sens de \lex{break} dans le contexte de l'exemple \ref{ex:transitive} serait plutôt \sem{briser}. Cela a un impact direct sur la syntaxe, puisque le premier sens ne peut prendre qu'un seul argument, tandis que le second en prend nécessairement minimum deux. Cette lacune théorique de Levin est aussi présente dans VerbNet.

Bref, le projet de Levin a inspiré beaucoup de chercheurs, notamment l'équipe de VerbNet. C'est pourquoi ils ont repris une grande partie du travail de Levin dont l'organisation hiérarchique de VerbNet en classe et le regroupement des verbes en classes verbales. Toutefois, les auteurs de VerbNet ont retravaillé l'architecture de Levin et y ont apporté des corrections et améliorations \citep{verbnet.2006}.

%%%%%%%%%%%%%%%%%%%%%%%%%%%%%%%%%%%%%%%%%%%%%%%%%%%%%%%%%%%%%%%%%%%%%%%%%%%%%%
% --------- C O M P O S A N T E S    DE     V E R B N E T   -----
%%%%%%%%%%%%%%%%%%%%%%%%%%%%%%%%%%%%%%%%%%%%%%%%%%%%%%%%%%%%%%%%%%%%%%%%%%%%%%

\subsection {Composantes de VerbNet}  

Maintenant que nous avons présenté la contribution de Levin au développement de VerbNet, nous pouvons décrire les composantes de ce dictionnaire. Comme l'a fait Levin, il est aussi organisé en classes verbales. Toutefois, avant de l'implémenter, Schuler dut revisiter le classement initial de Levin puisqu'elle n'approuvait pas totalement du traitement de certaines entrées lexicales\citep{SchulerVerbnetBroadcoverageComprehensive2005}.

Chaque classe contient un ensemble de membres, une liste de rôles thématiques (accompagnés de restrictions sélectionnelles) utilisés pour décrire les arguments, puis un ensemble de cadres syntaxico-sémantiques. Chaque cadre est composé d'une brève description, suivi d'un exemple, puis d'une description syntaxique et sémantique\citep{SchulerVerbnetBroadcoverageComprehensive2005}.

\subsubsection{Classes verbales: organisation hiérarchique}

Les auteurs de VerbNet se sont fortement inspirés de Acquilex Lexical Knowledge Base \citep{CopestakeACQUILEXLKBrepresentation1992} pour l'organisation du lexique. Acquilex ordonnait l'information lexicale en hiérarchie. VerbNet a donc aussi implémenté un aspect hiérarchique à son dictionnaire en créant jusqu'à trois niveaux de profondeur pour organiser les classes verbales. 

Cela a entraîné la création des sous-classes. Celles-ci héritent de l'entièreté du contenu lexical de la classe qui la domine. Les sous-classes ont été créées pour spécifier qu'un sous-ensemble de verbes issus d'une classe mère démontrent des comportements syntaxiques différents du reste de la classe. Ceux-ci comprennent: les constructions syntaxiques, les prédicats sémantiques et les restrictions sélectionnelles sur les rôles thématiques \citep{SchulerVerbnetBroadcoverageComprehensive2005}. Prenons un exemple tiré de VerbNet pour illustrer cette hiérarchie à plusieurs niveaux \footnote{\url{https://verbs.colorado.edu/verb-index/VerbNet_Guidelines.pdf}, 15-02-18}.

\begin{minipage}{\linewidth}
\begin{lstlisting}[language=XML, caption = Hiérarchie, label=hierarch]
<VNCLASS ID="spray-9.7">
    <SUBCLASSES>
        <VNSUBCLASS ID="spray-9.7-1">
                <VNSUBCLASS ID="spray-9.7-1-1">
        <VNSUBCLASS ID="spray-9.7-2">
            <SUBCLASSES/>
        </VNSUBCLASS>
    </SUBCLASSES>
</VNCLASS>
\end{lstlisting}
\end{minipage}

\texttt{Spray-9.7} est le nom de la classe qui englobe toutes les autres ici. À l'intérieur de celle-ci, on spécifie tous les membres appartenant à cette classe, les rôles thématiques, les cadres syntaxiques et les prédicats sémantiques. Puis \texttt{Spray-9.7-1} représente un sous-ensemble de \texttt{Spray-9.7} dont les comportements syntaxiques lui sont propres .  Puis, \texttt{Spray-9.7-1-1} est une sous-classe d'une sous-classe, et ainsi de suite. Elle héritera des traits de sa classe mère ainsi que de la classe qui domine sa classe mère. Finalement \texttt{Spray-9.7-2} est la classe sœur de \texttt{Spray-9.7-1} donc, elle hérite aussi des traits de \texttt{Spray-9.7} mais ne partage pas les particularités de \texttt{Spray-9.7-1}.

Tel que démontré dans l'exemple \ref{hierarch}, les classes et sous-classes sont numérotées. Cette numérotation sert à expliciter la hiérarchie à l'intérieur d'une classe de VerbNet, mais elle sert aussi à regrouper des classes verbales en fonction de leur signifié. Cette numérotation est directement héritée du système de \cite{verb-classes.levin.1993}. Les nombres vont de 9 à 109 \footnote{\url{https://verbs.colorado.edu/verb-index/VerbNet_Guidelines.pdf}, 15-02-18}. Le numéro associé à une classe sert à représenter le partage de caractéristiques sémantiques (et syntaxique) entre les classes qui partagent ce numéro. Par exemple, les classes signifiant \sem{mettre quelque chose} commenceront par le chiffre 9:

\begin{easylist}[itemize]
  & \texttt{put 9.1}
	& \texttt{put spatial 9.2}
	& \texttt{funnel 9.3}
	& \texttt{put direction 9.4}
	& \texttt{pour 9.5}
	& \texttt{coil 9.6}
	& \texttt{spray 9.7}
	& \texttt{fill 9.8}
	& \texttt{butter 9.9}
	& \texttt{pocket 9.10}
	
\end{easylist}

\subsubsection{Membres}
Traditionnellement, les entrées lexicales dans un dictionnaire représentent un seul et unique verbe. En ce qui concerne VerbNet, les entrées sont des classes verbales regroupant  plusieurs verbes à la fois. Cela permet à VerbNet de couvrir largement l'anglais sans recourir à une quantité excédante d'entrées. Pour garnir leur section \emph{Members}, VerbNet a puisé dans les travaux de Levin \cite{verb-classes.levin.1993},dans la base de données LCS \citep{AyanGeneratingParsingLexicon2002a} et a mené sa propre enquête pour délimiter à quelle classe verbale un verbe appartient.

Concrètement, cette information est encodée directement dans les entrées lexicales de VerbNet en XML. La figure suivante \ref{membre} démontre à quoi ressemble la section \emph{Members}. Cet exemple nous démontre que \lex{deal, lend, loan, pass, peddle} et \lex{refund} sont les membres issus de la classe \texttt{give-13.1}.

\begin{minipage}{\linewidth}
\begin{lstlisting}[language=XML, caption = Les membres d'une classe, label=membre]
<VNCLASS ID="give-13.1" xmlns:xsi="http://www.w3.org/2001/XMLSchema-instance"
 xsi:noNamespaceSchemaLocation="vn_schema-3.xsd">
    <MEMBERS>
        <MEMBER name="deal" 
				wn="deal%2:40:01 deal%2:40:02 deal%2:40:07 deal%2:40:06" 
				grouping="deal.04"/>
        <MEMBER name="lend" 
				wn="lend%2:40:00" 
				grouping="lend.02"/>
        <MEMBER name="loan" 
				wn="loan%2:40:00" 
				grouping=""/>
        <MEMBER name="pass" 
				wn="pass%2:40:00 pass%2:40:01 pass%2:40:13 pass%2:38:04" 
				grouping="pass.04"/>
        <MEMBER name="peddle" 
				wn="peddle%2:40:00" 
				grouping="peddle.01"/>
        <MEMBER name="refund" 
				wn="refund%2:40:00" 
				grouping="refund.01"/>
        <MEMBER name="render" 
				wn="render%2:40:02 render%2:40:01 render%2:40:00 render%2:40:03" 
				grouping="render.02"/>
        <!--removed "trade" from class because doesn't take "to-PP"-->
        <!--removed "volunteer "from class because doesn't fit dative or-->
        <!--PP recipient PP frames-->
    </MEMBERS>
\end{lstlisting}
\end{minipage}

\subsubsection{Rôles thématiques}
VerbNet critiquait les autres dictionnaires verbaux qui n'offraient pas de contenu sémantique \citep{SchulerVerbnetBroadcoverageComprehensive2005}. C'est pourquoi ils font la promotion de leur aspect sémantique via l'emploi de rôles thématiques. VerbNet emploie 23 rôles thématiques pour identifier les arguments sélectionnés par les verbes dans chaque cadre syntaxique. Il existe d'autres approches dont la numérotation des arguments\emph{Arg-1 Verbe Arg-2} comme on le voit dans PropBank \citep{PalmerPropositionBankAnnotated2005}, mais Schuler considérait que l'usage des rôles thématique permettait d'ajouter de l'information de nature sémantique. Effectivement, l'assignation d'un rôle thématique à un argument nous donne de l'information quant sur le type d'argument nécessaire pour un verbe donné. À la base, les rôles thématiques ont été mis de l'avant par Fillmore \cite{fillmore:case} et Jackendoff \cite{Jackendoff1972-JACSII-2}. Toutefois, VerbNet a créé sa propre banque de rôles thématiques. Beaucoup sont inspirés de Fillmore et Jackendoff, mais certains ont été créés pour VerbNet. Les auteurs de VerbNet précisent donc que la quantité des rôles thématiques et la qualité des rôles thématiques est assez arbitraire. Il n'y a pas de justification théorique derrière ce chiffre, mais c'est ce qu'ils ont convenu d'utiliser.Schuler voulait des rôles pouvant identifier tous les arguments possibles contenus dans les patrons de régime. Donc, des rôles assez génériques pouvant se prêter à divers cadres.Ces rôles ne sont pas spécifiques à des classes en particulier

Voici la liste des rôles thématiques qu'ils ont choisis : \texttt{actor, agent, asset, attribute, beneficiary, cause, location, destination, source, experiencer, extent, goal, instrument, material, product, patient, predicate, recipient, stimulus, theme, time, topic}.

Les rôles thématiques sont listés dans la section \lstinline|<THEMROLES>| de chaque classe verbale. Une section \lstinline|<THEMROLES>| peut revenir dans une sous-classe lorsque celle-ci possèdent des rôles thématiques plus spécifiques à cette sous-classe de verbes. Ils sont ensuite mappés aux arguments dans les cadres syntaxiques et sémantiques (qu'on peut voir aux figure \ref{cadresynt} et , \ref{cadresem}).

\begin{minipage}{\linewidth}
\begin{lstlisting}[language=XML, caption = Les rôles thématiques] % Majuscule aux captions
    <THEMROLES>
        <THEMROLE type="Agent">
            <SELRESTRS logic="or">
                <SELRESTR Value="+" type="animate"/>
                <SELRESTR Value="+" type="organization"/>
            </SELRESTRS>
        </THEMROLE>
        <THEMROLE type="Theme">
            <SELRESTRS/>
        </THEMROLE>
        <THEMROLE type="Recipient">
            <SELRESTRS logic="or">
                <SELRESTR Value="+" type="animate"/>
                <SELRESTR Value="+" type="organization"/>
            </SELRESTRS>
        </THEMROLE>
    </THEMROLES>
\end{lstlisting}
\end{minipage}

Pour les besoins de notre travail, nous n'utilisons pas les rôles thématiques, mais nous voulions souligner qu'ils étaient importants pour les créateurs de VerbNet. Comme nous utilisons la théorie Sens-Texte dans notre réalisateur profond, les rôles thématiques n'ont pas leur place dans les patrons de régime que nous avons extraits de VerbNet. Pour plus de détails concernant la non-utilisation des rôles thématiques selon la TST, nous vous renvoyons à \cite[pp.~227--234]{mel2012semantics}.

\subsubsection{Restrictions sélectionnelles}
Les restrictions sélectionnelles s'ajoutent aux rôles thématiques. Ces traits imposent des contraintes aux arguments possibles pour un patron de régime donné. Dans l'exemple fourni ici, on remarquera que l'\texttt{Agent} doit être soit de type animé ou doit être une organisation.

\begin{minipage}{\linewidth}
\begin{lstlisting}[language=Xml, caption = Les restrictions sélectionnelles]
    <THEMROLES>
        <THEMROLE type="Agent">
            <SELRESTRS logic="or">
                <SELRESTR Value="+" type="animate"/>
                <SELRESTR Value="+" type="organization"/>
            </SELRESTRS>
        </THEMROLE>
\end{lstlisting}
\end{minipage}

\subsubsection{Cadres syntaxiques}

Voici maintenant la section qui nous intéressait le plus: les cadres syntaxiques. Ceux-ci sont compris dans la section \lstinline{<FRAMES>} de VerbNet. À l'intérieur de cette balise, on retrouve une autre balise se nommant \lstinline{<FRAME>} qui contient la balise \lstinline{<SYNTAX>}.

\lstinline{<SYNTAX>} nous donne de l'information de nature syntaxique (et sémantique via les rôles thématiques). Elle nous présente linéairement un patron de régime d'une classe verbale donnée. On précise linéairement car les syntagmes sont listés du haut vers le bas en fonction de leur ordre de syntaxe superficielle. Bien que le format présenté ici ne représente pas la syntaxe de surface en TST, nous pouvons quand même en retiré du contenu syntaxique et le mouler par la suite. À ce sujet, le maniement des cadres syntaxiques de VerbNet sera décrit dans le chapitre suivant \ref{ch:python}.

Nous avons choisi VerbNet car nous voulions un dictionnaire qui énumère exhaustivement tous les comportements syntaxiques possibles que démontre un verbe (une classe verbale dans ce cas-ci). Ainsi, cette section démontre explicitement comment le verbe se combine en surface, avec quel type d'argument et quelle préposition est sélectionnée.

Ce cadre syntaxique  \ref{cadresynt} provient de la classe verbale \texttt{give-13.1}. De plus, chaque cadre syntaxique est accompagné d'une phrase servant d'exemple. Ce cadre permet la réalisation de surface \form{They lent a bicycle to me}. \lex{they} est le \lstinline{<NP value="Agent">}, \lex{lend} est le \lstinline{<VERB/>}, \lex{bicycle} est le \lstinline{<NP value="Theme">} et \lex{me} est le \lstinline{<NP value="Recipient">}.

\begin{minipage}{\linewidth}
\begin{lstlisting}[language=Xml, caption = cadres syntaxiques, label=cadresynt]

            <SYNTAX>
                <NP value="Agent">
                    <SYNRESTRS/>
                </NP>
                <VERB/>
                <NP value="Theme">
                    <SYNRESTRS/>
                </NP>
                <PREP value="to">
                    <SELRESTRS/>
                </PREP>
                <NP value="Recipient">
                    <SYNRESTRS/>
                </NP>
            </SYNTAX>
\end{lstlisting}
\end{minipage}

\subsubsection{Prédicats sémantiques}
Dans la revue de littérature de la thèse de \cite{SchulerVerbnetBroadcoverageComprehensive2005}, on remarque que les dictionnaires servant de comparaison sont généralement critiqués par leur manque d'information sémantique. C'est pourquoi Schuler a insérer un segment sémantique à VerbNet : \lstinline{<SEMANTICS>}. Cette section est constituée d'une suite de prédicats sémantiques. Chaque prédicat est décrit par une liste d'arguments qui sont, à leur tour, décrits par deux caractéristiques: \emph{type} et \emph{value}.  Le cadre sémantique ci-dessous complémente le cadre syntaxique que nous venons d'exposer en \ref{cadresynt}. Ce cadre sémantique décrit à la fois \form{They lent a bicycle to me} et \form{They lent me a bicycle}.

\begin{minipage}{\linewidth}
\begin{lstlisting}[language=Xml, caption=Les prédicats sémantiques, label=cadresem]
<SEMANTICS>
                <PRED value="has_possession">
                    <ARGS>
                        <ARG type="Event" value="start(E)"/>
                        <ARG type="ThemRole" value="Agent"/>
                        <ARG type="ThemRole" value="Theme"/>
                    </ARGS>
                </PRED>
                <PRED value="has_possession">
                    <ARGS>
                        <ARG type="Event" value="end(E)"/>
                        <ARG type="ThemRole" value="Recipient"/>
                        <ARG type="ThemRole" value="Theme"/>
                    </ARGS>
                </PRED>
                <PRED value="transfer">
                    <ARGS>
                        <ARG type="Event" value="during(E)"/>
                        <ARG type="ThemRole" value="Theme"/>
                    </ARGS>
                </PRED>
                <PRED value="cause">
                    <ARGS>
                        <ARG type="ThemRole" value="Agent"/>
                        <ARG type="Event" value="E"/>
                    </ARGS>
                </PRED>
            </SEMANTICS>
\end{lstlisting}
\end{minipage}

Cette section met fin aux composantes de VerbNet. Pour plus d'informations, nous vous invitons à consulter la thèse de Schuler \cite{SchulerVerbnetBroadcoverageComprehensive2005} et le guide \footnote{\url{https://verbs.colorado.edu/verb-index/VerbNet_Guidelines.pdf}, 15-02-18}. Dans le chapitre suivant, nous expliquerons comment nous avons prélevé les informations que nous cherchions de VerbNet. Le tout est décrit en des scripts en Python qui seront expliqués en détails.

%%%%%%%%%%%%%%%%%%%%%%%%%%%%%%%%%%%%%%%%%%%%%%%%%%%%%%%%%%%%%%%%%%%%%%%%%%%%%%%%%%%%
% --------- A P P L I C A T I O N S     T A L      DE    V E R B N E T   ---------
%%%%%%%%%%%%%%%%%%%%%%%%%%%%%%%%%%%%%%%%%%%%%%%%%%%%%%%%%%%%%%%%%%%%%%%%%%%%%%%%%%%%

\section {Utilisation de VerbNet dans des applications \ac{TAL}}
Après avoir comparé les dictionnaires potentiels de la section~\ref{dictionconcu}, nous avons opté pour VerbNet. Toutefois, Nous voulions savoir dans quels contextes VerbNet était utilisé. Nous avons donc fait une recherche sur des applications \ac{TAL} pour vérifier à quelle fin la ressource lexicale VerbNet était employée. Ce que nous en retenons c'est que VerbNet est utiliser pour construire des graphes conceptuels automatiquement, faire de l'analyse sémantique, faire de la désambiguisation et comme base de données pour un système de question-réponse. Dans cette section nous expliquerons brièvement à quoi servait VerbNet dans chacune de ces tâches.

\textbf{Construction de graphes conceptuels}
Généralement, les graphes conceptuels servent à représenter le sens d'un énoncé. Hensman et Dunnion se sont servis des informations lexicales contenues dans VerbNet pour créer des graphes automatiquement\citep{HensmanAutomaticallyBuildingConceptual2004}. Pour ce faire, ils ont extrait des phrases de corpus, puis les ont analysés syntaxiquement. Par la suite, ils font correspondre le verbe trouvé dans la phrase à une entrée de VerbNet. Grâce à cela, ils peuvent automatiquement créer le graphe conceptuel en sachant comment les arguments de la phrase doivent se rattacher au verbe puisqu'ils ont dorénavant accès au patron de régime du verbe. Ces graphes sont ensuite ajoutés dans une base de données de graphes et facilitent l'extraction d'information.

\textbf{Analyse sémantique avec VerbNet, FrameNet et WordNet}
Dans ce projet, \cite{Shi:2005:PPT:2132047.2132058} combinent VerbNet à FrameNet et WordNet pour faire de l'analyse sémantique . La force de VerbNet dans ce projet est son exhaustivité des différents patrons de régime de l'anglais et l'identification des arguments par des rôles sémantiques. Les auteurs suggèrent que l'utilité de VerbNet soulignent que la couverture de l'anglais de VerbNet est cruciale pour ce genre d'opération. Concrètement, grâce à la combinaison de ces trois ressources, on peut faire ressortir la structure syntaxique et les composantes sémantiques d'un texte non-annoté.

\textbf{Un système de questions-réponses}
VerbNet est aussi utilisé dans des systèmes de question-réponse. D'ailleurs, pour qu'un tel système soit jugé bon, il se doit d'être extrêmement précis quant à la réponse qu'il produit. Cette précision découle directement de la richesse des ressources utilisées. Le système de \cite{DBLP:conf/nlpke/WenJH08} extrait les informations syntaxiques et sémantiques des questions, puis, il formule des réponses potentielles en fonction des informations extraites de la questions à partir du web. Finalement, VerbNet entre en jeu pour analyser la structure syntaxique des questions et réponses. Ensuite, leur système choisi la phrase candidate répondant le mieux à la question et tenant compte de la structure sémantique et syntaxique de la question.

\textbf{Un algorithme supervisé pour la désambiguisation des verbes}
\cite{AbendSupervisedAlgorithmVerb2008} a développé un modèle d'apprentissage supervisé pour mapper des lexèmes à des classes verbales de VerbNet. Comme la polysémie est un enjeu majeur en \ac{TAL}, son travail consistait à analyser un texte pour trouver les verbes. Par après, le système doit l'associer à la bonne classe. D'ailleurs, les auteurs soulignent que le développement de ce genre d'outil en \ac{TAL} est crucial car les systèmes informatiques ne voient les verbes que comme des chaînes de caractères dépourvues de sens.

%%%%%%%%%%%%%%%%%%%%%%%%%%%%%%%%%%%%%%%%
% --------- S Y N T H È S E  ---------
%%%%%%%%%%%%%%%%%%%%%%%%%%%%%%%%%%%%%%%%

\section{Synthèse}
\draft{refaire cette section}

Nous avons choisi VerbNet car c'est une ressource qui s'est distinguée de ses concurrentes. D'abord, par son imposante couverture de la langue anglaise 6394 acceptions de 4423 vocables. Ensuite, par son architecture interne héritée de Levin et améliorée. Le regroupement en classes verbales facilite énormément le travail, et le découpage hiérarchique suivi des mécanismes d'héritage en font un outil très performant et précis, sans être encombrant et facilement repérable. Les verbes sont aussi classés dans plusieurs classes verbales, donc ils sont partiellement désambigguisés, ce qui est extrêmement utile, car on veut être le plus précis possible en NLG. 

Effectivement, \cite{SchulerVerbnetBroadcoverageComprehensive2005} le présente très bien:
\begin{quote}
A variety of natural language tasks including machine translation, language generation (Dorr, 1997), document classifcation (Klavans and Kan, 1998), lexicography (San llppo, 1994), semantic role labeling (Gildea and Jurafsky, 2002), word sense disambiguation (Dang, 2004), and subcategorization acquisition (Korhonen, 2002) have benefited from the use of verb classes. Some of this work however has been done in a small scale and/or has dealt with restricted domains. A large-scale lexical resource that exploits the notion of verb classes is needed for real-world tasks.
\end{quote}
\vspace{-\baselineskip}
\hfill
\cite{SchulerVerbnetBroadcoverageComprehensive2005}

De plus, les descriptions syntaxiques encodées en XML sont facilement exportable et maléable dans un format qui nous convenait, par le fait même, le traitement en Python devenait très accessible puisque NLTK avait déjà fait un pré-traitement de VerbNet, ainsi il existait des modules dont nous pouvions nous inspirer pour extraire l'information dans les balises de VerbNet. Un autre point important est qu'il existe beaucoup de ressources linguistiques à la VerbNet dans d'autres langues que l'anglais, dont le (mettre les citations de tous les verbnet étrangers) français, le portugais, l'italien, l'estonien, l'espagnol, le catalan, le tchèque. Finalement, puisque notre système GenDR se veut un générateur multilingue dans sa première version, c'est un premier pas pour facilement implémenter d'autres langues à notre système puisque nous avons déjà des scripts une architecture prête pour acceuillir des ressources lexicales similaires. 

De plus, dans la section précédente, on montre qu'il est utilisé pour des applications de TAL diverses, mais on a aussi trouvé des systèmes récents qui utilisaient VerbNet pour faire de la NLG. Des chercheurs ont extrait le mapping qui avait été fait entre XTAG-VerbNet afin de donner une couverture imposante de l'anglais pour l'interface sémantique-syntaxe de son générateur. Ils se sont surtout servis de VerbNet pour créer leur grammaire. Leur projet s'appelle S-STRUCT \citep{PfeilAlgorithmsResourcesScalable2016}, un générateur de texte automatique dont on a ajouté un module d'apprentissage machine pour la partie \emph{discourse planning}. Ainsi, en améliorant la partie discourse planning avec du machine learning, le système apprend à générer les phrases dans un ordre statistiquement plus logique.

D'un autre côté, Wanner et Mille ont publié un court article expliquant qu'ils prévoyaient utiliser VerbNet comme base de données lexicales car ils avaient besoin d'une ressource lexicale riche dans le cadre de génération automatique de texte \citep{MilleLargeCoverageDetailed2015}. De telles ressources sont importantes pour la couverture et qualité des en GAT. Ils mentionnent que des réaliseurs classiques comme KPML, surge, et REALpro \FL{citations} nécessitaient un enrichissement lexical pour être encore plus performant. Car, pour arriver à une structure syntaxique de surface, où tous les unités lexicales sont réalisées, il faut qu'un système de GAT possède des ressources lexicales capables de générer tous les actants d'un verbe en fonction des restrictions possibles et des bonnes prépositions, il faut que la génération soit impeccable aussi. Ces informations se retrouvent dans des dictionnaires de cadres de sous-catégorisation tel que ceux mentionnés plus haut. En anglais, VerbNet rempli très bien les caractéristiques recherchées pour des dictionnaires lexicaux car c'est un lexicon à grande surface qui couvre une grande partie de la grammaire anglaise et détaille avec précisions les patrons de régime possibles, ainsi que beaucoup d'autres informations. Toutefois, ils précisent quelque chose qui est très vrai, VerbNet est principalement utilisé pour des tâches comme semantic role labelling, information retrieval. En résumé, les ressources lexicales éxistantes sur le marché sont incomplètes et difficile à implémenter pour la NLG. Ce qui a découlé de ce travail est Forge \citep{DBLP:conf/semeval/MilleCBW17}, un générateur de texte automatique. Toutefois, ils n'ont pas utilisé VerbNet dans leur système, tel que nous l'avons vu dans la section précédente. Ce qui démontre l'importance de ce travail car nous avons décidé d'utiliser VerbNet en tant que ressource lexicale pour faire la tâche qu'ils avaient entamés, mais sur laquelle ils n'ont pas publié de résultats.
