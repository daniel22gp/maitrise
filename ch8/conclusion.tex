%!TEX root = ../memoire.tex

\chapter*{Conclusion}

Problématique 9-10 lignes:

\draft{Problématique1: Il faut avoir des ressources lexicales riches décrivant les différents comportements syntaxiques des verbes car les verbes contrôlent la structure de la plupart des énoncés et leurs comportements sont très imprévisibles. Donc, en détenant les propriétés syntaxiques des verbes d'une langue donnée, on peut couvrir une grande partie des constructions de phrases possibles pour celle-ci. De là vient notre objectif, qui est d'intégrer au réalisateur profond GenDR un dictionnaire de comportements syntaxiques renfermant une quantité exhaustive des constructions possibles de l'anglais. De plus, comme GenDR se veut multilingue, si l'intégration est réussie, alors cela nous amènerait vers l'acquisition d'autres ressources lexicales similaires dans d'autres langues afin d'exploiter l'aspect multilingue de GenDR à pleine capacité. Pour clarifier l'expression \scare{comportements syntaxiques}, il s'agit des cooccurences syntaxique d'une unité lexicale donnée avec les arguments que cette lexie sélectionne. Par exemple, la relation entre un verbe et son sujet, ou bien la relation entre un nom et le complément qu'il sélectionne. La raison pour laquelle nous encodons généralement ces comportements dans des dictionnaires est du à l'aspect imprévisible de ces comportemetns. En effet, on ne peut pas prédire le nombre d'actants qu'un prédicat gouverne, ou bien les prépositions qu'il régit. ce qui est illustré par \cite{MilicevicSchemaregimepont2009}: \form{on se souvient de $X$}, mais \form{on se rappelle $X$}.

Problématique2: De plus, une autre raison qui nous a poussé à acquérir une telle ressource provient du fait que l'architecture présente de GenDR, le réalisateur que nous utilisons présente des limites quant à l'encodage de divers comportements syntaxiques pour un lexème.Pour un verbe donné, on ne peut pas avoir deux parties du discours différentes qui compétitionnent pour la même position syntaxique. Autrement dit, si nous voulions exhaustivement représenter les comportements du verbe \lex{want}, il nous faudrait un \ac{GP} qui puisse tenir compte du fait que le second actant syntaxique de ce verbe peut avoir une \ac{DPOS} de type verbale ou nominale: \form{I want to eat} vs \form{I want a dog}. Cela nous était impossible à encoder dans MATE avec les paramètres que nous avions car le système ne nous laissait pas donner deux versions de l'actant syntaxique II de \lex{want}.Nous nous sommes alors tourné vers l'idée d'ajouter un dictionnaire supplémentaire à notre ressource, qui encoderait tous les régimes existants de l'anglais. Puis, nous n'aurions qu'à encoder l'identification des \ac{GP} dans les unités lexicales appropriées ce qui nous permettait de contourner le problème des \acp{GP} multiples. Notre ancienne méthode restreignait conséquemment le nombre de réalisations possibles pour un verbe donné.}

Application TAL gagnerait à avoir plus de coverage, dont la GAT. Par exemple, GenDR

Rappeler l'objectif du mémoire:\draft{L'objet de ce mémoire est de pourvoir GenDR d'une couverture linguistique beaucoup plus grande que celle qu'il a présentement en y intégrant VerbNet qui est une base de données lexicales décrivant les comportements syntaxiques de 6\,394 verbes. On devrait pouvoir modéliser toutes les constructions possibles par une langue en ayant des descriptions très précises des verbes puisque les verbes contrôlent la plupart des énoncés. Nous voulions voir si l'implémentation d'une telle ressource était viable dans un réalisateur profond et si c'était faisable.}

Le mémoire répond à la problématique en faisant la tentative d'implémenter une telle ressource dans un réalisateur profond.

Chapitre 1: Fait le tour des différents réalisateurs
Chapitre 2: Étudié GenDR en profondeur
Chapitre 3: Regarder pourquoi VerbNet était gagnant
Chapitre 4: Extrait VerbNet
Chapitre 5: Implémenter la ressource dans GenDR : passe de 500 à 6394 verbes, passe de 4-5 patrons de régime généraux à 274	
Chapitre 6: Évaluer le rappel et la précision ,avec deux méthodes rappel est de x, précision est de x. GenDR performe à 100 pourcent partout, VerbNet excellent pour le rappel, mais moins top pour la précision, 

Contribution:
- comment encoder une telle ressource dans un système à base de règles: besoin d'un dictionnaire de patron de régime et la dynamique des verbes vers les classes c'est top
-on a évalué VerbNet en même temps
- on montre qu'avec une telle ressource, notre réalisateur est capable de générer presque toute les phrases input
- contribution en quelques semaines on a intégré ça à notre système et ça marche déjà très bien sans ajustement, avec les ajustements le système serait super performant
- on montre l'apport d'avoir une telle ressource en GAT, c'tait facile à implémenter et les résultats sont très bons
- pas tous les réalisateurs profonds qui intègrent ça

donc ça nous a permi à la fois d'implémenter une telle ressource dans notre système mais en plus de regarder si VerbNet s'implémente bien dans un réalisateur profond. Quelles sont ses limites

Pour encore plus améliorer le système, nous pourrions implémenter les régimes des noms, grâce aux autres ressources. Nous pourrions aussi aller chercher les gps manquant dans d'autres ressources que nous avons répertoriés. Maintenant qu'on sait comment s'y prendre, ça n'est plus trop compliqué. De plus, comme il existe des VerbNet dans d'autres langues, on aurait avantage à aller les chercher, ce sera facile vu qu'on a déjà le frame d'importation. Et comme GenDr est multilingue ça vient rejoindre le but.


