% N.B. : Assurez-vous de compiler ce fichier en employant "pdflatex" afin que les images soient incluses.

% Tout commentaire est bienvenu et devrait être adressé à "support@dms.umontreal.ca".

    % L'appel de \Requireackage{natbib} est fait dans dms.cls si \documentclass est appelé
    % avec l'option <natbib>.  Les options de <natbib> sont données ici.
    % Vous pouvez les modifier, p.ex. ajouter <round> si vous préférez des parenthèse (1) 
    % plutôt que des crochets [1].
\PassOptionsToPackage{longnamesfirst}{natbib}  
\documentclass[12pt,maitrise,frenchb,natbib,twoside,initial]{dms} % OPTION: policeTNR  pour Times New Roman
% La commande précédente charge la classe "dms.cls" avec les options suivantes : 
%   -police de caractères en 12 pts 
%   -format adapté à une thèse de doctorat 
%   -écriture française (par défaut avec la classe)
%   -impression recto-verso.
%   -adapté pour le dépôt initial (enlever l'option pour le dépôt final)
%
% Modifiez cette commande selon vos besoin à l'aide des options suivants :
% maitrise			mémoire de maîtrise;
% phd				thèse de doctorat;
% phdart                        thèse de doctorat par articles;
% rapport			rapport de stage;
% travaildirige			travail dirigé;
% oneside			impression recto;
% twoside			impression recto-verso.
% initial			depot initial (sans l'option pour depot final)
% policeTNR                     pour utiliser (l'équivalent de) Times New Roman (sinon <lmodern> est chargé, la fonte par défaut)
% nobabel			document en anglais seulement
% frenchb			document en français
% frenchb,english		document contenant du français et de l'anglais (utiliser \selectlanguage{} )

\usepackage[utf8]{inputenc}	% Fait dans le classe		
\usepackage[T1]{fontenc}

% La commande \sloppy peut avoir des effets étranges sur les
% lignes de certains paragraphes.  Dans ce cas, essayez \fussy
% qui suppresse les effets de \sloppy. (\fussy est le comportement par défaut.)
% On redéfinit \sloppy, pour tenter de réduire les comportements étranges.  
% Le seul changement apporté à la version originale est la valeur de \tolerance.
\def\sloppy{%
  \tolerance 500%  %9999 dans LaTeX ordinaire, mauvaise idée.
  \emergencystretch 3em%
  \hfuzz .5\csname p@\endcsname
  \vfuzz\hfuzz}
\sloppy  


\newcommand*{\footcite}[1]{\footnote{\cite{#1}}}
\usepackage[usenames,dvipsnames]{xcolor}
%\usepackage{courier}

\usepackage{graphicx,amsmath,amsfonts,amssymb,setspace,subfigure,color,icomma,dsfont}
% graphicx		Pour importer des images (PDF, JPG, PNG).
% amsmath		Écriture selon les normes de l'AMS.
% amsfonts		Fontes additionnelles de l'AMS.
% amssymb		Écriture des symbols de l'AMS.
% setspace		Permet de régler la distance interligne dans le document.
% subfigure		Simplifie l'inclusion de figures côtes-\`a-côtes.
% color			Pour l'utilisation de couleurs dans le texte.
% icomma		Reconnait la virgule comme caractère mathematique de facon intelligent
% dsfont		symboles mathématiques

\usepackage[pdfpagemode=UseNone,pdfstartview={XYZ null null null}]{hyperref}	% Cette extension permet l'insertion d'hyperliens dans votre document pdf.
 \definecolor{dark-red}{rgb}{0.4,0.15,0.15}					% Ici, trois couleurs sont définies et seront utilisées pour colorer les "hyperliens".
 \definecolor{dark-blue}{rgb}{0.15,0.15,0.4}
 \definecolor{medium-blue}{rgb}{0,0,0.5}
 \hypersetup{colorlinks,linkcolor={dark-red},citecolor={dark-blue},urlcolor={medium-blue}}
\usepackage{bookmark}  % Remédie à des petits problème de <hyperref> (important qu'il soit chargé après <hyperref>)
  % Enlever les commentaires et remplir cette section avec l'information pertinente.
  % Ceci ajoute des « méta-données » au pdf.  C'est optionnel, mais recommandé.
  % Vous pouvez voir ces méta-données en ouvrant un visionneur de pdf et en cherchant les
  % propriétés du pdf.  (Vous pouvez aussi tapez ' pdfinfo <nom-du-pdf> '  dans un terminal.)
  % Ces données sont utiles, par exemple, pour augmenter les chances qu'un algorithme de
  % recherche trouve votre document sur Internet, une fois diffusé.  Autrement dit, ceci
  % peut aider à augmenter la visibilité de votre travail.
\hypersetup{
   pdftitle = {Intégration de VerbNet dans un réalisateur profond}, 
   pdfauthor = {Daniel Galarreta-Piquette},
   pdfsubject = {Mémoire de maîtrise, Université de Montréal},
   pdfkeywords = {GAT, GATM, NLG, MNLG, régime des verbes, VerbNet, interface sémantique-syntaxe}
}

% Numérotation des équations par section et numérotation des tableaux et figures par chapitre.
\numberwithin{equation}{section}
\numberwithin{table}{chapter}
\numberwithin{figure}{chapter}

% Définition des environnements utiles pour un mémoire scientifique.
% 1	%FR
\newtheorem{cor}{Corollaire}[section]
\newtheorem{deff}{Définition}[section]
\newtheorem{exemple}{Exemple}[section]
\newtheorem{lemme}{Lemme}[section]
\newtheorem{prop}{Proposition}[section]
\newtheorem{rem}{Remarque}[section]
\newtheorem{theo}{Théor\`eme}[section]

% Si vous préférez que les corollaires, definitions, théor\`emes, etc. soient numérotés par le même compteur, utilisez plutôt ce bloc de commandes : 
%2	% FR
%\newtheorem{corollary}{Corollaire}[section]
%\newtheorem{definition}[corollary]{Définition}
%\newtheorem{example}[corollary]{Exemple}
%\newtheorem{lemma}[corollary]{Lemme}
%\newtheorem{proposition}[corollary]{Proposition}
%\newtheorem{remark}[corollary]{Remarque}
%\newtheorem{theorem}[corollary]{Théor\`eme}

%Pour faire des énumérations : easylist
\usepackage[ampersand]{easylist}



\onehalfspacing				% Fixe la distance interligne \`a "1.5". Pour une interligne double, utilisez plutôt "\doublespacing".

\allowdisplaybreaks			% Cette commande autorise LaTeX \`a briser les blocs d'équations, permettant ainsi une couverture plus uniforme des pages.

%Commande pour numéroter les tableaux en chiffres romains (préfixe: le numéro du chapitre)
\renewcommand{\thetable}{\thechapter. \Roman{table}}

\setlength{\parskip}{1ex plus 3pt minus 1pt}

% Commandes pour linguistique
\usepackage{lng,lng-acro-fr}

% Commandes générales
\usepackage{gen}

% Soulignement etc.
\usepackage{soul}

% Feedback
\newcommand{\FL}[1]{\draft{\textbf{(FL)}}\footnote{\draft{#1}}}

\usepackage{booktabs}
\usepackage{float}
\usepackage{longtable}

%lstlistings for writing code in Python and XML
\usepackage{listings}
\usepackage{color}
\usepackage{setspace}
\definecolor{Code}{rgb}{0,0,0}
\definecolor{Decorators}{rgb}{0.5,0.5,0.5}
\definecolor{Numbers}{rgb}{0.5,0,0}
\definecolor{MatchingBrackets}{rgb}{0.25,0.5,0.5}
\definecolor{Keywords}{rgb}{0,0,1}
\definecolor{self}{rgb}{0,0,0}
\definecolor{Strings}{rgb}{0,0.63,0}
\definecolor{Comments}{rgb}{0.38, 0.25, 0.32}
\definecolor{Backquotes}{rgb}{0,0,0}
\definecolor{Classname}{rgb}{0,0,0}
\definecolor{FunctionName}{rgb}{0,0,0}
\definecolor{Operators}{rgb}{0,0,0}
\definecolor{Background}{rgb}{0.98,0.98,0.98}
\definecolor{gray}{rgb}{0.4,0.4,0.4}
\definecolor{darkblue}{rgb}{0.0,0.0,0.6}
\definecolor{cyan}{rgb}{0.0,0.6,0.6}

\lstset{
  basicstyle=\ttfamily,
  % upquote=true,
  columns=fullflexible,
}
\lstdefinelanguage{mate}{
  basicstyle=\ttfamily\scriptsize\setstretch{1},
  columns=fullflexible,
  xleftmargin=1em,
  framextopmargin=2em,
  framexbottommargin=2em,
  showspaces=false,
  showtabs=false,
  showstringspaces=false,
  frame=l,
  tabsize=4,
  %float,
  emph={self},
}

\lstdefinelanguage{Python}{
  xleftmargin=1em,
  framextopmargin=2em,
  framexbottommargin=2em,
  showspaces=false,
  showtabs=false,
  showstringspaces=false,
  frame=l,
  tabsize=4,
  % Basic
  basicstyle=\ttfamily\scriptsize\setstretch{1},
  backgroundcolor=\color{Background},
  % Comments
  commentstyle=\color{Comments},
  % Strings
  stringstyle=\color{Strings},
  % keywords
  morekeywords={import,from,class,def,for,while,if,is,in,elif,else,not,and,or,print,break,continue,return,True,False,None,access,as,,del,except,exec,finally,global,import,lambda,pass,print,raise,try,assert},
  keywordstyle={\color{Keywords}\bfseries},
  % additional keywords
  morekeywords={[2]@invariant,pylab,numpy,np,scipy},
  keywordstyle={[2]\color{Decorators}\slshape},
  %float,
  emph={self},
  emphstyle={\color{self}\slshape},
}
\linespread{1.3}

\lstdefinelanguage{XML}
{
  morestring=[b]",
  morestring=[s]{>}{<},
  morecomment=[s]{<?}{?>},
  stringstyle=\color{black},
  identifierstyle=\color{darkblue},
  keywordstyle=\color{cyan},
  morekeywords={xmlns,version,type}% list your attributes here
  numbers=left,
  numberstyle=\footnotesize,
  numbersep=1em,
  xleftmargin=1em,
  framextopmargin=2em,
  framexbottommargin=2em,
  showspaces=false,
  showtabs=false,
  %float,
  showstringspaces=false,
  frame=l,
  tabsize=4,
  % Basic
  basicstyle=\ttfamily\scriptsize\setstretch{1},
  backgroundcolor=\color{Background},
}

\newcommand{\mate}[4]{
  \begin{figure}[htb]
    (PG:) #2

    (PD:) #3

    (COND:) #4
  \caption{#1}
  \end{figure}
	

}

%%%%%%%%%%%%%%%%%%%%%%%%%%%%%%%%%%%%%%%%%%%%%%%%%%%%%%%
% ---------  D É B U T  D U  D O C U M E N T  ---------
%%%%%%%%%%%%%%%%%%%%%%%%%%%%%%%%%%%%%%%%%%%%%%%%%%%%%%%

\begin{document}

% La commande "\brouillon" imprime, au bas de chaque page, la date ainsi que l'heure de la derni\`ere compilation de votre fichier.
%\brouillon            

%!TEX root = memoire.tex

% Voici les variables pour la création de votre page titre.

\title{Intégration %des régimes
 de VerbNet dans un réalisateur profond}
\author{Daniel Galarreta-Piquette}
\copyrightyear{2018}
\date{\today}									% Date de dépôt du document.
	% ces éléments ne doivent plus apparaittre selon les dierectives de la FESP
	% si toutefois vou souhaitez les inclure, il faudra aussi modifier le document dms.cls
% \president{Nom du président du jury}
% \directeur{Nom du directeur de recherche}
% \codirecteur{Nom du codirecteur}
% \membrejury{Nom du membre du jury} 
% \examinateur{Nom de l'examinateur externe}
% %\membresjury{ala, beta, gamma}
% %\plusmembresjury{psi, zeta, omega} 
% \repdoyen{Nom du représentant du doyen} 
\dateacceptation{Date d'acceptation}
\sujet{linguistique}							% Votre discipline de recherche, soit "mathématiques" ou "statistique".
%\orientation{mathématiques fondamentales}		% Cette commande est optionnelle. Les choix courants sont : "mathématiques fondamentales", "mathématiques de l'ingénieur" et "mathématiques appliquées".

\department{Département de traduction et linguistique}

% Fin des variables \`a définir. La commande "\maketitle" créera votre page titre.


\pagenumbering{roman}
\maketitle    
 

\chapter*{Sommaire} 	% La commande "\chapter*" crée un chapitre sans numéro, contrairement \`a la commande "\chapter" réguli\`ere.
% N.B. : La commande "\noindent" force LaTeX \`a ne pas indenter le nouveau paragraphe.
La \ac{GAT} a comme objectif de produire du texte compréhensible en langue naturelle à partir de données non-linguistiques. Les générateurs font essentiellement deux tâches, d'abord ils déterminent le contenu d'un message à énoncer puis ils sélectionnent les mots qui serviront à transmettre le message, aussi appellée la réalisation linguistique. Pour réaliser du texte le plus naturel possible, un système de \ac{GAT} doit se doter de ressources lexicales riches. Si on veut couvrir un maximum de de formes d'énoncés, il nous faut avoir accès aux différents comportements des unités lexicales d'une langue donnée. Puisque les verbes sont au c\oe{}r de chaque énoncé et qu'ils contrôlent généralement la structure d'une phrase, il faudrait encoder leurs propriétés afin de produire du texte représentant la richesse réelle des langues. De plus, les verbes sont imprévisibles en termes de comportements syntaxiques, c'est pourquoi il faut les encoder dans un dictionnaire. Par exemple, on ne peut pas prévoir que le verbe \lex{parler} demandera la préposition \lex{de} pour la phrase \form{Paul parle d'un évènement à Martine}, tandis que \lex{relater} n'en a pas besoin \form{Paul relate un évènement à Martine}. Cet exemple illustre que deux verbes synonymiques engendrent des constructions très différentes, arbitrairement. Il faut donc encoder ce savoir dans un dictionnaire pour que les applications \ac{TAL} comme la \ac{GAT} reproduisent ces comportements correctement.

Ce mémoire porte sur l'intégration de VerbNet, une ressource lexicale riche sur les verbes et leurs comportements syntaxiques, à un réalisateur profond GenDR, dans le cadre de la \ac{GAT}. Pour procéder à cette implémentation, nous avons utilisé le langage de programmation Python et le module \emph{xml.etree.cElementTree} pour extraire les données de VerbNet et les manipuler pour les adapter à GenDR un réalisateur profond opérant dans le cadre de la théorie Sens-Texte. Nous avons ainsi intégré 274 cadres syntaxiques à GenDR ainsi que 6\,394 verbes.

\textbf{Mots-clés}: génération automatique de texte; réalisation linguistique; patrons de régime; cadres syntaxiques; verbes; Théorie Sens-Texte; traitement automatique des langues; linguistique

\chapter*{Summary}



\textbf{Keywords}: natural language generation; linguistic realisation; government patterns; syntactic frames; Meaning-Text theory; linguistics; natural language processing
% TABLE DES MATIÈRES
\cleardoublepage
\pdfbookmark[chapter]{\contentsname}{toc}  %Crée un bouton sur la bar de navigation
\tableofcontents				% Table des mati\`eres.
% LISTE DES TABLEAUX
\cleardoublepage
\phantomsection
\listoftables
% LISTE DES FIGURES
\cleardoublepage
\phantomsection
\listoffigures	


%%%%%%%%%%%%%%%%%%%%%%%%%%%%%%%%%%%%%
%% LISTE DES SIGLES ET ABRÉVIATION %
%%%%%%%%%%%%%%%%%%%%%%%%%%%%%%%%%%%%%
%% Il est obligatoire, selon les directives de la FESP, 
%% pour une thèse ou un mémoire d'avoir une liste des sigles et 
%% des abréviations.  Si vous considérez que de telles listes ne seraient pas
%% pertinentes (si, par exemple, vous n'utilisez aucun sigle ou abré.), son
%% inclusion ou omission est laissé à votre discrétion.  En cas de doute,
%% parlez-en à votre directeur de recherche, le coadministrateur ou, ultimement, à
%% la FESP directement.
%%
%% Dans le cas où vous incluez une table des sigles et des abréviations,
%% vous pouvez enlever les % de la section suivante pour faire apparaître
%% un exemple d'une telle liste « fait à la main ».  Il existe des outils
%% plus sophistiqués si vous devez inclure une multitude de sigles et abréviations.
%% (Par exemple, le package <glossaries> peut faire des index élaborés.  Comme
%% son utilisation est technique, il n'y a pas d'exemple directement dans ce gabarit.
%% On invite les gens qui aurait à l'utiliser à consulter le wiki
%% du dms, le coadministrateur ou faire leur propre recherche.)

\chapter*{Liste des sigles et des abréviations}
\begingroup %Pour que le \renewcommand soit local
%Modifiez ce nombre (p.ex.remplacez 2 par 1.5) pour augmenter ou diminuer l'espace entre les lignes du tableau.
\renewcommand{\arraystretch}{2} 
\noindent\begin{tabular}{p{.2\textwidth} p{.7\textwidth}}
  GAT & Génération automatique de texte \\
  GP & Patron de régime, de l'anglais \textit{Government Pattern}\\
  DPOS  &  Partie du discours profonde, de l'anglais \textit{Deep Part of Speech}\\
  TST & Théorie Sens-Texte\\
  VN & \emph{VerbNet}\\
\end{tabular}
\endgroup  %Fin du groupe local pour \arraystretch



\chapter*{Remerciements}

\noindent Remerciements\dots

% Fin des pages liminaires. À partir d'ici, les premi\`eres pages des chapitres ne doivent pas être numérotées.

% Voici maintenant le chapitre d'introduction.
\NoChapterPageNumber 



%!TEX root = ../memoire.tex

\chapter*{Introduction}

% \noindent c'est mieux de changer la définition de \chapter et autres titres pour que ça soit uniforme

Dans les dernières années, les percées en intelligence artificielle ne cessent de fasciner la population. Plus particulièrement les applications liées au \ac{TAL} comme les systèmes de dialogue ordinateur-humain. Ainsi, énormément de recherche s'est orientée vers le développement de système informatique pouvant communiquer avec les humains. Or, généralement la communication implique deux processus: la compréhension du langage et la production de celle-ci. La deuxième composante de ce processus a donné naissance à la \ac{GAT} qui est une branche du \ac{TAL} dont le but est de produire du texte compréhensible en langue naturelles à partir de données non-linguistiques. Pour générer automatiquement des documents, ces systèmes se servent des connaissances linguistiques et des connaissances de domaine d'application. Les premiers systèmes de \ac{GAT} ayant vu le jour ont été créés pour produire des rapports automatiquement dans le but d'alléger la charge de travail des humains \citep{ReiterBuildingNaturalLanguage2000}. Autrement, l'écriture de ces documents est effectuée par un humain, ce qui entraîne des coûts continus. Ainsi, la \ac{GAT} est un domaine de recherche menant vers des applications concrètes et pouvant faciliter la tâche des humains. Que ce soit pour générer du texte ou de la parole, les capacités d'usage de tels systèmes sont infinies.

Par exemple, en termes d'applications concrètes la \ac{GAT} peut être utilisée pour produire des documents résumant des informations complexes pour des gens n'ayant pas les connaissances de bases requises pour les comprendre. Dans cette veine, MARQUIS \citep{WannerMARQUISGENERATIONUSERTAILORED2010} génère du texte à partir de données numériques brutes sur la qualité de l'air et les textes généres sont adaptés au lecteur. La \ac{GAT} a aussi fait une incursion dans le journalisme, ce qu'on appelle le robo-journalisme \citep{W17-3513}. Ainsi, des articles journalistiques peuvent être publiés à partir de données brutes dans le but d'informer les gens, sans qu'un humain soit derrière le texte, ce qui donne lieu à la rédaction d'articles sportif écrit par des machines.

Traditionnellement, un système de \ac{GAT} se divise en deux parties. La première étant ce que \cite{DanlosPresentationmodelegeneration1983} appelle le \emph{quoi-dire} et que \cite{gatt18} appellent le \emph{early process} qui équivaut à déterminer et structurer le contenu à générer. Autrement dit, il s'agit d'évaluer parmi les données s'offrant à nous, lesquelles voulons-nous transmettre et dans quel ordre. Ensuite il y a ce que Danlos appelle le \emph{comment-le-dire} puis que Gatt et Kramer appellent le \emph{late process} qui revient à choisir les unités lexicales qui serviront à transmettre le \emph{quoi-dire}, on appelle cela la réalisation linguistique. Cette étape du processus a fait l'oeuvre de nombre de recherche, entre autre parce qu'il y existe diverses approches pour l'effectuer. Nous nous concentrerons sur l'une d'entre elles: la réalisation linguistique à base de règles. Il s'agit de modéliser les connaissances linguistiques d'une langue donnée dans des règles de grammaire et des dictionnaires. Comme il existe une quantité faramineuse de mots, la plupart des réalisateurs profonds n'ont pas accès à tous les mots d'une langue ainsi que leurs propriétés lexico-syntaxiques. Toutefois, la majorité des catégories syntaxiques de mots présentent des régularités. À l'exception des verbes qui sont la partie du discours la plus imprévisible. Ainsi, si on veut qu'un réalisateur génère du texte le plus fidèle à la langue, on peut aller chercher les verbes qui sont la partie du discours qui contrôlent la plupart des énoncés et dont les comportemetns sont les moins prévisibles. Si on a ça, alors on obtient un système extrêmement flexible d'un point de vue linguistique.Pour clarifier l'expression \scare{comportements syntaxiques}, il s'agit des cooccurences syntaxique d'une unité lexicale donnée avec les arguments que cette lexie sélectionne. Par exemple, la relation entre un verbe et son sujet, ou bien la relation entre un nom et le complément qu'il sélectionne. La raison pour laquelle nous encodons généralement ces comportements dans des dictionnaires est du à l'aspect imprévisible de ces comportemetns. En effet, on ne peut pas prédire le nombre d'actants qu'un prédicat gouverne, ou bien les prépositions qu'il régit. ce qui est illustré par \cite{MilicevicSchemaregimepont2009}: \form{on se souvient de $X$}, mais \form{on se rappelle $X$}.

Donc, en détenant les propriétés syntaxiques des verbes d'une langue donnée, on peut couvrir une grande partie des constructions de phrases possibles pour celle-ci. \cite{SchulerVerbnetBroadcoverageComprehensive2005,Korhonenlargesubcategorizationlexicon2006} pensent qu'un meilleur traitement des langues naturelles passe par la connaissance des comportements des verbes. De là vient notre objectif, qui est d'intégrer au réalisateur profond GenDR un dictionnaire de comportements syntaxiques renfermant une quantité exhaustive des constructions possibles de l'anglais. GenDR, un réalisateur profond multilingue qui reprend les rouages de la réalisation du générateur MARQUIS, fonctionne avec la TST. GenDR a déjà fait l'objet d'un mémoire focusant sur sa capacité à rendre compte de phénomènes langagiers complexes: les collocations, GeCo. 

\draft{VerbNet a été créé dans un contexte où il y avait un réel besoin pour un dictionnaire décrivant la richesse et la complexité des verbes \citep{KipperClassBasedConstructionVerb2000}. \cite{SchulerVerbnetBroadcoverageComprehensive2005} trouvait qu'il y avait un manque de lignes directrices par rapport à l'organisation des verbes dans les dictionnaires destinés à des applications \ac{TAL}. Son dictionnaire est organisé en une hiérarchie de classes verbales héritées de \cite{verb-classes.levin.1993}. \cite{verb-classes.levin.1993} proposait une méthode de classification des verbes. Dans sa classification, les verbes de la langue anglaise sont placés dans un nombre fini de classes verbales. L'appartenance d'un verbe à l'une d'entre elles est motivée par le partage de comportements syntaxiques communs. Levin a tenté de délimiter tous les patrons de régime possibles pour les verbes de la langue anglaise. Lorsque plusieurs présentaient des caractéristiques communes sur le plan syntaxique, elle rassemblait ces verbes dans une classe.}

Pour implémenter VerbNet à GenDR, nous avons utilisé le langage de programmation Python qui nous a permi de manipuler et d'extraire les données de VerbNet. Manipuler pour que les informations marchent avec la TST. Pour compléter l'implémentation de cette ressource, nous avons du revoir certaines règles de grammaire et l'architecture de nos dictionnaires pour tenir compte des nouvelles informations lexicales auxquelles nous avons accès. Finalement, nous avons testé le tout pour vérifier si l'intégration de VerbNet fonctionnait.

Nous avons séparé ce mémoire en 6 chapitres dont les trois premiers chapitre correspondent à la préparation du projet et les trois derniers correspondent à l'expérience comme telle. Ainsi, le premier chapitre décrit ce qu'est la \ac{GAT} en expliquant les six étapes du processus classique de génération de texte, puis dans ce même chapitre, nous approfondissons une de ces étapes: la réalisation. Nous décrivons en détails ce qu'est la réalisation puis nous présentons quelques réalisateurs pour montrer les différences qui existent entre ceux-ci. Ensuite, le deuxième chapitre porte entièrement sur un réalisateur profond nommé GenDR: un réalisateur multilingue fonctionnant dans le cadre de la \ac{TST}. Nous décrivons comment ce réalisateur fait pour modéliser les phénomènes langagiers en expliquant en détails les modules qui le composent. Le troisième chapitre est dédié à la description de ressources lexicales focusant sur les comportements syntaxiques des verbes. Comme GenDR ne possède pas de telles ressources, nous avons fait des recherches pour trouver quel sera le meilleur et nous sommes tombés sur VerbNet. Nous décrirons plus en détails cette ressource lexicale.

Dans le quatrième chapitre, nous expliquons comment nous avons procédé à l'extraction des verbes et des cadres syntaxiques pour enrichir notre dictionnaire. Nous expliquons aussi comment nous avons créer le dictionnaire de patron de régime qui servira à notre réalisateur.Le tout a été effectué en Python. Puis, le cinquième chapitre décrit comment nous avons adatpé le réalisateur profond GenDR pour qu'il incorpore les données que nous avons extraite de VerbNet. Nous décrivons comment les règles de grammaire ont été adpatées ainsi que les dictionnaires. Finalement, le sixième chapitre renferme l'évaluation du système, ce qui consiste à expliquer comment nous procédons à l'évaluation puis nos critères et finalement ce que les données disent. Puis, pour conclure le tout, nous faisons une synthèse du travail en montrant la contribution de ce mémoire à l'état de l'art, puis nous évoquons quelques pistes à explorer.

\pagenumbering{arabic}
%!TEX root = ../memoire.tex

\chapter{Génération automatique de texte}

La génération automatique de texte découle de la branche qu'est le \ac{TAL}. Reiter et Dale \citep{ReiterBuildingNaturalLanguage2000} définissent cette branche comme étant un domaine à la croisée des chemins entre l'intelligence artificielle et la linguistique computationnelle. L'objectif est de développer des sytèmes pouvant produire du texte compréhensible en langue naturelle à partir de données non-linguistiques. Bien que l'objectif est commun à tous ces systèmes, les moyens pour s'y rendre sont de divers ordres. Entre autre car il existe des inputs de diverses natures : données, texte et images(Thomason et al., 2014). Ensuite car il y existe diverses approches de réalisation de texte : templates, règles, stochastiques \citep{gatt18}.

Toutefois, avant d'entrer dans les détails de la \ac{GAT}, il serait intéressant de mentionner l'origine de ces systèmes. À la base, ils ont été conçus pour, entre autre, générer des rapports automatiquement afin de faciliter le travail des humains. Effectivement, il est très utile pour certains de pouvoir lire un résumé de données numériques analysées par un système informatique. Tel que JSreal le mentionne, avec la GAT on peut présenter un résumé d'information provenant d'input numérique qui serait incompréhensible, mais utile pour un humain qui n'est pas un expert, et donc incapable de déchiffrer ces données. En plus d'être capable de résumé ces informations, le système informatique a l'avantage de fournir ces rapports sans se fatiguer à faire une tâche extrêmement monotone, qui pourrait être très coûteuse en termes de temps et de ressources pour des être humains.  D'ailleurs, pour qu'ils soient utiles, les textes générés automatiquement n'ont pas besoin d'être lus par une quantité phénomènale de gens. Dès qu'ils remplissent leur fonction, en étant utile à ne serait-ce que quelques personnes, leur raison d'être sera justifiée.  Dans cette même optique, il s'est développé des systèmes pouvant générer du texte en fonction de l'utilisateur. Ainsi, on pourrait générer un rapport X en fonction du professionnel, du technicien ou du client (Mahammood Reiter,2011) qui lira ce texte. Cette utilisation de la \ac{GAT} s'applique à de nombreux domaines dont les textes journalistiques. Les articles décrivant les matchs sportifs qui ne bénéficient pas de couverture médiatique (Van der Lee, 2017) sont des applications concrètes et utile du développement en \ac{GAT}. De même que des rapports générés automatiquement sur la qualité de l'air en fonction de l'utilisateur \citep{WannerMARQUISGENERATIONUSERTAILORED2010}. Ainsi, via le développement de ces systèmes, l'effacité des ordinateurs et de l'accessibilité aux données grandissante, on est capable de générer un article de journal automatiquement suivant les minutes où un évènement se produit(Oremus,2014). On parle là de robo-journalisme à son plein essor.

De nos jours la \ac{GAT} a grandement changé depuis que Dale et Reiter ont publié leur livre. Bien que ce soit le livre le plus complet entourant la NLG, le domaine a changé avec l'émergence de text-to-text generation et de vision-to-text generation(Hendricks et al., 2016a). qui reposent plus sur des méthodes statistiques que les modèles traditionnels de data-to-text qui étaient rule-based ou template based \citep{gatt18}. D'ailleurs, les frontières entre les diverses approches s'affaissent aussi avec le temps et nous sommes témoins d'une apparition constante de systèmes hybrides. Par exemple, des systèmes à base de patrons utilisant des règles de grammaire, des systèmes à base de règles utilisant des méthodes statistiques pour combler certaines tâches \citep{gatt18}. C'est donc un domaine très vaste qui, à ce jour, n'est pas uniforme. Ce qui est directement lié à une grande variété en NLG: divers types d'input (type de données), divers objectifs (tâches), diverses approches. Il est aussi important de préciser que la \ac{GAT} a aussi une valeur linguistique théorique. Il existe des linguistes qui testent leur théories via le développement de générateur automatique de texte. C'est effectivement une bonne manière de vérifier si un modèle théorique fonctionne\citep{DanlosPresentationmodelegeneration1983}. 

Dans le cadre de ce mémoire, notre projet est un réalisateur. La réalisation linguistique étant une étape du processus entourant la \ac{GAT}, nous ne travaillons que sur celle-ci. D'ailleurs notre réalisateur est créé avec des perspectives linguistiques. Ce qui exclut les systèmes à base de patrons et les systèmes statistiques, car ceux-ci ne nécessitent pas d'analyse linguistique pour le bon fonctionnement de la réalisation. Notre système se base sur des règles grammaticales. Cependant, avant de décrire notre projet, nous allons jeter les bases de la \ac{GAT} en décrivant le pipeline classique et en exposant quelques réalisateurs linguistiques.

\section{Pipeline classique GAT}

À la base, le pipeline classique observé par Dale et Reiter est un processus séquentiel séparé en diverses sections \citep{ReiterBuildingNaturalLanguage2000}. Traditionnellement, les six étapes suivantes sont les plus utilisées selon Dale et Reiter, comme illustré à la figure~\ref{fig:Pipeline}.
%les figures latex bougent, donc évite les formulations comme ça.
\begin{figure}[htb] % utilise toujours [htb]
	\centering
	\includegraphics[width=1\textwidth, trim = {0cm 0cm 0cm 0cm},clip]{ch2/figs/pipeline.pdf}
	\caption{Pipeline classique}
	\label{fig:Pipeline}
\end{figure}

\draft{Beaucoup s'entendent pour dire qu'il y a deux parties. Le \emph{quoi-dire} et le \emph{comment-le-dire} selon \citep{DanlosPresentationmodelegeneration1983}, puis le \emph{early process} et le \emph{late process}  selon \citep{gatt18}. Le quoi-dire (\emph{early process}) fait référence à blablabla et le comment-le-dire (\emph{late process}) fait référence à blabla.}

\subsection{Sélection du contenu}

Un système de \ac{GAT} doit sélectionner quelles informations seront inclues dans le texte en construction et quelles informations n'y seront pas. La sélection du texte dépend entre autre de l'objectif de la tâche (par exemple si un texte est adressé à un débutant ou un expert), ainsi typiquement , il y a plus d'information ou de détails dans les données que d'information que nous voulons transmettre en texte. Ainsi, la sélection de contenu implique des choix. Par exemple, s'il s'agit de données pour un match de soccer, on ne voudrait probablement pas réaliser toutes les passes et les fautes commises durant le match, bien que ces informations figurent dans les données en input. Il faut donc les filtrer et créer des représentations sémantiques de l'information qui sont souvent exprimé en représentation formelle du langage comme des représentations logiques, des base de données, des matrices ou des graphes.

\subsection{Structuration du document}
Après avoir sélectionné le contenu, un système NLG doit décider l'ordre dans lequel les informations seront présentées. Par exemple, si on utilise encore l'exemple du soccer, on commencerait par les informations générales liées au match (où et quand le match a eu lieu, entre quelles équipes, qui était blessé cette journée, etc), puis la description des buts comptés en ordre chronologique. Ce qui résulte de cette étape du processus est le plan d'un texte: une représentation ordonnée et structurée de messages à transmettre. Durant les dernières années, il y a eu des tentatives d'implémenter des méthodes d'apprentissage machine pour que la structuration de document se fasse automatiquement (Dimitromanolaki et Androutsopoulos,2003; Althaus et al., 2004)

\subsection{Agrégation}
Ce ne sont pas tous les messages sélectionnés dans le plan qui doivent être exprimés dans des phrases individuelles. En combinant des phrases individuelles en une seule et même phrase, on génère un texte beaucoup plus fluide et agréable à lire (Cheng et Mellish, 2000). Ainsi, la majeure partie du temps, elle sert à enlever la redondance dans le texte. Encore une fois, des chercheurs tentent d'automatiser cette étape en implémentant des méthodes d'apprentissage machine pour que le système NLG apprenne les règles d'agrégation et les applique lorsqu'il se doit (Barzilay et Lapata, 2006).

\subsection{Lexicalisation}
À cette étape, nous avons des données sélectionnées, puis structurées et que les futures phrases ont été combinées, on peut commencer à traduire les données en langue naturelle. Cette partie est très importante car c'est là qu'on choisi les mots qui seront utilisés pour transmettre le message. Toutefois, cette section se complique parfois car il existe naturellement plusieurs manières différentes de dire la même chose. La complexité de ce processus de lexicalisation dépend des mécanismes intégrés au système pour gérer cela. Certains traitent de lexicalisation en surface, d'autres la traite profondément. Toutefois, ceux qui traitent la lexicalisation en surface sont beaucoup plus restreints dans leur choix et sont très rigides. Tandis qu'un modèle profond permet de mieux tenir compte de la richesse lexicale d'une langue, mais il faudra une approche qui unit les représentations conceptuelles avec des règles de grammaire qui encoderont les choix lexicaux et syntaxiques. D'ailleurs, (Elhadad et al. 1997) avait grandement contribué pour cela.

Convertir un graphe sémantique d'entités et de relations, en un graphe syntaxique de mots et de relations syntaxiques. 

\subsection{Génération d'expressions réferentielles}
Souvent référée comme étape discriminatoire dont le but est de déterminer quelles informations doivent être générées pour qu'on puisse distinguer toutes les entités en jeu. Le but est de s'assurer que le lecteur peut idientifier chaque entité dans le texte. La meilleure manière de référer à une entité donnée.
Il s'agit de la tâche pour sélectionner des mots ou des phrases qui identifieront des entités. C'est une étape du processus qui a reçu beaucoup d'attention, car il n'y a toujours pas de consensus quant à la manière de faire et c'est extrêmement difficle. (Compléter avec les autres articles)

\subsection{Réalisation linguistique}
La dernière étape de ce pipeline NLG classique est la réalisation linguistique. Lorsque tous les mots, puis successivement les phrases ont été choisies, il faut les combiner en des phrases grammaticales. Cette tâche implique couramment l'application de traits morpho-syntaxiques et la linéarisation. De même qu'insérer les mots fonctionnels (auxiliaires, etc.) et la ponctuation. (Lambrey,2016) Objectif: appliquer règles et procédés d'ordre grammaticaux aux représentations abstraites pour que l'output satisfasse les contraintes syntaxiques et morphologiques de la langue en question.

\subsubsection{À base de patrons}
Est utilisé lorsque le domaine est petit et que les variations sont minimes. (McRoy et al., 2003). Montrer un exemple avec le soccer. (p.19).  L'avantage d'utiliser les templates est qu'on peut très bien prévoir ce qui sera généré en output, donc les risques d'erreurs grammaticales sont extrêmement faibles, du au contrôle qu'on a. Ils peuvent aussi être complémenter par des modules de règles pour pallier à certains problèmes, ce qui les rend très performant. Toutefois, l'inconvénient de tels systèmes est qu'ils sont très coûteux en termes de temps, puisque tout doit être fait à la main. Bien qu'avec la mode de l'apprentissage machine, certains systèmes apprennent à écrire des patrons (Angeli et al., 2012). Finalement, ils ne fonctionnent pas bien avec des tâches qui requièrent beaucoup de variation linguistique.

\subsubsection{À base de règles grammaticales}
en parler brièvement

\subsubsection{Statistique}

en parler brièvement et nommer quelques systèmes
mentionner que c'est à la mode


\subsubsection{Règles vs statistiques : avantages/inconvénients}
(Belz et Kow, 2009) et (Vicente et Al., 2015)
Depuis le début de cette section, nous disons qu'il y existe les méthodes staitsitques, et bcp vantent leur mérite, mais certains se sont penchés pour voir jusqu'à quel point ils sont bons. Dans cette partie, parler du blog de E.Reiter et l'article de Belz (si le temps, l'article de Vicente)

\draft{Mentionner le blog de E.Reiter concernant les approches ML et la NLG (rule-based est meilleur, mais on pourrait se servir de ML, sans toutefois rely dessus à 100\% comme certains chercheurs le font.
source (\url{https://ehudreiter.com/2016/12/12/nlg-and-ml})}

\section{Réalisation}

Tel qu'explicité dans le tableau~\ref{fig:Pipeline}, la réalisation est la dernière étape dans le processus de \ac{GAT}. Toutefois, pour beaucoup de chercheurs, elle ne représente pas toujours les tâches décrites précédemment. Il règne une ambiguité quant aux concepts qu'incarne la réalisation. Lambrey le mentionne \citep{LambreyImplementationcollocationspour2017} dans son mémoire, effectivement, \emph{réalisation} peut faire référence aux engins qui font la tâche de réalisation linguistique telle que décrite par le pipeline classique, mais elle réfère aussi aux engins qui englobent l'ensemble des tâches du module de réalisation linguistique. Pour effectuer celles-ci, ces engins partent d'un niveau plus abstrait que la réalisation dans le pipeline classique. C'est pourquoi nous faisons une distinction entre réalisateur de surface et réalisateur profond.Comme le mentionne aussi Dale et Essers 1998 \citep{EssersChoosingSurfaceRealiser1998}, puisque le pipeline est très long à construire en NLG, il n'est pas rare que des gens construisent leur système de NLG comme bon leur semble, mais qu'ils utilisent des réalisateurs déjà existant sur le marché. :The task of building a complete natural language generation nlg system is difficult and complex. Rather than start from scratch, it makes sense to make use of whatever reusable components are available, just as someone building a complete natural language analysis system would be likely to make use of an existing parser for syntactic analysis. In the context of nlg the state of the art is such that a number of reusable components are available for that subtask in the nlg process generally referred to as linguistic realisation.\draft{comment mettre une citation en LaTex : Comme Lavoie le mentionne \citep{LavoieFastPortableRealizer1997} : It is in the realizer that knowledge about the target language resides (syntax, morphology, idiosyncratic properties of lexical items). La nécessité de créer des réalisateurs provient du fait que Realization is fairly well understood both from a linguistic and from a computational point of view, and therefore most projects that use text generation do not include the realizer in the scope of their research.Instead, such projects use an off-the-shelf realizer like Surge.}

Polguère \citep{PolguerePourmodelestratifie} parle des différents modèles de GAT.

\subsection{Réalisateurs de surface}

Les réalisateurs de surface sont appelés ainsi car ils partent d'une représentation syntaxique et lexicalisée, donc peu abstraite en comparaison avec les réalisateurs profonds. Ils correspondent plus à la phase de réalisation linguistique classique mentionnée dans le pipeline classique de Dale et Reiter. Dans le sens où ils s'occupent de blablabla.

\subsubsection{SimpleNLG}
 \citep{GattSimpleNLGRealisationEngine2009}
Dans les années passés, un consensus s'est fait autour de la tâche de réalisation via RealPro, KPML, Surge. La réalisation implique deux tâches logiquement distinctes. La première est faire les bons choix linguisitques compte tenu de l'input sémantique donné. Puis une fois que c'est fait, construire la représentation syntaxique, faire la morphologie et linéariser les phrases sont des opérations assez mécaniques. Ainsi, Simple NLG est un engin de réalisation qui s'occupe des tâches plus mécaniques. Son rôle est de résumer un large volume de données numériques. Processus de réalisation : couvrir la syntaxe et la morphologie anglaise et la linéarisation. Simple NLG c'est une bibliothèque Java, qui fournit des interfaces offrant un contrôle direct sur le processus de réalisation (comment les phrases sont construites, fléchies, combinées et linéarisées).

Ce que SimpleNLG fait: défins un ensemble de types lexicaux et phrastiques correspondant aux catégories grammaticales majeures, il décrit aussi les manières de les combiner. SimpleNLG prend en entrée des constituants simples ou des morceaux de phrases préfabriquées.

Pour construire la structure syntaxique et la linéraiser, il faut faire les quatre étapes suivantes :
Premièrement, initilaiser les constitutants de base avec leurs unités lexicales correspondantes, deuxièmement déterminer les trais lexicaux des unités, troisièmement combiner les constituents en de plus grandes structures syntaxiques et quatrièmement, passer les structures résultantes au linéarisateur qui s'occupe de la linérisation et de mettre les formes fléchies aux unités lexicales en fonction des règles morphologiques.

Ainsi, il y existe un module lexical et un module syntaxique. Le module lexical comprend : le dictionnaire comme tel qui contient les unités lexicale, leur appartenance à une catégorie grammaticale, leurs propriétés syntaxiques et morphologiques. Le module syntaxique décrit les phénomènes morpho-syntaxiques et les règles qui permettent de créer des syntagmes jusqu'à créer une structure syntaxique complète d'un énoncé. Il se veut aussi un réaliseur facile d'utilisation, tel que son nom l'indique \citep{DaoustJSREALTextRealizer2015}.

Noter que SimpleNLG a été traduit dans plusieurs langues : espagnol, italien, et français, portugais (Mazzei et al., 2016, Ramos-Soto 2017, Vaudry ;Lapalme 2013 ; Oliveira et Sripada)

\subsubsection{JSreal} \citep{DaoustJSREALTextRealizer2015}

JSreal qui est en fait JavaScript Realiser. C'est un réaliseur de texte orienté pour les programmeurs web. C'est un réaliseur de texte en français qui génère des expressions et phrases bien formées qui elles peuvent être formattées en HTML pour être explosé dans un browser. Il peut aussi s'employer seul à des fins purement linguistiques ou être intégéré dans des générateurs de textes. Les Specs de JSreal sont similaires à ceux de SimpleNLG.

Pour générer du texte JSreal prend en input des arbres syntaxiques (écrits en java), qui seront ensuite parser en un arbre syntaxique, qui sera par la suite linéarisé. Pour générer du texte, JSreal possède les modules suivants : un lexicon, en ensemble de règles syntaxiques et un ensemble de règles morphologiques. Son dictionnaire défini la catégorie des mots qui le peuple, et les traits (genre, nombre, irrégularités, etc.). Les règles morphologiques : permet d'utiliser les bonnes formes fléchies. Les règles syntaxiques 

Florie ": JSreal prend en entrée des spécifications d'arbres syntaxiques en constituants immédiats écrites en fonctions JavaScript (JS). Les mots, syntagmes, etc. sont les « unités » manipulées. Les unités de base sont les catégories grammaticales (nom, verbe, etc.). Chacune déclenche l'application d'une fonction prenant un lemme comme argument et retournant un objet JS avec les propriétés et méthodes correspondants. Les syntagmes sont des unités déclenchant des fonctions prenant d'autres unités comme arguments. Les lemmes et leurs caractéristiques (traits sont répertoriés dans un dictionnaire à part. L'application des fonctions permet de construire une structure de données sous forme d'arbre regroupant toutes les propriétés des lemmes. Cet arbre est ensuite parcouru et fournit la liste des tokens de la phrase finale."

Mentionner qu'il existe aussi une version bilingue de JSreal \citep{MolinsJSrealBBilingualText2015}

\subsection{Réalisateurs profonds}

Ce qui caractérise les réalisateurs profonds : nécessite un théorie sous-jacente pour traiter les phénomènes, traitent l'interface sémantique-syntaxe,

\subsubsection{RealPro}
\citep{LavoieFastPortableRealizer1997}
RealPro est implémenté en C++, il peut donc se transposer à d'autres plateformes. Il prend en input des arbres de dépendances. Les connaissances syntaxiques et lexicales sont encodées dans un fichier ASCII, ce qui leur permet d'être mis-à-jour. Donc, il prend en input des strctures syntaxiques profondes, inspiré de la TST(melcuk). Les noeuds des arbres syntaxiques sont déjà lexicalisés, ce qui fait que ce système part d'une arbre syntaxique déjà lexicalisé et non d'un input sémantique. Les arcs qui lient les noeuds entre eux sont étiquettées par des relations syntaxiques. Ainsi, RealPro ne fait pas l'étape du choix lexical, donc la réalisation se fait à partir d'unités déjà lexicalisées. Ils ont préféré laisser la lexicalisation à un autre module. L'architecture de RealPro est basé sur la TST. Séquence de correspondance entre différents niveaux de représentations. Ainsi, pour passer de la structure profonde à la strucutre de surface, le logiciel vérifie dans son dictionnaire et ses règles de grammaires pour s'assurer que le passage à la seconde représentation syntaxique est correcte. Son linguistic Knowledge Base est réparti en divers modules dont  le lexique, les règles de grammaires. Ceux-ci seront réquisitionnés par les diverses composantes qui composent le pipeline de ce système. Voici en ordre, les composantes les plus importantes : Deep-syntactic component, Surface syntactic component, Deep-morphological component.

Voici un graphique représentant le pipeline de ce système :
\begin{figure}[h]
	\centering
	\includegraphics[width=1\textwidth, trim = {0cm 0cm 0cm 0cm},clip]{ch2/figs/realpro.pdf}
	\caption{RealPro}
	\label{fig:RealPro}
\end{figure}

Le dictionnaire contient de l'information quant à la partie du discours et les irrégularités morphologiques pour contourner les règles de grammaires dont celle qui ajoute \emph{-ed} à la fin des verbes au passé. Ainsi, cette information est encodée dans l'entrée de dictionnaire.

\begin{lstlisting}[language=Xml, caption=Entrée de dictionnaire]
LEXEME: SEE
CATEGORY: verb
MORPHOLOGY:[([mood:past-part] seen [inv])
            ([tense:past]     saw  [inv])]
\end{lstlisting}

\draft{je pourrais aussi ajouter la structure de dépendance pour montrer l'input de : This boy sees Mary.}

\subsubsection{KPML}
KMPL\citep{BatemanEnablingTechnologyMultilingual1997} est une extension multilingue du système de PENMAN (citation). C'est un système basé sur la grammaire fonctionnelle systémique (Halliday,1985). KPML contient trois éléments : un engin computationnel qui passe au travers de ressources grammaticales, une collection de ressources grammaticales, un environnement de développement pour écrire et débugger de nouvelles grammaires. Il s'agit d'un système puissant et complexe. Sa grammaire s'appelle NIGEL, il s'agit d'une grammaire de l'anglais. Toutefois, KPML se veut une ressource multilingue et elle couvre aussi des langues comme l'Allemand et le Néerlandais.

KPML prend des SPL en entrée. Il s'agit de \emph{Sentence Planning Language}. Afin d'illustrer à quoi ressemble l'input, nous allons nous servir de Dale et Reiter qui montre une phrase exemple 'March had some rainy days'.
\begin{lstlisting}[language=Xml, caption=SPL: input de KPML]
(S1 \ generalized-possession
  :tense past 
	:domain (N1 \ time-interval
	            :lex march
							:determiner zero)
	:range (N2 \ time-interval
	           :number plural
						 :lex day
						 :determiner some
						 :property ascription
						 (A1 \ quality :lex rainy)))
\end{lstlisting}
Compléter avec Florie

\subsubsection{MARQUIS}
\citep{WannerMARQUISGENERATIONUSERTAILORED2010}

\subsubsection{Forge: héritier de MARQUIS}
\citep{DBLP:conf/semeval/MilleCBW17}

\subsubsection{Surge}
\citep{Elhadad98surge:a}
Surge qui signifie \emph{Systemic Unification Realisation Grammar of English}, c'est une grammaire à grande couverture. Elle est écrite en FUF \emph{Functional Unification Formalism}. Il s'agit d'un langage de programmation créé pour construire des grammaires computationnelles, plus particulièrement pour les besoins de la réalisation linguistique dans un cadre de grammaire d'unification. Est basé sur FUG (Kay,1979) et ils se servent de graphes d'unification comme input à leur système. Cet input est sous forme de FD (Functional Description), il s'agit d'une collection de paires attributs-valeurs dont l'union fournit une spécification de l'énoncé à génerer. Chaque FD contient un trait CAT qui indique la catégorie syntaxique de la forme à produire. Les autres attributs dépendent de l'information requise par la grammaire pour générer une structure de cette catégorie

Par exemple, tel que trouvé dans Dale et Reiter, voici l'exemple d'une structure d'input pour 'March had some rainy days'.
\begin{lstlisting}[language=Xml, caption=FD: input de Surge]
((cat clause)
 (proc ((type possessive)))
 (tense past)
 (partic ((possessor ((cat proper) head ((lex "March"))))
					(possessed ((cat common) head ((lex day)))
											(describer ((lex rainy)))
											(selective yes) (number plural)))))
\end{lstlisting}

Compléter avec Florie

\subsection{Différences entre réalisateur de surface et profond}
Tel que Lambrey dans \citep{LambreyImplementationcollocationspour2017} l'a dit, les réalisateurs de surface sont plus facile à construire que les réalisateurs profonds. Entre autre parce qu'un RS est (à compléter). Tandis qu'un RP nécessite des linguistes pour développer l'application. Ces derniers sont plus complexes ce qui entraîne des coûts au niveau du temps et des ressources. Leur avantage réside dans le fait qu'ils sont dotés d'une profondeur linguistique leur permettant de faire l'analyse de phénomènes langagiers beaucoup plus complexes qu'un RS peut faire. 

Les RP présentés dans cette section encodent sans exception leurs connaissances linguistiques dans des dictionnaires et des grammaires. On remarquera que leurs différences majeures sont directement liées aux théories linguistiques sous-jacentes qu'ils utilisent pour modéliser la langue. Ainsi des réalisateurs comme Surge ou G-TAG (que nous n'avons pas décrit ici), ne possèdent pas de dictionnaires car leur module de règles contient des arbres partiellement lexicalisés. Tandis que MARQUIS, Forge, RealPro et KPML possèdent respectivement un module de règles grammaticales et des dictionnaires. Ce qui rend ces systèmes plus enclin à traiter plusieurs langues en même temps, et ce avec facilité. Un exemple de système qui fonctionne ainsi est GenDR \citep{lareau18}, un réalisateur profond multilingue qui possède respectivement un dictionnaire séparé de sa grammaire. Ainsi, le système peut générer du texte à partir d'un même input sémantique dans plusieurs langues en même temps en faisant appel à des règles de grammaires partagées (certaines règles sont spécifiques à chaque langue) et des dictionnaires propres à chaque langue.

Dans le cadre de ce mémoire, nous allons nous baser sur GenDR, une extension de MARQUIS. GenDR est un projet en cours de développement dirigé par François Lareau à l’Observatoire de linguistique Sens-Texte. Nous entrerons dans les détails au chapitre suivant.
%!TEX root = ../memoire.tex

\chapter{GenDR}\label{chapgendr}

GenDR (Generic Deep Realizer) est un réalisateur profond multilingue \citep{lareau18}. Il a hérité de l'architecture de MARQUIS. C'est-à-dire que GenDR modélise le langage dans des paramètres similaires, utilise le même transducteur de graphe et se sert de la même théorie linguistique. GenDR se démarque par sa capacité à traiter des phénomènes langagiers complexes en traitant l'interface sémantique-syntaxe en profondeur. Il offre une couverture beaucoup plus importante que MARQUIS en ce qui concerne les locutions, les collocations via un traitement large des fonctions lexicales. Cette couverture lexicale a été opérée par Lareau et Lambrey \cite{LambreyImplementationcollocationspour2017}, \cite{lambrey15}. Mais couvre la réalisation jusqu'à la syntaxe de surface, ne se rend pas jusqu'au bout.

Tel que mentionné, GenDR est une grammaire multilingue. Les règles grammaticales de base sont directement héritée de MARQUIS(citation) qui pouvait réaliser du texte en : catalan, anglais, français, polonais, portugais et espagnol. Ils n'ont gardé que les règles de base qui décrivent des phénomènes langagiers de base: lexicalisation simple, complementation, modification,etc. Ces règles forment le noyau du système et sont partagées (la majorité) par l'ensemble des langues du systèmes. Les règles spécifiques aux langues modélisent des phénomènes comme la sélection des auxiliaires, les déterminants, etc.

Le mapping entre les graphes sémantiques et les structures syntaxiques de surface se fait en deux étapes. Ces deux étapes sont effectuées par l'entremise de deux modules de règles différents opérant les interfaces respectives. Le module sémantique mappe les graphes sémantiques aux structures syntaxiques profondes correspondantes (Melcuk,2013), tandis que le module syntaxique mappe les structures syntaxiques profondes aux strucutres syntaxiques de surface. Cette architecture stratifiée est directement inspirée de la théorie Sens-Texte (melcuk et compagnie).

Ainsi le module sémantique contient 21 règles dont la plupart sont héritées de MARQUIS et 132 règles de lexicalisation \citep{LambreyImplementationcollocationspour2017}. Pour sa part, le module syntaxique contient nettement moins de règles. 20 règles dont 12 partagées entre les langues. Chaque règle modélise un phénomène langagier sur quoi la grammaire compte sur la richesse des dictionnaires.

%%%%%%%%%%%%%%%%%%%%%%%%%%%%%%%%%%%%%%%%%%%%%%%%%%%%%%%
% --------- A R C H I T E C T U R E  GENDR  ---
%%%%%%%%%%%%%%%%%%%%%%%%%%%%%%%%%%%%%%%%%%%%%%%%%%%%%%%
\section{Architecture de GenDR}

Dans la section suivante, nous allons brièvement présenter MATE, la plateforme sur laquelle GenDR fonctionne. Nous allons ensuite brièvement décrire la composante multilingue de GenDR.

\subsection{MATE}
\citep{Lareau2007TowardsAG}

MATE a été créé par Bernd Bohnet. Sa raison d'être provient du désir d'implémenter les couches de représentations à la Théorie Sens-Texte dans un transducteur de graphe. Ainsi, la transduction d'un graphe à un autre correspond au passage d'une couche de représentation à une autre. Ce logiciel permet aussi de tester, développer et maintenir la tâche qu'est la réalisation linguistique dans le cadre de la \ac{GAT} citep{BohnetDevelopmentEnvironmentMTTbased2000}. Afin de développer et tester le matériel, MATE comprend les composantes suivantes : un éditeur de dictionnaire, de graphes et de grammaires. Les dictionnaires encodent les unités sémantiques, lexicales et fonctionnelles. Les grammaires sont composées de règles modélisant le passage d'une représentation à une autre. C'est via les règles que la transduction de graphes se fait. L'éditeur de graphes permet de construire l'input et de le visualiser soit en format textuel ou graphique. Il y aussi un module d'inspecteur permettant de voir le déroulement de l'application des règles. Autrement dit, chaque étape de la transduction est explicitée et cela nous permet de voir quelles règles ont été utilisées ou quelles règles auraient dû être utilisées. Pour plus de détails concernant ce système, nous vous référons à ces articles \citep{BohnetDevelopmentEnvironmentMTTbased2000}, \citep{BohnetOpensourcegraph2010},\citep{LambreyImplementationcollocationspour2017}.

Nous allons maintenant montrer brièvement à quoi ressemble les éditeurs dictionnairiques, grammaticaux et graphiques.
%%%%%%%%%%%%%%%%%%%%%%%%%%%%%%%%%%%%%%%%%%%%%%%%%%%%%%%
% ---------D I C T I O N N A I R E  ------------------
%%%%%%%%%%%%%%%%%%%%%%%%%%%%%%%%%%%%%%%%%%%%%%%%%%%%%%%

\subsubsection{Dictionnaires}\label{dictio}
GenDR se sert de trois dictionnaires: un dictionnaire sémantique, un dictionnaire lexical et un dictionnaire de fonctions lexicales. Puisque les fonctions lexicales en génération automatique de texte ont déjà traitées par \cite{LambreyImplementationcollocationspour2017}, nous ne n'élaborerons pas sur ce dictionnaire. D'abord, le dictionnaire sémantique sert à encoder les unités sémantiques qui composent les noeuds des graphes d'entrées. Voici un entrée typique dans un dictionnaire sémantique (\emph{semanticon})

\begin{lstlisting}[language=Xml, caption = semanticon, label = semanticon]
owe { lex = owe
      lex = debt }
\end{lstlisting}
Ce dictionnaire mappe les sémantèmes à des lexèmes ou des locutions. Un même sémantème peut ainsi pointer vers deux lexèmes synonymes, mais aussi vers deux lexèmes signifiant la même chose, mais appartenant à des parties du discours différentes. Par exemple owe(sens) correspond à owe(lexème) un verbe, et debt(lexème) un nom.

Le dictionnaire lexical contient ce genre d'information:
\begin{lstlisting}[language=Xml, caption = lexicon]
verb_dit : verb_dt {                   // direct transitive
  gp = {
     III = {
        dpos = N
        rel = indir_objective
        prep = to  

     }
  }
}
[...]
owe : verb_dit {
  gp = { II = { dpos=Num } }
  gp = { II = { dpos=N } }
}
\end{lstlisting}
information détaillée à propos de chaque unité lexicale d'une langue. Les lexèmes et locutions ont leurs entrée dans ce dictionnaire. Une entrée a : PDD, diathèse, sous-catégorisation et collocations contrôlées. MATE incorpore aussi des mécanismes d'héritage facilitant l'encodage des ressources lexicales. Il est possible de créer des classes, et lorsqu'on pointe une unité vers une classe, elle hérite de tous les traits de cette classe. C'est de cette manière qu'on peut créer des classes de verbes. Par exemples, owe est un verbe ditransitif et il pointe vers la classe de verbe ditransitif. Cette classe a des traits X, mais elle hérite des traits de la classe verb transitif qui contient les traits communs à tous les verbes direct, puis cette classe hérite de verb qui contient les traits communs à tous les verbes. 

1500 mots les plus courants de fr et ang, puis quelques persan et lithuanien.

%%%%%%%%%%%%%%%%%%%%%%%%%%%%%%%%%%%%%%%%%%%%%%%%%%%%%%%
% ---------G R A M M A I R E  ------------------
%%%%%%%%%%%%%%%%%%%%%%%%%%%%%%%%%%%%%%%%%%%%%%%%%%%%%%%

\subsubsection{Grammaires}

\begin{figure}[htb]
	\centering
	\includegraphics[width=1\textwidth, trim = {0cm 0cm 0cm 0cm},clip]{ch3/figs/grammaire.png}
	\caption{Règle créant la racine syntaxique}
	\label{fig:root}
\end{figure}

Donc, en haut, on a : interface, nom de la règle et regroupement de règles. Le côté leftside est le point de départ de la transduction. Le côté droit est l'arrivée de la transduction. La partie en bas est les conditions. C'est là qu'on encode toutes les règles et dans ce format. Nous reviendrons aux règles dans la section exemple

%%%%%%%%%%%%%%%%%%%%%%%%%%%%%%%%%%%%%%%%%%%%%%%%
% ---------G R A P H E S ------------------
%%%%%%%%%%%%%%%%%%%%%%%%%%%%%%%%%%%%%%%%%%%%%%%%

\subsubsection{Graphes}\label{entree-sortie}
\draft{mettre les labels des figures au lieu de dire: figure suivante}
L'input de GenDR est un graphe sémantique de type TST. Dans celui-ci les prédicats sont liés à leurs arguments par des relations étiquettées avec des chiffres indiquant la position de l'argument. L'une des unités sémantiques doit être identifiée comme étant la plus saillante de l'énoncé. Il s'agit du noeud dominant de la phrase. Lors du transfert à la syntaxe profonde, il sera mappé comme racine syntaxique (le sommet de l'arbre).

Voici à quoi ressemble l'input :
\begin{lstlisting}[language=XML, caption = Input sémantique, label=input]
structure Sem debt {
S {
owe {
tense=PRES
1-> Paul {class=proper_noun}
2-> "\$500K" {class=amount}
3-> bank {number=SG definiteness=DEF}}
main-> owe}}
\end{lstlisting}

Mais pour simplifier la chose, nous montrerons une version graphique
\begin{figure}[htb]
	\centering
	\includegraphics[width=1\textwidth, trim = {0cm 4cm 0cm 3cm},clip]{ch3/figs/owe_sem.pdf}
	\caption{Graphe sémantique en visuel}
	\label{fig:graphesem}
\end{figure}

L'output de ce système consiste en un ensemble de structures de dépendances syntaxiques de surface. Ainsi, avec l'input sémantique en \ref{input} le réalisateur génère les six structures suivantes. Effectivement, c'est la preuve de la flexibilité du réalisateur GenDR. 

\begin{figure}[htb]
	\centering
	\includegraphics[width=0.5\textwidth, trim = {0cm 0cm 0cm 0cm},clip]{ch3/figs/exemples_real.png}
	\caption{6 réalisations syntaxe de surface}
	\label{fig:realsurf}
\end{figure}

Toutefois, dans cette figure, les arbres de dépendances ont été linéarisés pour faciliter la compréhension du lecteur. Dans les faits, les arbres générés en output ne sont pas linéarisés. Il s'agit toujours d'arbres de dépendances en syntaxe de surface. 

\begin{figure}[htb]
	\centering
	\includegraphics[width=1\textwidth, trim = {0cm 2cm 0cm 2cm},clip]{ch3/figs/realsurfex.pdf}
	\caption{Réalisation de surface}
	\label{fig:realsurfex}
\end{figure}

Si on souhaite une réalisation linéarisée,  il faudra passer par un réalisateur de surface. GenDR ne traite pas l'interface morpho-syntaxique et ne linéarise pas le texte. C'est dans ces contextes, que les réalisateurs de surface mentionnées précédement entrent en jeu. Les forces de GenDR concernent les tâches plus profondes de la réalisation, notamment l'arborisation et la lexicalisation.

%%%%%%%%%%%%%%%%%%%%%%%%%%%%%%%%%%%%%%%%%%%%%%%%%%%%%%%%%%%%%%%%%%%%%%%%%%%%%
% --------- I N T E R F A C E   S É M A N T I Q U E- S Y N T A X E ---------
%%%%%%%%%%%%%%%%%%%%%%%%%%%%%%%%%%%%%%%%%%%%%%%%%%%%%%%%%%%%%%%%%%%%%%%%%%%%%

\section{Interface sémantique-syntaxe en TST}

Ainsi, tel que mentionné, nous utilisons des graphes sémantiques en entrée dans ce système. Pour mieux comprendre de quoi il s'agit, nous ferons un retour sur les fondements de la théorie Sens-Texte

Pourquoi la TST : importante dans le domaine{Vicentegeneracionlenguajenatural2015} et a fait ses preuves avec MARQUIS, FORGe, RealPro.

Expliquer en quoi l'interface sémantique-syntaxe permet de réaliser des phénomènes linguistiques plus profonds que les réalisateurs de surface (peut-être faire un petit retour sur ce qui a été dit). Cette interface implique deux processus intimement liés : l'arborisation et la lexicalisation. 

%%%%%%%%%%%%%%%%%%%%%%%%%%%%%%%%%%%%%%%%%%%%
% ---------T H É O R I Q U E  ------
%%%%%%%%%%%%%%%%%%%%%%%%%%%%%%%%%%%%%%%%%%%

\subsection{Arborisation théorique}\label{arbo}
Pour mieux comprendre ce qu'est l'arborisation, nous ferons un court retour sur la Théorie Sens-Texte. C'est une théorie linguistique qui vise la description de la correspondance entre le Sens et le Texte (comme son nom l'indique). Celle-ci se fait via la construction de modèles formels \citep{PolgueretheorieSensTexte1998}. Le schéma \ref{fig:modeletst} présente comment on schématise le fonctionnement de ces modèles.

\begin{figure}[htb]
	\centering
	\includegraphics[width=1\textwidth, trim = {0cm 3cm 0cm 3cm},clip]{ch3/figs/polguere1.pdf}
	\caption{Structure d'un modèle TST}
	\label{fig:modeletst}
\end{figure}

La figure \ref{fig:modeletst} provient de Polguere 1998. illustre le fait qu'un modèle Sens-Texte est une machine virtuelle qui prend en entrée des représentations de sens d'énoncés et retourne en sortie un ensemble de Textes, qui contient toutes les paraphrases permettant d'exprimer le Sens donné en entrée. 

Tel qu'on l'a vu dans l'exemple en \ref{entree-sortie}. Ainsi, pour ce rendre au format de sortie, la représentation sémantique a subie diverses opérations qui lui ont permis de traverser divers niveaux de représentations. Pour illustrer ces représentations, nous vous renvoyons au tableau \ref{fig:processustst}.

\begin{figure}[htb]
	\centering
	\includegraphics[width=1\textwidth, trim = {0cm 0cm 0cm 0cm},clip]{ch3/figs/polguere2.pdf}
	\caption{Processus d'un modèle TST}
	\label{fig:processustst}
\end{figure}

Dans le cadre de ce travail, nous ne nous intérresserons qu'aux niveaux de rep sémantique jusqu'à syntaxe de surface. Car, comme nous l'avons mentionné précédemment, GenDR est un réalisateur profond qui opère uniquement dans ces niveaux de représentations. C'est pourquoi nous allons expliquer plus en détails de le transfert d'information entre ces niveaux de représentations et les modules linguistiques requis pour passer de RSem à RSyntP, puis à RSyntS.

Selon la TST, la structure syntaxique d'un énoncé représente l'ensemble des liens de dependance fonctionnelle qui existe entre les unités lexicales de cet énoncé. On représente formellement ces structures syntaxiques par des arbres qui sont caractérisés comme des arbres de dépendances. Cette approche syntaxique provient de Tesnière 1965. Bref, comme on l'a montré dans le schéma en \ref{fig:processustst}, il y existe deux niveaux de représentation syntaxique en TST : la RSyntP et la RSyntS. Pour mieux les comprendre nous les décrirons. D'abord, la RSyntP prend son origine dans la RSem. Ainsi, les relations et les lexies qui existent en RSyntP sont des conséquences directes d'une transition de la RSem. Formellement, le passage de la RSem à la RSyntP se nomme l'arborisation. Car on cherche à arboriser la RSem et cela donne en conséquence un arbre de dépendance en RSyntP.Ce passage est effectuée via des règles de correspondance sémantiques. Ces règles sont la partie 'grammaire' du schéma montré plutôt à la figure \ref{fig:modeletst}.

Pour que l'arborisation se fasse correctement, la première étape est d'identifier l'unité sémantique qui correspondera à la racine de l'arbre syntaxique profond. Une fois que nous avons cette racine (la hiérarchisation) \citep{PolguereStructurationmisejeu1990}. Cette hiérarchisation est aussi prise en compte par le système de règles de correspondances sémantiques. Une fois que cette étape est fait, on va partir de ce noeud en sémantique pour lexicaliser les noeuds qui lui sont liés et par le fait même réaliser les règles actancielles qui permettent la transition de la sémantique à la syntaxe des relations prédicats-arguments. citation"chaque arc est considéré successivement dans l'ordre du parcours, puis est traduit en une micro-structure syntaxique profonde grâce aux règles de correspondance sémantique de la grammaire." p.273 de \citep{PolguereStructurationmisejeu1990}. 


\draft{- construire le sommet de l'arbre syntaxique profond, qui correspond au ND de la RSémR
- construire récursivement toutes les relations actancielles syntaxiques dépendant du sommet de l'arbre.}

Une fois que l'arborisation est complétée, l'arbre de dépendance profond transitera vers un niveau de surface. Cette transition implique les opérations suivantes. D'abord, faire le calcul des relations syntaxiques de surface. Ainsi, la relation I deviendra sujet, la relation II deviendra complément d'objet direct, etc. Pour une meilleure compréhension, nous ferons un exemple plus bas. On incorpore à ce stade les lexies vides (prépositions, déterminants, etc.). La pronominalisation fait aussi partie de cette étape. Ainsi que la construction des structures communicative, anaphorique et prosodique du niveau syntaxique de surface. Mais, ces deux étapes ne sont pas effectuée dans GenDR.

\subsection{lexicalisation théorique}

La lexicalisation se passe entre le niveau RSem et la RSyntP \citep{PolguereStructurationmisejeu1990}. Le but est de choisir les unités sémantiques qui seront réalisées en syntaxe profonde en tant que lexèmes (unité lexicale). La lexicalisation est ainsi très liée à l'arborisation car ce sont deux étapes qui se produisent lors du passage de la RSem à la RSyntP. D'ailleurs, ces deux étapes se produisent simultanément. Effectivement la construction de la représentation syntaxique profonde est une conséquence de l'arborisation des liens de la représentation sémantique et de la lexicalisation des unités sémantiques. Ce qui résulte en un arbre de dépendance lexicalisé.

%%%%%%%%%%%%%%%%%%%%%%%%%%%%%%%%%%%%%%%%%%%%%%%%%%%%%
% --------- C O M P U T A T I O N N E L L E  ------
%%%%%%%%%%%%%%%%%%%%%%%%%%%%%%%%%%%%%%%%%%%%%%%%%%%%

\section{Arborisation et lexicalisation computationnelle}

Maintenant que nous avons présenté le côté théorique de l'interface sémantique-syntaxe, nous exposerons comment ces mécanismes se transposent dans GenDR.

\subsection{Arborisation computationnelle}
Tel que mentionné à la \draft{section \ref{arbo}}, on identifiait un noeud dans la structure d'input comme étant le noeud principal. Il faut faire cela car un graphe sémantique n'a naturellement pas de point d'ancrage. Or, si on veut générer une structure où l'un des sens est le noeud principal, il faut l'indiquer dans la structure d'input. Par exemple, si on voulait réaliser l'input sémantique suivant sans préciser le noeud principal \draft{reprendre l'exemple de françois}. On peut ainsi laisser le soin à GenDR de choisir le noeud principal, mais en règle général, on veut contrôler la structure communicative de l'output. L'arbre syntaxique profond est construit avec un algorithme top-down. Le top étant le noeud identifié comme noeud principal. L'arbre syntaxique est ainsi construit en se basant sur le graphe sémantique. L'algo est ressemble beaucoup à celui de MARQUIS et FORGe puisqu'ils sont tous des tenants de la TST. 

L'arborisation dans GenDR se fait en trois étapes. Nous les décrirons à la section suivante.

\subsubsection{Règles de base pour l'arborisation}

1.root\_standard On construit la racine de l'arbre syntaxique et on la fait correspondre au noeud principal de la structure sémantique. À cette étape, on crée seulement le noeud sans le lexicaliser. Mais on lui ajoute des contraintes. La contrainte principale est qu'il doit s'agir d'un verbe avec le mood indicatif (bien que des règles autres pourraient produire contrôler des alternatives). Cette étape n'arrive qu'une fois par graphe, par après, on ne se sert plus de celle-ci.

2. Une fois que le noeud a été créé et contraint en syntaxe profonde, on cherche dans notre grammaire pour une règle de lexicalisation qui peut satisfaire les contraintes, puis on regarde dans le dictionnaire pour un lexème qui correspond au sens demandé et aux contraintes demandées.

3.actant\_gp, actant\_guess, attr\_lex. 
Une fois que le noeud racine a été lexicalisée dans la structure syntaxique profonde, on regarde les arcs en relation avec le noeud sémantique principal. Donc on retourne voir dans le graphe sémantique. Les arcs qui partent du noeud sémantique pointant vers ses arguments sont réalisés comme des compléments. Tandis que les arcs qui pointent vers le noeud seront réalisés comme des modificateurs. En syntaxe profonde cette relation sera réalisée comme un dépendant du noeud étiquetté ATTR. Si l'argument réalisé est un complément, on doit regarder dans le patron de régime du gouverneur dans le dictionnaire. Le GP donne de l'information à propos du mapping sémantique, syntaxique profond, syntaxique surface des actants d'un lexème. Le GP spécifie aussi la partie du discours des compléments sélectionnés et les prépositions si nécessaires. Cette étape crée de nouveaux noeuds, mais pas lexicalisés, mais qui ont des contraintes. Pour chacun de ces noeuds, on retourne à l'étape 2, puis cycliquement, nous retournons à l'étape 3,  jusqu'à ce que le graphe sémantique soit complètement réalisé en surface profonde.

\subsection{Lexicalisation computationnelle}

La lexicalisation dans GenDR implique trois niveaux de représentations : la sémantique, syntaxe profonde, syntaxe de surface. La première étape est de prendre une unité lexicale profonde servant à exprimer un sémantème donné. Ça c'est la lexicalisation profonde. Elle introduit des mots pleins de sens et des verbes supports. Puis les lexèmes de surface sont choisis pour exprimer les unités lexicales profondes. Il s'agit de la lexicalisation superficielle. Celle-ci introduit les mots fonctions.

GenDR performe 6 types de lexicalisation. Celles-ci sont produites par l'intéraction des règles et des dictionnaires : lexicalisation simple pour les lexèmes, lexicalisation de patron pour les idiômes, bound lexicalisation pour les collocations, lexicalisation à base de classes pour les noms propres,etc. et lexicalisation fallback pour les mots inconnus et lexicalisation grammaticale pour les mots fonctions.  Nous ne traiterons pas des idiomes ni des collocations dans cette section. Pour plus de détails, vous référer à (Lambrey-Lareau, Lambrey, lareau)

La lexicalisation s'opère via des ressources lexicales (règles de lexicalisation et dictionnaires). Ainsi, pour choisir le lexème qui conviendra dans la structure syntaxique désirée, il faudra consulter les dictionnaires lexicaux. Ce sont eux qui encodent ce type d'information. Dans GenDR, nous avons quelques dictionnaires qui intéragissent avec nos règles. Nous avons d'abord le dictionnaire sémantique, puis le dictionnaire lexémique et un dictionnaire de fonction lexicale. Il s'agit de ceux qui ont été présentés en \ref{dictio}. En ce qui concerne les règles. Nous décrirons ici quelques règles de lexicalisation de base.

\subsubsection{Règles de base pour la lexicalisation}

\draft{revoir cette section}
1.Les lexicalisation simples sont traitées par la règle lexicalization\_standard. Pour une unité sémantique donnée dans un graphe, on regarde dans le dictionnaire sémantique à quoi cette entrée correspond pour les unités lexicales (tel que démontré en \ref{semanticon}). Donc on regarde dans l'entrée dans le semanticon et on retrieve l'ensemble des unités lexicales qui peuvent exprimer ce sens. Pour choisir la bonne lexicalisation de ce sens, on regarde tous les traits qui leurs sont associés et c'est la DPOS qui va permettre de confirmer le choix d'une lexie par rapport à une autre. Effectivement, si la lexie a la DPOS demandée par le noeud dans l'arbre de dépendance en construction et qu'elle satisfait cette contrainte, alors on pourra la lexicaliser et l'arbre aura une branche dont le noeud sera consommé par cette lexie. D'ailleurs, c'est ce mécanisme qui permet le paraphrasage, puisque si plus d'une lexicalisation répond aux contraintes demandées par le noeud, alors on créera autant de structures différentes qu'il y a de lexicalisations possibles. 

2. Class-based lexicalization: numbers, etc. Les règles de lexicalisation qui se charge des unités sémantiques que nous ne voulons pas nos dictionnaires sémantiques et lexicaux.  Parce qu'ils sont trop nombreux et que leur comportement est prédictible. On parle des chiffres, nom propres, dates, etc. Pour traiter la lexicalisation de ceux-ci, on met l'attribut "class" dans la structure sémantique de départ pour préciser qu'il s'agit d'un objet appartenant à cette classe. Cela va déclencher la règle class-based lexicalization quand viendra le temps de lexicaliser ce sens. Pour ce faire, il y existe une classe pour les dates, une classe pour les noms propres et etc. Ceux-ci vont donc directement hérité de tous les traits lexicaux encodés (DPOS et traits grammaticaux) dans ces classes et seront réalisés par après. 

Fallback lexicalization: Cette règle sert à lexicaliser une unité sémantique qui se retrouve dans le graphe d'input, mais qui n'a pas d'entrée dans le semanticon. Donc au lieu de faire planter la réalisation, le système a été désigné pour tenir compte de cet imprévu. Puisque le système n'a que 1500 mots les plus courant de l'anglais, il est clair que tous les unités sémantiques n'y figurent pas. Mais pour pouvoir réussir à réaliser quelque chose quand même. Lareau et Al. a développé une règle pour lexicaliser quand même. S'il y a une entrée dans le lexicon, le système va directement prendre la lexicalisation. Ce qui fait que tous les sens qui se réalise par une seule lexicalisaiton n'ont pas besoin d'être dans le semanticon (sauve du temps). Si le mot n'existe pas dans le lexicon, alors le système prend un guess. Concrètement l'étiquette de l'unité sémantique sera transféré en syntaxe et donc le sémantème sera lexicalisé avec la même étiquette. S'il y a des contraintes sur le noeud d'arrivée, alors le système suppose que l'unité lexicalisée satisfait les contraintes. S'il n'y a pas de contraintes, alors le système supposera que c'est un nom. Ces lexèmes devinés seront mis en évidence dans la réalisation d'output afin que l'utilisateur sache que cette partie de l'arbre a été devinée dû à un manque d'information. Le tout peut ainsi être filtré au besoin.
	
Grammatical lexicalization: Les règles de lexicalisation précédentes prenaient place dans le passage de la Rsem à la RSyntP. Cette règle introduit des lexèmes au niveau de surface. Elle introduit deux types de mots fonctionnels. Les mots exprimant un sens grammatical comme les auxilaires et les déterminants. Puis les mots sélectionnés par les patrons de régime d'un lexème (comme les prépositions) et encodés dans le gp de ceux-ci. Ils existent en syntaxe profonde, mais pas comme des noeuds dans l'arbre, plutôt comme des traits assignés aux lexèmes dans l'arbre. Ils seront lexicalisés dans le passage de la RSyntP à la RSyntS. Comme ces mots appartiennent à une classe fermée et qu'ils sont peu nombreux et souvent spécifiques à une langue, ce sont donc des règles particulières qui traitent ces lexiclaisaiton dans chaque module de langue. Les prépositions sont introduits en syntaxe comme des noeuds extra entre un gouverneur et son dépendant. On va donc chercher sa lexicalisation de surface dans le patron de régime du gouverneur. 


%%%%%%%%%%%%%%%%%%%%%%%%%%%%%%%%%%%%%
% --------- E X E M P L E ---------
%%%%%%%%%%%%%%%%%%%%%%%%%%%%%%%%%%%%%

\section{Exemple}

Nous utiliserons l'exemple d'input de la figure \ref{input} pour démontré comment avec une telle structure d'entrée notre système génère des arbres de surface.

\subsection{1ère phase: RSem à RSyntP}
Nous reprendrons donc l'input de la figure \ref{input} que nous allons passer à notre système. À l'aide de l'inspecteur que nous avions mentionné à la section \draft{mettre la section} nous pourrons illustrer les étapes successives qui ont permis à GenDR de passer de la RSem à la RSyntP.Cet input contient aussi les grammèmes de temps nombre et définitude. De même que l'appartenance à une classe. Donc, on passe cet input au système et la première étape est de créer un noeud vide en syntaxe, qui sera la racine de l'arbre à construire. Ce noeud est contraint par la règle root que nous avons vu plus tôt. Donc, il faudra que ce soit un dpos=V fini et à l'indicatif. Dans la figure \ref{fig:rootstand} on voit que le système crée un noeud non-étiquetté, c'est ce que nous entendons par noeud vide. Il se fait donc assigner une étiquette aléatoire par le système.

\subsubsection{root standard : crée le root}
\begin{figure}[htb]
	\centering
	\includegraphics[width=0.6\textwidth, trim = {0cm 0cm 0cm 0cm},clip]{ch3/figs/inspecteur_root.png}
	\vspace{-0.5cm}
	\caption{application de root\_standard}
	\label{fig:rootstand}
\end{figure}

\subsubsection{lex standard : owe (satisfait les contraintes du noeud)}
Par la suite, la règle de lexicalisation standard s'applique. Cela est possible parce que le système regarde dans le semanticon pour l'unité sémantique dans le graphe d'input. Si elle existe dans le semanticon, alors il regarde ce qu'il y a à l'intérieur. Elle renferme deux lexicalisations possibles : owe et debt. Donc on va tenter ces deux lexèmes pour voir si ça fonctionne. Effectivement le système permet de prendre owe puisque owe est un verbe et qu'il satisfait la contrainte. Mais il permet aussi de prendre debt car celui-ci incorpore des verbes supports qui peuvent satisfaire la contrainte de root. Ainsi, un mécanisme complexe de lexicalisation permettra de prend un noeud sémantique en input et de le scinder en deux lors de l'arborisation. Le verbe support deviendra la racine dans ce cas. et le mot debt deviendra un de ses dépendants. Mais nous n'entreront pas dans les détails. POur cela nous vous renvoyons à Lambrey et Lareau-Lambery. Donc nous constatons qu'à ce point, plusieurs arborisation sont possibles, mais nous choisissons de travailler avec la lexicalisation simple. C'est pourquoi nous optons pour l'arborisation qui prend le verbe owe comme racine. 
\begin{figure}[htb]
	\centering
	\includegraphics[width=0.7\textwidth, trim = {0cm 0cm 0cm 0cm},clip]{ch3/figs/lex_standard_root.png}
		\vspace{-0.5cm}
	\caption{application de lex\_standard}
	\label{fig:lexstand1}
\end{figure}

\subsubsection{actant gp : crée les branches et les noeuds vides}
Une fois que owe est lexicalisé, cela déclenche l'application de la règle actant\_gp. Pour chaque relation actancielle, elle puise dans le patron de régime de l'entrée lexicale quelconque. Puis elle fait la conversion des actants sémantiques en actants syntaxiques. OWe a trois actants sémantiques dans l'input. Ils seront réalisés en actants syntaxiques en fonction de la diathèse de cette entrée. (parler de la diathèse). Le patron de régime\draft{utiliser la macro pour gp} de owe impose des restrictions aux actants syntaxiques. La partie de discours doit être X. Au bout de ces branches nouvellement ajoutées, on crée des noeuds vides, mais contraints par une partie du discours spécifique.
\begin{figure}[htb]
	\centering
	\includegraphics[width=1\textwidth, trim = {0cm 0cm 0cm 0cm},clip]{ch3/figs/actant_gp1.png}
	\caption{application de actant\_gp}
	\label{fig:actantgp}
\end{figure}

\subsubsection{lex class et lex standard}
Donc, la suite des choses est de répétée la phase de lexicalisation puisque nous avons des noeuds non-étiquetés. Alors on regarde l'unité sémantique spécifiée dans l'input qui est 'bank' 'paul' '\$500'. On va donc regarder dans le semanticon s'ils s'y trouvent. On trouvera 'bank' et sa lexicalisation 'bank' mais pas 'Paul' ni '\$500' car ceux-ci ont le trait 'class' en input. Donc lex\_standard s'applique et on lexicalise 'bank' au noeud du IIIe\draft{comment mettre le petit e} actant syntaxique car il satisfait les contraintes du noeud. Effectivement, dans le lexicon 'bank' est de type dpos=N ce qui correspond à la contrainte demandée par la règle actant\_gp précédement via le gp du verbe owe. Pour ce qui est de 'Paul' et '\$500', ils ont le trait class dans l'input sémantique. C'est là que la règle de lex\_class entre en ligne de compte. Elle passer directement au lexicon, puisque les sens ne sont pas encodés dans le semanticon comme les classes ont des comportements prédictibles. Dans le lexicon par contre, c'est là que les classes sont décrites. Effectivement la classe proper\_noun et la classe numeral vont permettre l'application de la règle lex\_class et de lexicaliser paul et 500 piastres. Les classes ont les informations de type dpos et autres traits grammaticaux. L'étiquette du noeud sémantique est directement copiée en syntaxe. on s'assure finalement que les contraintes de ces classes respectent les contraintes du noeuds généré par gp\_actant. Lorsque c'est le cas, la lexicalisation se produit. Les noms propres héritent des traits de la classe des noms, donc la dpos est là, puis les montants héritent des traits de la classe des nombres.

\begin{figure}[htb]
	\centering
	\includegraphics[width=1\textwidth, trim = {0cm 0cm 0cm 0cm},clip]{ch3/figs/lex_standard2.png}
	\caption{application de lex\_standard}
	\label{fig:lexstand2}
\end{figure}

\subsubsection{traits grammaticaux}
\draft{à revoir, pourquoi j'avais ajouté ça dans mon plan}


\subsection{2ème phase: RSyntP à RSyntS}

\subsubsection{lex class et lex lu}
On va chercher les lexicalisations de surface de chacune des unités lexicales. Provenant des classes et des entrées directement. \draft{demander à françois à quoi sert cette étape}. Lex class fait paul et 500 , puis lex lu fait bank et owe.
\begin{figure}[htb]
	\centering
	\includegraphics[width=0.7\textwidth, trim = {0cm 0cm 0cm 0cm},clip]{ch3/figs/rsyntslexicalisation1.png}
	\caption{application de lex class et lex lu}
	\label{fig:lexsurf}
\end{figure}

\subsubsection{synt actant subj}
Réalise la relation subjectale entre le verbe et le nom qui a cette relation. Cette information est encodée dans le gp du verbe owe qui est hérité de verb dit, puis de verb dt jusqu'à verb tout court. Qui dit que son premier actant I, est le sujet.

\draft{peut-être mettre du code inline ici ?}

\subsubsection{synt actant dir}
Crée la relation d'objet direct entre le verbe et l'actant II. Cela est encodé via le verbe owe, qui est un verbe dit, donc hérité de verb dt, qui a comme info que son IIe actant correspond à la relation objet direct.

\subsubsection{synt actant prep}
Celle-ci est la plus complexe des règles actancielles de surface. Car elle va prendre un noeud syntaxique profond et le scinder en 2 pour réaliser le mot fonctionnel qu'est la préposition. 

\subsubsection{det def}
Cette règle de surface ajoute les déterminants.

La figure \ref{syntsurf} démontre l'application simultanée de toutes ces règles de surface.

\begin{figure}[htb]
	\centering
	\includegraphics[width=0.7\textwidth, trim = {0cm 0cm 0cm 0cm},clip]{ch3/figs/rsynts_syntactisation.png}
	\caption{application de subj,dir et prep}
	\label{fig:syntsurf}
\end{figure}

%%%%%%%%%%%%%%%%%%%%%%%%%%%%%%%%%%%%%%%%%%%%%%%%
% --------- P R O B L É M A T I Q U E ---------
%%%%%%%%%%%%%%%%%%%%%%%%%%%%%%%%%%%%%%%%%%%%%%%%

\section{Problématique}\label{problema}
La raison d'être de ce mémoire prend vie à cause de GenDR. En travaillant avec ce réalisateur profond pour réaliser divers types d'output et en voulant couvrir le plus large possible les phénomènes langagiers, on s'est rendu compte que la couverture d'un réalisateur passe impérativement par les verbes. Ainsi, pour avoir une meilleure couverture des verbes, nous nous sommes retournés vers les dictionnaires de verbes. Plus précisément des dictionnaires contenant de l'information de nature lexicale sur les verbes. Si on a accès aux patrons de régime d'une grande quantité de verbe, notre couverture se vera largement agrandie.Ça c'est la première raison, plus théorique et pragmatique. 

D'ailleurs, 

présenter les limites de ce système et pourquoi nous voulions aller chercher l'aide d'une ressource comme VerbNet pour pallier à ce problème.
mentionner les dictionnaires qui disent que les verbes sont les plus importants
Expliquer c'est quoi un patron de régime/cadre de sous-catégorisation, valence, etc.
permettent de couvrir large
patrons de régime des verbes y sont encodés
\draft{est-ce que je parle de ce problème ici, ou pas du tout, ou bien juste quand on parle du gpcon directement: problématique liée aux verbes : mécanisme d'héritage pas capable d'avoir deux parties du discours différentes pour un objet direct par exemple, restreint le nombre de génération, ou ça peuple inutilement le dictionnaire d'entrée verbale par type de complément qu'il prend. Pas logique en aucune manière. C'est là qu'un dictionnaire de patron de régime entre en ligne de compte}.
Forge utilise un dictionnaire de valence, aussi un héritié de MARQUIS utilise un dictionnnaire de valence.
%!TEX root = ../memoire.tex

\chapter{Un dictionnaire de patrons de régime: VerbNet}

Avant de parler de VerbNet, nous expliquerons pourquoi nous l'avons choisi parmi tant de candidats possibles. Dans la section qui suit, nous ferons un bref survol de ces candidats. Nous avons analysé les composantes de: WordNet, FrameNet, XTAG, LCS, Comlex, Valex, VDE et le Dicovalence. Parmi ces dictionnaires, il y en a qui traitent de d'autres parties du discours tandis que certains ne traitent uniquement que des verbes. Le Dicovalence est le seul qui analyse le français, les autres sont originalement programmés en anglais.

%%%%%%%%%%%%%%%%%%%%%%%%%%%%%%%%%%%%%%%%%%%%%%%%
% --------- D I C T I O N N A I R E S  ---------
%%%%%%%%%%%%%%%%%%%%%%%%%%%%%%%%%%%%%%%%%%%%%%%%

\section{Dictionnaires verbaux concurrents}

\subsection{WordNet}
Wordnet \citep{Fellbaum1998} est une base de données lexicales traitant les verbes, noms, adjectifs et adverbes de la langue anglaise. Cette base de données s'organise en \emph{synset}: des ensembles de synonymes. Il ne s'agit pas nécessairement de synonymes exacts, mais plutôt d'un ensemble de lexies, d'une même partie du discours, unis par des traits conceptuo-sémantiques. Ces \emph{synset} sont associés à une définition et un exemple d'utilisation. WordNet est une base de données hiérarchisée. Elle est construite via des liens d'hyperonymie et d'hyponomie entre les \emph{synsets}. Les synsets sont répartis parmi les classes suivantes. S'il s'agit de verbes dénotant des actions ou des évènements, ils seront classés parmi: \emph{ motion, perception, contact, communication, competition, change, cognition, consumption, creation, emotion, perception, possession, bodily care and functions, social behavior and interactions}. S'il s'agit de verbes dénotant des états, on le retrouvera parmi les classes de type: \emph{resemble, belong, suffice}, ou des classes de type: \emph{want, fail, prevent, succeed}, ou de types aspectuels comme \emph{begin} \citep{Fellbaum1998}. À l'intérieur d'une entrée, on retrouve aussi des liens lexicaux: synonymes, antonymes, troponymes, implication et causation \citep{SchulerVerbnetBroadcoverageComprehensive2005}. Cela leur a permis de tisser une toile sémantique assez volumineuse.

À la base, WordNet a été conçu comme réseau lexical. Cela explique pourquoi il contient peu d'information syntaxique. La ressource fournit des définitions, des exemples et des \emph{synsets}, mais ne nous donne pas d'information sur la structure sémantique ou syntaxique des verbes (comme les patrons de régime ou les prédicats sémantiques). Elle est systématiquement implicite. Comme nous voulions nous construire un dictionnaire verbal automatiquement, il nous fallait un dictionnaire qui explicite ce genre d'information lexicale. Nous voulions une base de données explicitant très clairement les différents actants régis par un verbe, ainsi que les prépositions sélectionnnées par celui-ci. Toutefois, chaque verbe dans VerbNet est mappé à un \emph{synset} de WordNet (lorsque c'est possible) \citep{SchulerVerbnetBroadcoverageComprehensive2005}.

\draft{plus de 21 000 verb word forms (comment traduire verb word forms?) \citep{MillerWordNetonlinelexical1990} refaire cette section en relisant l'article LARGE-SCALE LEXICOGRAPHY IN
THE DIGITAL AGE}

\subsection{FrameNet}

Parallèlement au projet WordNet, s'est développé FrameNet \citep{BakerBerkeleyFrameNetProject1998}. Le projet de Berkeley FrameNet est basé sur un corpus manuellement annoté, qui contient de l'information sur les noms, adjectifs et verbes de la langue anglaise. Dans FrameNet, les unités lexicales sont décrites en termes de \emph{frame semantics}, qu'on traduirait par la sémantique des cadres.Les semantic frames sont définis comme des représentations schématiques de situations impliquant des participants , propositions et d'autres rôles conceptuels Le but de cette ressource est d'encoder la sémantique du lexique de l'anglais dans un modèle que peuvent lire les machines \citep{BakerBerkeleyFrameNetProject1998}. Ce projet couvre la sémantique des domaines suivants: santé, chance, perception, communication, transaction, temps, espace, corps , motion, étapes de la vie, contexte sociaux, émotion et cognition. Cette base de données lexicales est composée de trois modules. D'abord, un dictionnaire dont les entrées sont les unités lexicales traitées. Suivi d'un dictionnaire de \emph{frames} et complété par des exemples annotées manuellement correspondant aux \emph{frames}. Ainsi, il faut passer par le dictionnaire d'entrées lexicales, pour ensuite identifier le cadre qui lui est associé dans le dictionnaire de cadres sémantiques. Ainsi, il faut d'abord chercher dans le dictionnaire d'entrées lexicales pour ensuite trouver le ou les cadres qui lui sont associés dans le dictionnaire de cadres sémantiques. D'ailleurs, les descriptions des frames sont encodés en structures conceptuelles. Et les phrases exemples manuellement annotées sont des preuves empiriques que les \emph{frames} ont lieu d'être. Les frames décrivent la structure argumentale d'une unité lexicale. Ces arguments sont identifiés par des étiquettes similaires aux rôles thématiques. On les appelle des \emph{frame elements} et ils sont extrêmement nombreux car ils sont spécifiques aux cadres qu'ils décrivent.En frame semantics, un frame correspond à un scénario qui implique une intéraction et des participants \citep{Shi:2005:PPT:2132047.2132058}. À noter qu'il y existe aussi une organisation hiérarchique où on a des sous-frames qui héritent de traits des frames parents. 

Tout comme VerbNet l'a fait avec WordNet, un mapping a été effectué entre les entrées de VerbNet et FrameNet. Cela s'est fait en deux étapes, ils ont mapper les classes de VN avec les frames de FN, puis les frames elements aux rôles thématiques \citep{Shi:2005:PPT:2132047.2132058}. Finalement, d'un point de vue pratique, FN est généralement utilisé comme \emph{semantic parser}. Des chercheurs font des parse tree syntaxique mais qui tiennent compte des participants et de leur relation avec le verbe\citep{Shi:2005:PPT:2132047.2132058}.

\draft{manque les statistiques}

\draft{FrameNet is a computational lexicography project that extracts information about the linked semantic and syntactic properties of English words from large electronic text corpora, using both manual and automatic procedures, and presents this information in a variety of web-based reports. The name ‘FrameNet’, inspired by ‘WordNet’ (Fellbaum 1998), reflects the fact that the project is based on the theory of Frame Semantics, and that it is concerned with networks of meaning in which words participate.}

\draft{The central idea of Frame Semantics is that word meanings must be described in relation to semantic frames – schematic representations of the conceptual structures and patterns of beliefs, practices, institutions, images, etc. that provide a foundation for meaningful interaction in a given speech community. FrameNet identifies and describes semantic frames, and analyzes the meanings of words by directly appealing to the frames that underlie their meanings and studying the syntactic properties of words by asking how their semantic 
properties are given syntactic form}

\draft{The primary units of lexical analysis in FrameNet are the frame and the lexical unit (LU: Cruse 1986), defined as a pairing of a word with a sense (e.g., the hot of temperature and the hot of taste experiences are two among the many lexical units that use the adjective hot). Generally speaking, the separate senses of a word correspond to the different semantic frames that the word can participate in (or, as we will see below, different sets of frames). When a word’s sense is based on a particular frame, we say that the word evokes the frame: thus, the word hot is capable of evoking a temperature scale frame in some contexts and a particular taste experience frame in others. Interpreting a sentence containing this word requires assumptions about which frame is relevant in the given context.}

\draft{ In FrameNet, information about valence must be specified in both semantic and syntactic terms; the semantic roles that complements play with respect to the meaning of the word must be accounted for, and the grammatical properties of the possible complements of a word must be identified. Semantic valence information is often recorded in a notation that is similar to logic, and referred to as argument structure. }

\draft{FrameNet data is stored in a relational database that reflects, insofar as possible, the theoretical basis of the project. Because of the different kinds of information that is represented in the FrameNet database, it is convenient to characterize it in terms of two parts: the lexical database and the annotation database}

\draft{We refer to patterns like these as valence patterns. One of the main purposes of FrameNet is to identify valence patterns for a large number of English verbs, nouns, adjectives, adverbs, and prepositions, and to annotate corpus citations to show how those valence patterns are instantiated in actual sentences}

\draft{exemple d'application de FrameNet: Maintaining the balance between knowledge and the lexicon in terminology: a methodology based on Frame Semantics}

\subsection{XTAG}
XTAG \citep{ResearchGroupLexicalizedTreeAdjoining2001} est une grammaire de l'anglais basé sur le formalisme de \acf{TAG}. Cette ressource offre des descriptions syntaxiques riches des verbes. Chaque unité lexicale se fait assigner un ensemble d'arbres-TAG. Ceux-ci décrivent les comportements syntaxiques permis par la lexie en question. Ces arbres reflètent donc la structure argumentale des lexies.

La construction des arbres se fait à partir de deux opérations: la substitution et l'adjonction. En joignant de nouvelles branches ou en substituant des branches, la grammaire \ac{TAG} permet de rendre compte des divers phénomènes linguistiques de la langue anglaise \draft{revenir sur ce passage, on comprend pas tout à fait comment TAG fonctionne}. XTAG couvre les descriptions syntaxiques de 9000 verbes. 

Ce dictionnaire organise l'information syntaxique en regroupant les arbres \ac{TAG} en familles d'arbres. À l'intérieur de celles-ci, ce qui distingue un arbre d'un autre est l'alternance syntaxique. Ainsi, dans XTAG, Les classes verbales sont organisées de cette manière: chaque verbe dans le dictionnaire correspond à plusieurs familles d'arbres et chaque famille regorge d'arbres individuels issus de différentes transformation syntaxiques de surface pour une même structure argumentale canonique. Grâce à ce mécanisme, on n'a pas à lister tous les arbres possibles correspondant à une lexie verbale, car celle-ci se fait assigner des familles d'arbres \citep{DoranXTAGSystemWide1994}.

Finalement, comme avec WordNet et FrameNet, Les entrées lexicales VerbNet sont mappés aux arbres de XTAG \citep{W04-3326}. D'ailleurs, ce mappage leur a permis de couvrir des descriptions syntaxiques qu'ils n'avaient pas répertoriés.

\draft{XTAG aurait été un bon candidat, mais ne fait pas la désambiguisation des verbes, et est trop lié à une théorie linguistique. Cela le rend plus difficile à implémenter en TST. Tandis que VerbNet est relativement neutre.}

\subsection{Lexical conceptual structures (LCS)}
La base de données LCS de Dorr s'est construite à partir des théories de sémantique lexicale de Jackendoff. Celui-ci argumente en faveur d'une approche de décomposition sémantique des verbes. Ceux-ci sont décrits en termes de leur structure conceptuelle lexicale (lCS)\citep{DorrUseLexicalSemantics1992}. Une structure LCS est un graphe sémantique et les structures syntaxiques de surface découlent directement des représentations sémantiques.Ces graphes sont des représentations hiérarchiques non-linéaires composées d'une tête logique (la racine du graphe), d'un sujet logique (un seul) , d'arguments logiques et de modificateurs logiques. En ce qui concerne le traitement des verbes dans LCS, le verbe est la racine du graphe et les arguments du verbe (sujet et objets) sont les arguments logiques liés à la racine.

Chaque noeud des graphes LCS ont trois spécifications: type, primitif sémantique et champ. Ceux-ci permettent de restreindre les unités lexicales qui peuvent satisfaire les noeuds des graphes sémantiques.

Une décomposition sémantique des verbes en termes de structures lexicales conceptuelles explique leur propriété syntaxiques. Tel que Levin l'avait perçu, les propriétés sémantiques des verbes influenceront leur comportements syntaxiques. À l'intérieur de ce cadre théorique, on pense que les verbes avec des LCS similaires partagent aussi des comportement syntaxiques comme des alternances de diathèses. La base de données LCS de Dorr s'inspire fortement des travaux de \cite{verb-classes.levin.1993}. Dans cette base de données, les verbes sont rassemblés en classes verbales. Ce qui unit les membres à une classe verbale est le partage d'une structure LCS commune. Ainsi, tous les membres d'une classe partage la même structure sémantique \citep{TraumGenerationLexicalConceptual2000}. 

Nous n'avons pas pris cette ressource car elle n'offrait pas le contenu syntaxique que nous cherchions. Notamment, une banque de patron de régime des verbes de l'anglais.

\subsection{Comlex}\label{comlex}
Comlex est une base de données lexicales développée pour l'anglais à NYU \citep{Grishman:1994:CSB:991886.991931}. C'est une ressource syntaxique riche, mais qui n'est pas libre d'accès. Comlex est un dictionnaire syntaxique des verbes créé à des fins computationnelles. Ses auteurs ont opté pour un système qui se voulait le plus neutre possible d'un point de vue théorique afin d'être utilisé par le plus grand nombre de systèmes. Ce dictionnaire ne traite pas uniquement que les verbes, mais c'est la partie qui nous intéresse ici. Comlex décrit pour chaque verbe les compléments possibles qu'il peut sélectionner et il explicite les spécificités propres à certaines constructions (comme le choix d'une préposition). Les entrées lexicales ont été manuellement encodées parce que les auteurs du sytème ne n'étaient pas en faveur des méthodes automatiques. Ils ne croyaient pas que les sytèmes automatiques d'acquisition étaient capables de traiter correctement les verbes à faibles fréquences. Finalement, Comlex contient des descriptions syntaxiques pour 6000 verbes. 

Nous n'avons pas pris ce système car, il faut payer la license. Puis, parce que \cite{SchulerVerbnetBroadcoverageComprehensive2005} en avait fait une évaluation qui nous semblait pertinente. Elle soulignait que Comlex ne fait pas la distinction de sens différents entre les verbes. Nous pensons aussi que c'est un aspect crucial dans le développement de dictionnaires computationnels.

\subsection{Valex}
 
Valex est un projet de Korhonen, il s'agit d'un dictionnaire de \acf{SCF} de l'anglais \citep{Korhonenlargesubcategorizationlexicon2006} qui traite 6397 verbes. Elle a bâti son dictionnaire à partir de méthodes d'acquition automatiques. Contrairement à Comlex que nous avons présenté à la section \ref{comlex}, Korhonen vante les mérites d'une acquisition automatique. Elle stipule que les dictionnaires bâtis manuellement comportent naturellement plus d'erreurs. Elle souligne aussi que la méthode automatique est moins coûteuse en termes de temps et de ressource. De plus, elle suggère que les dictionnaires manuellement acquis comportent une faille cruciale: le manque d'information statistique. Grâce aux informations statistiques acquises via le traitement de corpus, on a de l'information quant à la fréquence d'utilisation d'un \ac{SCF} pour un verbe donné. Finalement, elle souligne qu'en raison du nombre d'applications \ac{TAL} fonctionnant avec des méthodes probabilistes, la présence d'information statistique ajoute à leur bon fonctionnement. En ce qui nous concerne, notre application \ac{TAL} fonctionne à base de règles, donc les statistiques n'ajoutent en rien au système, mais ce sont des données qui seraient intéressantes à avoir.\draft{Est-ce que ça nous serait utile pour la réalisation à base de règles ?} 

Dans son article, Korhonen explique avoir utilisé le système d'acquisition de Briscoe et Caroll qui se base sur la méthode RASP \citep{BriscoeSecondReleaseRASP2006}. Brièvement, les \ac{SCF} sont extraits grâce au système RASP à partir de textes non-annotés. Ce système tokénise, étiquette puis lemmatise les données brutes. Ensuite, les \ac{SCF} sont extraits des phrases analysées du corpus. Finalement, un filtrage est effectué pour se débarrasser du bruit. Puis, le dictionnaire est construit. 

Dans Valex, une entrée lexicale comprend entre autre: la combinaison d'un verbe et d'un SCF, la syntaxe des arguments, la fréquence d'utilisation du SCF. Bien que ce système aurait été potentiellement bon, nous avons préféré nous tourner vers VerbNet en raison de son architecture hiérarchisée qui nous est très utile. L'architecture de Valex ne nous permet pas de tirer parti du mécanisme d'héritage des traits que nous avons vu à la section \ref{dictio}. Ce qui fait en sorte qu'en utilisant Valex, notre dictionnaire serait très lourd et saturé d'informations.

\subsection{Valency dictionary of English}
Le \acf{VDE} est un dictionnaire de valence de l'anglais \citep{HerbstValencyDictionaryEnglish2004}. Il contient les patrons de valences de 511 verbes (il traite aussi les noms et les adjectifs). Dans ce dictionnaire, chaque entrée décrit une valence possible pour un verbe. Le tout est accompagné d'un exemple provenant de la \emph{Bank of English} \draft{citation?}. Lors de sa création, le VDE n'était pas destiné à des applications TAL. Cependant, less auteurs se sont rapidement rendus compte que ce dictionnaire pourrait intéresser les linguistes computationnels. 

Cela a entraîné la création de l'\emph{Erlangen Valency Pattern Bank} \citep{faucris.1039365}. Un outil de TAL qui liste les patrons de valence identifié par le VDE.  Dans le VDE, les 511 verbes qui y figurent ont été choisis sur la base qu'ils sont fréquents dans la langue anglaise, qu'ils démontrent des propriétés complexes et qu'ils sont utiles pour des apprenants de l'anglais. Les patrons de valence qu'on retrouve dans le VDE proviennent d'une étude de corpus fait sur le COBUILD. Les patrons y sont décrits en termes de syntaxe de surface.  Leur dictionnaire est réparti en deux où d'un côté on a la liste des 511 verbes et les patrons de valence leur étant associés, puis dans un autre dictionnaire les patrons de valence de la langue anglaise. Il s'agit aussi d'un système qui distingue les différents sens que peuvent prendre les verbes. 

Bref, il s'agit d'un dictionnaire qui couvre de verbes, mais les plus fréquents. Toutefois, il s'agit d'un travail manuel, donc on s'attend à ce qu'il ne comporte pas beaucoup d'erreurs, et on pourrait ainsi en extraire une partie pour complémenter le dictionnaire de VerbNet si tel est le besoin.

%J'ai retiré cette ressource car elle est une sous-branche francophone de Valex. Ils fonctionnent exactement de la même manière. Ce n'est donc pas pertinent d'avoir les mêmes mécanismes en double.
%\subsection{LexSchem}
%LexSchem est un dictionnaire de verbe pour le français créé par Messiant. Il jusitifiait la valeur de son projet en disant que  que l'information la plus utile qu'un dictionnaire peut offrir sont les cadres de sous-catégorisation des verbes \citep{MESSIANT08.142}. Ces \emph{subcategorization frame} (SCF) capturent, au niveau syntaxique, les différentes combinaisons d'arguments qu'un prédicat lie. Messiant ajoute que comme les verbes sont au centre des énoncés, un dictionnaire qui se concentre sur les cadres de sous-catégorisation peut être très bénéfiques à des applications TAL.  Nous avons vu jusqu'à maintenant qu'ils peuvent être utilisés pour le parsing et la traduction automatique par exemple. Toutefois, suivant les pas de Korhonen \citep{Korhonenlargesubcategorizationlexicon2006}, Messiant a bâti un dictionnaire de SCF pour le français via une acquisition automatique. Il se justifie en disant que cette technique a déjà fait ses preuves dans des applications réelles malgré le fait qu'elle n'est pas aussi précise et détaillée qu'une approche manuelle. Mais elle est beaucoup moins coûteuse en termes de temps et de ressource. De plus, une approche automatisée permet d'extraire de l'information qui pourrait s'avérer très utile pour des applications TAL. Notamment, les statistiques et la fréquence d'utilisation d'un SCF.  Son dictionnaire a été acquis à partir de corpus non-annoté. Par la suite,les SCF acquis automatiquement sont incorporés dans lexSchem. Voici la démarche qu'il utilisa, d'abord ils prennent des données brutes, puis il étiquette et lemmatise les mots  pour ensuite parser le tout. Après il ne reste qu'à en extraire les SCF. Dans LexSchem, Les entrées lexicales sont composées essentiellement de: L'unité lexicale, ses cadres de sous-catégorisation et des phrases exemples tirées de corpus ainsi que la fréquence d'utilisation du SCF.Ce que nous retenons de ce système, c'est qu'il pourrait être utile dans un avenir où nous voulions extraire VerbeNet qui est une version francophone de VerbNet. Nous pourrions ainsi complémenter la ressource francophone par une autre ressource franchophone. De plus, tel que VerbNet l'a fait, LexSchem construit ses entrées lexicales en misant sur les cadres de sous-catégorisation. Nous pensons aussi qu'un dictionnaire verbal en TAL devrait surtout incorporer ces données, ce qui nous intéresse sont les cadres de sous-catégorisation, car ceux-ci sont la partie la plus dure à traiter en TAL.

\subsection{Dicovalence}
Le Dicovalence est un dictionnaire de valence pour la langue française \citep{MertensdictionnairevalenceDICOVALENCE2006}. Comme il y a 3700 verbes traités et que la plupart des verbes comptent plus d'un cadre valenciel, il y a plus de 8000 entrées lexicales dans ce dictionnaire. Celles-ci correspondent à la combinaison d'un verbe, d'un cadre valenciel et d'un exemple. 

Les cadres valenciels sont décrits en fonction de l'approche pronominale \citep{blanche1987pronom} \draft{devrais-je approfondir sur cette approche ?}. Dans leur système, ce que plusieurs appellent des participants, des actants ou des arguments, sont appelés ici des paradigmes. Le Dicovalence a été créé dans une optique de \ac{TAL} et d'enseignement de la langue. Ce dictionnaire contient beaucoup d'information utile autre que les cadres valenciels. Les entrées lexicales sont accompagnées d'un nombre de traits utiles au traitement de la langue en surface et en profondeur: contraintes sur les paradigmes,types d'auxiliaire sélectionné puis traduction du lexème en allemand et en anglais.

Nous n'utiliserons pas ce système puisque nous avions décidé de traiter l'anglais. De plus, le système hiérarchique de VerbNet hérité de Levin a supplanté les autres systèmes. Toutefois, nous pourrions nous en servir pour complémenter la couverture du VerbNet francophone \citep{DanlosVerscreationVerb} si nous décidions de l'implémenter à GenDR \citep{lareau18}.

%%%%%%%%%%%%%%%%%%%%%%%%%%%%%%%%%%%%%%%%%%%%%%%%%%%%%%%%%%%%%%%%%%%%%%%%%%%%%%%%%%%%
% --------- A P P L I C A T I O N S     T A L      DE    V E R B N E T   ---------
%%%%%%%%%%%%%%%%%%%%%%%%%%%%%%%%%%%%%%%%%%%%%%%%%%%%%%%%%%%%%%%%%%%%%%%%%%%%%%%%%%%%

\section {Utilisation de VerbNet dans des applications TAL}

VerbNet a été utilisé dans un grand nombre de travaux de recherche. Nous en mentionnons ici quelques-uns.

\draft{
\subsection{Construction de graphes conceptuels}
\citep{HensmanAutomaticallyBuildingConceptual2004}\FL{intégrer les REF au texte}
Les graphes conceptuels servent à représenter le sens d'un énoncé. Dans une recherche, ils se sont servis des informations lexicales de VerbNet pour créer des graphes automatiquement. Ainsi, ils prennent des documents provenant de corpus, ils en extraient les phrases, ils les parsent syntaxiquement, puis les matche à un patron correspondant dans VerbNet. Après avoir un patron correspondant, la phrase est maintenant dotée de toutes les informations lexicales contenues dans les entrées de VerbNet et la création du graphe conceptuelle se fait automatiquement. Ces graphes sont ensuite ajoutés dans une base de données et ils facilient l'information retrieval.

\subsection{Parsing sémantique}
\citep{Shi:2005:PPT:2132047.2132058}
Dans ce projet, VerbNet est combiné à FrameNet et WordNet pour faire du semantic parsing. La force de VerbNet dans ce projet est son exhaustivité des différents patrons de régime de l'anglais, et les rôles sémantiques qui sont liés aux verbes. Ils suggèrent que l'utilité de VerbNet provient aussi de sa couverture de l'anglais et des patrons de régime possible est extrêment riche et donc cruciale pour ce genre d'opération. Le but d'un semantic parser est d'analyser le sens de la structure d'une phrase. Donc, via les ressources dont VerbNet, lorsqu'il parse un texte non annoté, le parseur cherche des patrons similaires à ceux se trouvant dans les bases de données lexicales pour ensuite étiquetté les phrases et en faire sortir la structure syntaxique et les caractérisitques sémantiques via les rôles thématiques.

\subsection{Un système de questions-réponses}
\citep{DBLP:conf/nlpke/WenJH08}
Pour un bon système de question-réponse, la précision de la réponse est cruciale. De parvenir à un haut niveau de précision pour une réponse, ils proposent un système de question-réponse reposant sur VerbNet. Leur système extait les informations syntaxiques et sémantiques des questions puis à partir du web formule des réponses candidates en fonction des informations extraites de la questions , puis VerbNet est appliqué dans leur système pour détecter les cadres syntaxiques des verbes dans les questions et dans les réponses candidates afin d'obtenir les informations syntaxiques et sémantiques. Puis, leur système choisi la phrase candidate répondant le mieux à la question et tenant compte de la structure sémantique et syntaxique de la question.

\subsection{A supervised algorithm for verb desambiguasation into VerbNet classes}
\citep{AbendSupervisedAlgorithmVerb2008}
Ils ont développé un modèle d'apprentissage supervisé pour mapper des lexèmes à des classes de VN. Comme la polysémie est un enjeu majeur en TAL, leur travail consistait à analyser un texte et lorsqu'il trouve un verbe, il doit l'associer à la bonne classe, en fonction des informations lexicales qu'on retrouve dans la classe. Développer ce genre d'outil en TAL est très important car, les machines ne voient les verbes que comme des chaînes de caractères dépourvues de sens. Le contexte d'utilisation d'un verbe nous permet de savoir à quelle classe VN il est présentement utilisé (comme bcp de verbes s'emploient avec des classes différentes selon le contexte). EN moyenne un verbe appartient à 2.5 classes différentes.

\subsection{VerbNet class assignment as a wsd task}
\citep{BrownVerbNetClassAssignment2011}
Des représentations verbales riches sont cruciales en NLP pour un parsing sémantique profond. Utiliser un lexicon déjà annoté à plusieurs niveaux peut être très utile pour ce genre de tâches. par exemeple la désambiguisation en serait très profitable. Leur classifieur sémantique s'apparente beaucoup à celui de Abbend \citep{AbendSupervisedAlgorithmVerb2008}.
}
%%%%%%%%%%%%%%%%%%%%%%%%%%%%%%%%%%%%%%%%
% --------- S Y N T H È S E  ---------
%%%%%%%%%%%%%%%%%%%%%%%%%%%%%%%%%%%%%%%%

\section{Synthèse}

Nous avons choisi VerbNet car c'est une ressource qui s'est distinguée de ses concurrentes. D'abord, par son imposante couverture de la langue anglaise \draft{(statistiques)}. Ensuite, par son architecture interne héritée de Levin et améliorée. Le regroupement en classes verbales facilite énormément le travail, et le découpage hiérarchique suivi des mécanismes d'héritage en font un outil très performant et précis, sans être encombrant et facilement repérable. Les verbes sont aussi classés dans plusieurs classes verbales, donc ils sont partiellement désambigguisés, ce qui est extrêmement utile, car on veut être le plus précis possible en NLG. De plus, les descriptions syntaxiques encodées en XML sont facilement exportable et maléable dans un format qui nous convenait, par le fait même, le traitement en Python devenait très accessible puisque NLTK avait déjà fait un pré-traitement de VerbNet, ainsi il existait des modules dont nous pouvions nous inspirer pour extraire l'information dans les balises de VerbNet. Un autre point important est qu'il existe beaucoup de ressources linguistiques à la VerbNet dans d'autres langues que l'anglais, dont le (mettre les citations de tous les verbnet étrangers) français, le portugais, l'italien, l'estonien, l'espagnol, le catalan, le tchèque. Finalement, puisque notre système GenDR se veut un générateur multilingue dans sa première version, c'est un premier pas pour facilement implémenter d'autres langues à notre système puisque nous avons déjà des scripts une architecture prête pour acceuillir des ressources lexicales similaires. 

De plus, dans la section précédente, on montre qu'il est utilisé pour des applications de TAL diverses, mais on a aussi trouvé des systèmes récents qui utilisaient VerbNet pour faire de la NLG. Des chercheurs ont extrait le mapping qui avait été fait entre XTAG-VerbNet afin de donner une couverture imposante de l'anglais pour l'interface sémantique-syntaxe de son générateur. Ils se sont surtout servis de VerbNet pour créer leur grammaire. Leur projet s'appelle S-STRUCT \citep{PfeilAlgorithmsResourcesScalable2016}, un générateur de texte automatique dont on a ajouté un module d'apprentissage machine pour la partie \emph{discourse planning}. Ainsi, en améliorant la partie discourse planning avec du machine learning, le système apprend à générer les phrases dans un ordre statistiquement plus logique.

D'un autre côté, Wanner et Mille ont publié un court article expliquant qu'ils prévoyaient utiliser VerbNet comme base de données lexicales car ils avaient besoin d'une ressource lexicale riche dans le cadre de génération automatique de texte \citep{MilleLargeCoverageDetailed2015}. De telles ressources sont importantes pour la couverture et qualité des en GAT. Ils mentionnent que des réaliseurs classiques comme KPML, surge, et REALpro \FL{citations} nécessitaient un enrichissement lexical pour être encore plus performant. Car, pour arriver à une structure syntaxique de surface, où tous les unités lexicales sont réalisées, il faut qu'un système de GAT possède des ressources lexicales capables de générer tous les actants d'un verbe en fonction des restrictions possibles et des bonnes prépositions, il faut que la génération soit impeccable aussi. Ces informations se retrouvent dans des dictionnaires de cadres de sous-catégorisation tel que ceux mentionnés plus haut. En anglais, VerbNet rempli très bien les caractéristiques recherchées pour des dictionnaires lexicaux car c'est un lexicon à grande surface qui couvre une grande partie de la grammaire anglaise et détaille avec précisions les patrons de régime possibles, ainsi que beaucoup d'autres informations. Toutefois, ils précisent quelque chose qui est très vrai, VerbNet est principalement utilisé pour des tâches comme semantic role labelling, information retrieval. En résumé, les ressources lexicales éxistantes sur le marché sont incomplètes et difficile à implémenter pour la NLG. Ce qui a découlé de ce travail est Forge \citep{DBLP:conf/semeval/MilleCBW17}, un générateur de texte automatique. Toutefois, ils n'ont pas utilisé VerbNet dans leur système, tel que nous l'avons vu dans la section précédente. Ce qui démontre l'importance de ce travail car nous avons décidé d'utiliser VerbNet en tant que ressource lexicale pour faire la tâche qu'ils avaient entamés, mais sur laquelle ils n'ont pas publié de résultats.

%%%%%%%%%%%%%%%%%%%%%%%%%%%%%%%%%%
% --------- V E R B N E T   -----
%%%%%%%%%%%%%%%%%%%%%%%%%%%%%%%%%%

\section{VerbNet}

VerbNet a été créé dans un contexte où il y avait un réel besoin pour un dictionnaire décrivant la richesse et la complexité que démontrent les verbes \citep{KipperClassBasedConstructionVerb2000}. Schuler trouvait qu'il y avait un manque de lignes directrices par rapport à l'organisation des verbes dans les dictionnaires destinés à des applications \ac{TAL}. \draft{ajouter du texte pour décrire VerbNet dans les grandes lignes et ce qu'on présentera dans la section. Levin, composantes de VN, dictionnaires concurrents, application de VN en TAL}

\subsection{Classes verbales de Levin}

Avant de parler de VerbNet, il est crucial de parler du travail qu'à fait \cite{verb-classes.levin.1993} car elle a ouvert la porte à la création de dictionnaires de verbes. Sa méthode de classification des verbes en a inspiré plusieurs dont VerbNet\cite{SchulerVerbnetBroadcoverageComprehensive2005}, et la LCS database\citep{AyanGeneratingParsingLexicon2002a}\citep{DorrUseLexicalSemantics1992}. D'ailleurs, l'architecture générale de VerbNet est basée sur ses travaux de classification des verbes.

\cite{verb-classes.levin.1993} a donc créé un dictionnaire où les verbes de la langue anglaise sont placés dans un nombre fini de classes verbales. L'appartenance à l'une d'entre elles est motivée par le partage de comportements syntaxiques communs. En observant les alternances de diathèses que démontrent les verbes, Levin remarquait que tout locuteur natif est conscient des alternances de diathèses possibles d'un verbe, et ce sans avoir de connaissances méta-linguistiques préalables. Ainsi, en se basant sur son intuition, Levin a tenté de délimiter tous les patrons de régime possibles pour les verbes de la langue anglaise. Lorsque plusieurs présentaient des caractéristiques communes sur le plan syntaxique, elle assemblait ces verbes ensemble. Le résultat de cela étant les classes verbales qu'elle posutle.

Bien que son travail s'insère dans le cadre de la syntaxe, elle supposait que les verbes qui se comportant de la même manière syntaxiquement possèdent probablement des propriétés sémantiques sous-jacentes communes. La logique de cette hypothèse étant que si deux verbes possèdent des composantes sémantiques similaires, il semble évident que cela se réflète à la surface par des comportement syntaxiques similaires. Toutefois, elle souligne que deux verbes d'apparence synonymiques peuvent très bien appartenir à deux classes différentes, tout comme deux verbes qui, en apparence, ne se ressemblent pas du tout, peuvent très bien partager des composantes sémantiques similaires.

\draft{Regrouper les verbes en classes verbales : avantage théorique et pratqiue. Théorique pcq démontre que des propriétés communces sémantiques mènent à des propriétés de surface communes aussi. Pratique pcq permet de construire un dictionnaire où les entrées lexicales ne sont pas prises individuellement, mais regroupées en classes. Cela faciliter l'entretien du dictionnaire.} Avec le système de Levin, lorsqu'on a fini de traiter une entrée verbale, on n'a pas besoin de décrire tous les patrons de régime qui lui sont associés car il ne nous reste qu'à l'ajouter à une classe déjà existante. Les auteurs de VerbNet qu'ils ont dû revisiter le classement initial de Levin puisqu'ils n'étaient pas d'accord avec le traitement de certaines entrées \citep{SchulerVerbnetBroadcoverageComprehensive2005}. 

Pour mieux comprendre le traitement des verbes à la Levin, voici un exemple tiré de la thèse de \cite{SchulerVerbnetBroadcoverageComprehensive2005} \draft{p.12-13}. On prend les verbes \lex{break} et \lex{cut}, puis on teste diverses configurations possibles pour décider s'ils appartiennent à la même classe. À prime abord, on pourrait penser que c'est le cas puisque leurs signifiés se ressemblent. \sem{Briser} et \sem{découper} partagent évidemment des composantes sémantiques car le sens d'altérer quelque chose est présent dans ces deux verbes. Cependant, le court exemple suivant nous démontre qu'ils appartiendraient à deux classes distinctes.

\ex. \label{transitive} \emph{Transitive construction}
	\a. John broke the window.
	\b. John cut the bread.
	
\ex. \label{middle} \emph{Middle construction}
	\a. Glass breaks easily.
	\b. This loaf cuts easily.
	
\ex. \label{intransitive} \emph{Intransitive construction}
	\a. The window broke.
	\b. \ungr{The bread cut.}

\ex. \label{conative} \emph{Conative construction}
	\a.\ungr{John broke at the window.}
	\b. John valiantly cut at the frozen loaf, but his knife was too dull to make a dent in it.

On voit d'abord que les constructions en \ref{transitive} et en \ref{middle} sont possibles pour ces deux verbes. Toutefois, en \ref{intransitive} et en \ref{conative}, on remarque qu'ils ne partagent pas ces cadres syntaxiques. \lex{Break} prend seulement la construction intransitive et exclut la conative, tandis que \lex{cut} prend la construction conative et exclut l'intransitive. Selon la logique de Levin, cela est dû à des différences de composantes sémantiques. Le verbe \lex{cut} décrit une série d'actions entreprises dans le but de séparer un objet en morceaux. Toutefois, il est possible de commencer à découper un objet sans que l'objet ne soit séparé. Dans ce scénario, on peut tout de même percevoir que l'objet a été découpé. En ce qui concerne \lex{break}, le changement d'état (le fait d'être séparé en morceau) est au c\oe{}ur même de l'évènement. Si on n'arrive pas au résultat final, une tentavive de briser quelque chose ne peut être perçue. 

Toutefois, nous devrons critiquer cette approche de Levin, car elle a omis de considérer un aspect important dans un traitement comme celui-ci. Dans l'exemple qu'elle nous fournit en \ref{intransitive} et en \ref{transitive}, \lex{break} n'a pas le même sens. Dans l'exemple \ref{intransitive}, on pourrait traduire le sens de \lex{break} par \sem{se briser} tandis que le sens de \lex{break} dans le contexte de l'exemple \ref{transitive} serait plutôt \sem{briser}. Cela a un impact direct sur la syntaxe, puisque le premier sens ne peut prendre qu'un seul argument, tandis que le second en prend nécessairement minimum deux. Cette lacune théorique de Levin est aussi présente dans VerbNet. Nous en discuterons plus en détails dans le chapitre \ref{eval} concernant l'évaluation de l'implémentation .

Bref, le projet de Levin a inspiré beaucoup de chercheurs, notamment l'équipe de VerbNet. C'est pourquoi ils ont repris une grande partie du travail de Levin dont l'organisation hiérarchique de VerbNet en classe et le regroupement des verbes en classes verbales. Toutefois, les auteurs de VerbNet ont retravaillé l'architecture de Levin et y ont apporté des corrections et améliorations \citep{verbnet.2006}.

%%%%%%%%%%%%%%%%%%%%%%%%%%%%%%%%%%%%%%%%%%%%%%%%%%%%%%%%%%%%%%%%%%%%%%%%%%%%%%
% --------- C O M P O S A N T E S    DE     V E R B N E T   -----
%%%%%%%%%%%%%%%%%%%%%%%%%%%%%%%%%%%%%%%%%%%%%%%%%%%%%%%%%%%%%%%%%%%%%%%%%%%%%%

\subsection {Composantes de VerbNet}  

Maintenant que nous avons présenté la contribution de Levin au développement de VerbNet, nous pouvons décrire les composantes de ce dictionnaire. Comme l'a fait Levin, il est aussi organisé en classes verbales. Chaque classe contient un ensemble de membres, une liste de rôles thématiques (accompagnés de restrictions sélectionnelles) utilisés pour décrire les arguments, puis un ensemble de cadres syntaxico-sémantiques. Chaque cadre est composé d'une brève description, suivi d'un exemple, puis d'une description syntaxique et sémantique\citep{SchulerVerbnetBroadcoverageComprehensive2005}.

\subsubsection{Classes verbales: organisation hiérarchique}

Les auteurs de VerbNet se sont fortement inspirés de Acquilex Lexical Knowledge Base \citep{CopestakeACQUILEXLKBrepresentation1992} pour l'organisation du lexique. Acquilex ordonnait l'information lexicale en hiérarchie. VerbNet a donc aussi implémenté un aspect hiérarchique à son dictionnaire en créant jusqu'à trois niveaux de profondeur pour organiser les classes verbales. 

Cela a entraîné la création des sous-classes. Celles-ci héritent de l'entièreté du contenu lexical de la classe qui la domine. Les sous-classes ont été créées pour spécifier qu'un sous-ensemble de verbes issus d'une classe mère démontrent des comportements syntaxiques différents du reste de la classe. Ceux-ci comprennent: les constructions syntaxiques, les prédicats sémantiques et les restrictions sélectionnelles sur les rôles thématiques {SchulerVerbnetBroadcoverageComprehensive2005}. Prenons un exemple tiré de VerbNet pour illustrer cette hiérarchie à plusieurs niveaux \draft{p.13 de Guidelines}

\begin{lstlisting}[language=XML, caption = Hiérarchie, label=hierarch]
<VNCLASS ID="spray-9.7">
    <SUBCLASSES>
        <VNSUBCLASS ID="spray-9.7-1">
                <VNSUBCLASS ID="spray-9.7-1-1">
        <VNSUBCLASS ID="spray-9.7-2">
            <SUBCLASSES/>
        </VNSUBCLASS>
    </SUBCLASSES>
</VNCLASS>
\end{lstlisting}

\texttt{Spray-9.7} est le nom de la classe qui englobe toutes les autres ici. À l'intérieur de celle-ci, on spécifie tous les membres appartenant à cette classe, les rôles thématiques, les cadres syntaxiques et les prédicats sémantiques. Puis \texttt{Spray-9.7-1} est une sous-classe qui hérite de l'information de sa mère, mais précise d'autres informations. \draft{Comme un sous-ensemble de verbes propres à ces comportements différents. Daniel: je ne comprends pas ce que tu veux dire ici} Puis \texttt{Spray-9.7-1-1} est une sous-classe d'une sous-classe, et ainsi de suite. Elle héritera des traits de sa classe mère ainsi que de la classe qui domine sa classe mère. Finalement \texttt{Spray-9.7-2} est la classe sœur de \texttt{Spray-9.7-1} donc, elle hérite aussi des traits de \texttt{Spray-9.7} mais ne partage pas les particularités de \texttt{Spray-9.7-1}.

Tel que démontré dans l'exemple \ref{hierarch}, les classes et sous-classes sont numérotées. Cette numérotation sert à expliciter la hiérarchie à l'intérieur d'une classe de VerbNet, mais elle sert aussi à regrouper des classes verbales en fonction de leur signifié. Cette numérotation est directement héritée du système de \cite{verb-classes.levin.1993}. Les nombres vont de 9 à 109 \draft{guidelines citation}. Le numéro associé à une classe sert à représenter le partage de caractéristiques sémantiques (et syntaxique) entre les classes qui partagent ce numéro. Par exemple, les classes signifiant \sem{mettre quelque chose} commenceront par le chiffre 9:

\FL{tu devrais peut-être te définir une macro pour avoir une typographie particulière pour les noms de classes : j'ai utilisé \texttt{}}

\begin{easylist}[itemize]
  & \texttt{put 9.1}
	& \texttt{put spatial 9.2}
	& \texttt{funnel 9.3}
	& \texttt{put direction 9.4}
	& \texttt{pour 9.5}
	& \texttt{coil 9.6}
	& \texttt{spray 9.7}
	& \texttt{fill 9.8}
	& \texttt{butter 9.9}
	& \texttt{pocket 9.10}
	
\end{easylist}

\subsubsection{Membres}
Traditionnellement, les entrées lexicales dans un dictionnaire représentent un seul et unique verbe. En ce qui concerne VerbNet, les entrées sont des classes verbales regroupant  plusieurs verbes à la fois. Cela permet à VerbNet de couvrir largement l'anglais sans recourir à une quantité excédante d'entrées. Pour garnir leur section \emph{Members}, VerbNet a puisé dans les travaux de Levin \cite{verb-classes.levin.1993},dans la base de données LCS \citep{AyanGeneratingParsingLexicon2002a} et a mené sa propre enquête pour délimiter à quelle classe verbale un verbe appartient.

Concrètement, cette information est encodée directement dans les entrées lexicales de VerbNet en XML. La figure suivante \ref{membre} démontre à quoi ressemble la section \emph{Members}. Cet exemple nous démontre que \lex{deal, lend, loan, pass, peddle} et \lex{refund} sont les membres issus de la classe \texttt{give-13.1}.

\begin{lstlisting}[language=XML, caption = Les membres d'une classe, label=membre]
<VNCLASS ID="give-13.1" xmlns:xsi="http://www.w3.org/2001/XMLSchema-instance"
 xsi:noNamespaceSchemaLocation="vn_schema-3.xsd">
    <MEMBERS>
        <MEMBER name="deal" 
				wn="deal%2:40:01 deal%2:40:02 deal%2:40:07 deal%2:40:06" 
				grouping="deal.04"/>
        <MEMBER name="lend" 
				wn="lend%2:40:00" 
				grouping="lend.02"/>
        <MEMBER name="loan" 
				wn="loan%2:40:00" 
				grouping=""/>
        <MEMBER name="pass" 
				wn="pass%2:40:00 pass%2:40:01 pass%2:40:13 pass%2:38:04" 
				grouping="pass.04"/>
        <MEMBER name="peddle" 
				wn="peddle%2:40:00" 
				grouping="peddle.01"/>
        <MEMBER name="refund" 
				wn="refund%2:40:00" 
				grouping="refund.01"/>
        <MEMBER name="render" 
				wn="render%2:40:02 render%2:40:01 render%2:40:00 render%2:40:03" 
				grouping="render.02"/>
        <!--removed "trade" from class because doesn't take "to-PP"-->
        <!--removed "volunteer "from class because doesn't fit dative or-->
        <!--PP recipient PP frames-->
    </MEMBERS>
\end{lstlisting}

\subsubsection{Rôles thématiques}
VerbNet critiquait les autres dictionnaires verbaux qui n'offraient pas de contenu sémantique \citep{SchulerVerbnetBroadcoverageComprehensive2005}. C'est pourquoi ils font la promotion de leur aspect sémantique via l'emploi de rôles thématiques. VerbNet emploie 23 rôles thématiques pour identifier les arguments sélectionnés par les verbes dans chaque cadre syntaxique. Il existe d'autres approches dont la numérotation des arguments\emph{Arg-1 Verbe Arg-2} comme on le voit dans PropBank \citep{PalmerPropositionBankAnnotated2005}, mais Schuler considérait que l'usage des rôles thématique permettait d'ajouter de l'information de nature sémantique. Effectivement, l'assignation d'un rôle thématique à un argument nous donne de l'information quant sur le type d'argument nécessaire pour un verbe donné. À la base, les rôles thématiques ont été mis de l'avant par Fillmore \cite{fillmore:case} et Jackendoff \cite{Jackendoff1972-JACSII-2}. Toutefois, VerbNet a créé sa propre banque de rôles thématiques. Beaucoup sont inspirés de Fillmore et Jackendoff, mais certains ont été créés pour VerbNet. Les auteurs de VerbNet précisent donc que la quantité des rôles thématiques et la qualité des rôles thématiques est assez arbitraire. Il n'y a pas de justification théorique derrière ce chiffre, mais c'est ce qu'ils ont convenu d'utiliser.Schuler voulait des rôles pouvant identifier tous les arguments possibles contenus dans les patrons de régime. Donc, des rôles assez génériques pouvant se prêter à divers cadres.Ces rôles ne sont pas spécifiques à des classes en particulier

Voici la liste des rôles thématiques qu'ils ont choisis : \texttt{actor, agent, asset, attribute, beneficiary, cause, location, destination, source, experiencer, extent, goal, instrument, material, product, patient, predicate, recipient, stimulus, theme, time, topic}.

Les rôles thématiques sont listés dans la section \lstinline|<THEMROLES>| de chaque classe verbale. Une section \lstinline|<THEMROLES>| peut revenir dans une sous-classe lorsque celle-ci possèdent des rôles thématiques plus spécifiques à cette sous-classe de verbes. Ils sont ensuite mappés aux arguments dans les cadres syntaxiques et sémantiques (qu'on peut voir aux figure \ref{cadresynt} et , \ref{cadresem}).

\begin{lstlisting}[language=XML, caption = Les rôles thématiques] % Majuscule aux captions
    <THEMROLES>
        <THEMROLE type="Agent">
            <SELRESTRS logic="or">
                <SELRESTR Value="+" type="animate"/>
                <SELRESTR Value="+" type="organization"/>
            </SELRESTRS>
        </THEMROLE>
        <THEMROLE type="Theme">
            <SELRESTRS/>
        </THEMROLE>
        <THEMROLE type="Recipient">
            <SELRESTRS logic="or">
                <SELRESTR Value="+" type="animate"/>
                <SELRESTR Value="+" type="organization"/>
            </SELRESTRS>
        </THEMROLE>
    </THEMROLES>
\end{lstlisting}

Pour les besoins de notre travail, nous n'utilisons pas les rôles thématiques, mais nous voulions souligner qu'ils étaient importants pour les créateurs de VerbNet. Comme nous utilisons la théorie Sens-Texte dans notre réalisateur profond, les rôles thématiques n'ont pas leur place dans les patrons de régime que nous avons extraits de VerbNet. Pour plus de détails concernant la non-utilisation des rôles thématiques selon la TST, nous vous renvoyons à \draft{Melcuk, Semantics: from meaning to text p.227-234}.

\subsubsection{Restrictions sélectionnelles}
Les restrictions sélectionnelles s'ajoutent aux rôles thématiques. Ces traits imposent des contraintes aux arguments possibles pour un patron de régime donné. Dans l'exemple fourni ici, on remarquera que l'\texttt{Agent} doit être soit de type animé ou doit être une organisation.

\begin{lstlisting}[language=Xml, caption = Les restrictions sélectionnelles]
    <THEMROLES>
        <THEMROLE type="Agent">
            <SELRESTRS logic="or">
                <SELRESTR Value="+" type="animate"/>
                <SELRESTR Value="+" type="organization"/>
            </SELRESTRS>
        </THEMROLE>
\end{lstlisting}

\subsubsection{Cadres syntaxiques}

Voici maintenant la section qui nous intéressait le plus: les cadres syntaxiques. Ceux-ci sont compris dans la section \lstinline{<FRAMES>} de VerbNet. À l'intérieur de cette balise, on retrouve une autre balise se nommant \lstinline{<FRAME>} qui contient la balise \lstinline{<SYNTAX>}.

\lstinline{<SYNTAX>} nous donne de l'information de nature syntaxique (et sémantique via les rôles thématiques). Elle nous présente linéairement un patron de régime d'une classe verbale donnée. On précise linéairement car les syntagmes sont listés du haut vers le bas en fonction de leur ordre de syntaxe superficielle. Bien que le format présenté ici ne représente pas la syntaxe de surface en TST, nous pouvons quand même en retiré du contenu syntaxique et le mouler par la suite. Le maniement des cadres syntaxiques de VerbNet sera décrit dans le chapitre suivant \draft{faire référence au chapitre suivant}.

Nous avons choisi VerbNet car nous voulions un dictionnaire qui énumère exhaustivement tous les comportements syntaxiques possibles que démontre un verbe (une classe verbale dans ce cas-ci). Ainsi, cette section démontre explicitement comment le verbe se combine en surface, avec quel type d'argument et quelle préposition est sélectionnée.

Ce cadre syntaxique  \ref{cadresynt} provient de la classe verbale \texttt{give-13.1}. De plus, chaque cadre syntaxique est accompagné d'une phrase servant d'exemple. Ce cadre permet la réalisation de surface \form{They lent a bicycle to me}. \lex{they} est le \lstinline{<NP value="Agent">}, \lex{lend} est le \lstinline{<VERB/>}, \lex{bicycle} est le \lstinline{<NP value="Theme">} et \lex{me} est le \lstinline{<NP value="Recipient">}.

\begin{lstlisting}[language=Xml, caption = cadres syntaxiques, label=cadresynt]

            <SYNTAX>
                <NP value="Agent">
                    <SYNRESTRS/>
                </NP>
                <VERB/>
                <NP value="Theme">
                    <SYNRESTRS/>
                </NP>
                <PREP value="to">
                    <SELRESTRS/>
                </PREP>
                <NP value="Recipient">
                    <SYNRESTRS/>
                </NP>
            </SYNTAX>
\end{lstlisting}

\subsubsection{Prédicats sémantiques}
Dans la revue de littérature de la thèse de \cite{SchulerVerbnetBroadcoverageComprehensive2005}, on remarque que les dictionnaires servant de comparaison sont généralement critiqués par leur manque d'information sémantique. C'est pourquoi Schuler a insérer un segment sémantique à VerbNet : \lstinline{<SEMANTICS>}. Cette section est constituée d'une suite de prédicats sémantiques. Chaque prédicat est décrit par une liste d'arguments qui sont, à leur tour, décrits par deux caractéristiques: \emph{type} et \emph{value}.  Le cadre sémantique ci-dessous complémente le cadre syntaxique que nous venons d'exposer en \ref{cadresynt}. Ce cadre sémantique décrit à la fois \form{They lent a bicycle to me} et \form{They lent me a bicycle}.

\begin{lstlisting}[language=Xml, caption=Les prédicats sémantiques, label=cadresem]
<SEMANTICS>
                <PRED value="has_possession">
                    <ARGS>
                        <ARG type="Event" value="start(E)"/>
                        <ARG type="ThemRole" value="Agent"/>
                        <ARG type="ThemRole" value="Theme"/>
                    </ARGS>
                </PRED>
                <PRED value="has_possession">
                    <ARGS>
                        <ARG type="Event" value="end(E)"/>
                        <ARG type="ThemRole" value="Recipient"/>
                        <ARG type="ThemRole" value="Theme"/>
                    </ARGS>
                </PRED>
                <PRED value="transfer">
                    <ARGS>
                        <ARG type="Event" value="during(E)"/>
                        <ARG type="ThemRole" value="Theme"/>
                    </ARGS>
                </PRED>
                <PRED value="cause">
                    <ARGS>
                        <ARG type="ThemRole" value="Agent"/>
                        <ARG type="Event" value="E"/>
                    </ARGS>
                </PRED>
            </SEMANTICS>
\end{lstlisting}

Cette section met fin aux composantes de VerbNet. Pour plus d'informations, nous vous invitons à consulter la thèse de Schuler \cite{SchulerVerbnetBroadcoverageComprehensive2005} et les \draft{Guidelines}

\chapter{Importation de VerbNet dans GenDR}\label{ch:python}

À l'aide du module \emph{xml.etree.cElementTree}\footnote{\url{https://docs.python.org/3/library/xml.etree.elementtree.html}, 01-06-17} créé pour le langage de programmation \emph{Python}, nous avons pu manipuler et extraire les données de VerbNet qui sont encodées en \emph{XML}. Ensuite, nous les avons compilées dans des fichiers\emph{.dict} pour les implémenter dans GenDR. Ainsi, nous créerons un dictionnaire de classes verbales, suivi d'un dictionnaire des membres de VerbNet et finalement, un dictionnaire de patron de régime. Les premiers dictionnaires seront fusionnées et intégrér à GenDR, puis le dictionnaire de patron de régime fonctionnera à part. De plus, nous avons aussi utilisé les données de VerbNet pour nous créer une banque de phrases exemples afin de tester notre système. L'extraction des exemples a aussi été effectuée dans l'environnement \emph{Python}.
 
\section{Adaptation du \emph{lexicon} de GenDR}

Le dictionnaire lexical que nous prévoyons utiliser pour GenDR se divisera en quatre sections: les classes abstraites (classes originaires de GenDR), les membres verbaux de VerbNet, les classes verbales et le reste du lexique. Pour adapter le \emph{lexicon}, il nous manque deux sections:\texttt{VERBNET CLASSES} et \texttt{VERBNET MEMBERS}. Nous décrirons dans les passages suivants comment les scripts Python que nous avons créé nous ont permi de bâtir ces futures section pour le \emph{lexicon} de la nouvelle version de GenDR.

D'abord, pour mieux comprendre la manière dont nous avons créé la section \texttt{VERBNET CLASSES}, il faut rappeler quelques notions de base du fonctionnement de GenDR. Plus tôt au chapitre \ref{chapgendr}, nous avons décrit le mécanisme d'héritage qui façonne l'architecture du lexique (section \ref{sec:dictio}). Ce mécanisme permet à une entrée d'hériter des traits d'une autre entrée. Concrètement, cela s'effectue en faisant pointer une entrée vers une autre (ex: \texttt{owe} : \texttt{verb\_dit}). Ainsi, nous n'avions pas à réécrire pour chaque verbe transitif, leur comportement syntaxique. En effet, on pouvait transmettre les propriétés du \ac{GP} des verbes transitifs à une entrée lexicale désirée. Nous avons donc importer l'architecture de VerbNet dans notre dictionnaire en ayant ce mécanisme en tête. Dans la version originale de VerbNet, les classes abstraites (ex: intransitif, transitif, ditransitif) incorporaient toute l'information syntaxique nécessaire: patrons de régime, diathèses, partie du discours, etc. Cela permettait, par exemple, à \lex{owe} d'en hériter via son association à l'une de ces classes.  

Nous reprendrons ce mécanisme pour deux objectifs. D'abord, pour reprendre l'architecture de VerbNet qui consiste à établir une hiérarchie entre les classes \scare{mères} et leurs classes \scare{filles}. Comme nous l'avons vu à la section \ref{sec:vnarchitecture}, les classes verbales sont hiérarchisées et les traits des classes dominantes sont transmis aux classes qu'elles gouvernent. Nous voulions donc reprendre cette architecture en créant la section \texttt{VERBNET CLASSES} avec le mécanisme d'héritage. Le but était de transmettre les patrons de régimes des classes \scare{mères} aux classes \scare{filles} en récupérant le mécanisme d'héritage. Puis, nous reprenons ce mécanisme pour transmettre les traits de partie du discours de la classe abstraite \texttt{VERB} afin d'éviter de répéter cette information pour chacune des classes verbales. Pour ce faire, on reprend le même concept qui lie les classes filles à leurs classes mères, mais cette fois-ci, on lie chaque classe mère à la classe abstraite VERB qui contient les \ac{SPOS} et \ac{DPOS} afin que les traits de partie du discours se transmette jusqu'à tous les lexèmes verbaux.

\begin{lstlisting}[language=XML, caption=Traits de la classe abstraite \texttt{VERB}]
verb {
  dpos = V
  spos = verb
}
\end{lstlisting}

Ensuite, nous avons extrait les verbes compris dans toutes les classes verbales de VerbNet et nous les avons soumis au mécanisme d'héritage. En effet, chaque membre pointe vers la classe ou la sous-classe qui le représente, ce qui fait en sorte qu'il hérite de tous les traits de la classe qui lui correspond. Par exemple, les lexèmes \lex{absorb}, \lex{ingest}, \lex{take in} hériteront tous les trois des traits compris dans l'entrée \texttt{absorb-39.8}. Cela nous permet de traiter 6\,393 acceptions sans avoir à décrire leur comportement syntaxique systématiquement.

\begin{lstlisting}[language=XML]
absorb: "absorb-39.8"
take_in: "absorb-39.8"
ingest_1: "absorb-39.8"
\end{lstlisting}

\subsection{Extractions de l'architecture de VerbNet}

Nous avons divisé la description du premier script en trois blocs pour faciliter la compréhension du fonctionnement de chacun de ceux-ci.

\subsubsection{Hiérarchie des classes verbales}

\textbf{L'objectif} de la fonction \emph{supers} est de recréer l'architecture de VerbNet pour pouvoir l'implémenter à GenDR. Cette fonction récupère l'identifiant de la classe \scare{mère}, puis de toutes les sous-classes imbriquées en elle. Ensuite, \textbf{la fonction} crée un dictionnaire dont la clé correspond à l'identifiant de la classe \scare{fille} et la valeur est l'identifiant de la classe \scare{mère}. Cela correspond au mécanisme qui nous permettra de transmettre les attributs d'une classe \scare{mère} à une classe \scare{fille}. Le résultat de cette fonction ressemble à \lstinline|"begin-55.1-1": "begin-55.1"|. Dans ce contexte, \texttt{begin-55.1-1} héritera des traits de la classe \texttt{begin-55.1}

\subsubsection{Création du dictionnaire de classes verbales}

\textbf{L'objectif} de la fonction \emph{treeframes} est de récupérer, pour chaque classe VerbNet, l'identifiant de la classe verbale, les identifiants des patrons de régime compris sous celle-ci et une phrase exemple. Ainsi, cette fonction rend un trio de ce type: \lstinline|{spray-9.7, NP_V_NP_destination, Jessica sprayed the wall}|.

La fonction \emph{treeframes} récupère d'abord l'identifiant de la classe \texttt{spray-9.7} puis, les identifiants des patrons de régime qui sont encodés dans la section \texttt{<FRAME>}. Ensuite, grâce aux expressions régulières, on manipule les identifiants de \ac{GP} pour qu'ils correspondent au type de code demandé par MATE. De plus, nous intégrons les prépositions régies par chaque patron de régime à l'intérieur de l'identifiant afin de les distinguer, puisque certains patrons de régime de VerbNet s'écrivent de la même manière, mais ne sélectionne pas les mêmes prépositions. Donc, lorsque nous encoderons les patrons de régime dans un dictionnaire, nous saurons quelle préposition inclure pour un patron de régime donné puisque ceux-ci se retrouveront dans l'identifiant que nous aurons repêché: \texttt{NP\_asset\_V\_NP\_PP\_from\_out\_of} est un identifiant de \ac{GP} dans lequel sont encodés les prépositions from et out of pour le troisième actant syntaxique.  

Finalement, on récupère les exemples accompagnant chaque patron de régime pour compléter le trio: identifiant de classe, de \acp{GP} et exemple. 

\subsubsection{Implémentation des données dans un fichier .dict}

L'objectif de ce dernier bloc de code est d'obtenir un dictionnaire qui contiendra les classes verbales de VerbNet avec tous les identifiants de patrons de régime que la classe englobe et les exemples qui correspondent à ceux-ci.

Pour procéder, nous ouvrons un fichier appellé \emph{lexicon.dict} dans lequel nous écrirons les informations que extrairons grâce aux deux fonctions que nous venons de créer. Nous fournirons ces fonctions à l'ensemble des documents XML qui composent la ressource VerbNet et le résultat de ce processus nous donne la section \texttt{VERBNET CLASSES} que nous implémenterons dans le dictionnaire de GenDR.

\begin{lstlisting}[language=Python, caption = Importation de l'architecture des classes verbales, label=fig:archivn]
# VERBNET HIERARCHY
def supers(t, i):
    ID = t.get('ID') # t = root of the verbal class, it contains the shared syntactic information.
    sc = {ID:i} # simulates the inheritance mechanism.
    subclasses = t.findall('SUBCLASSES/VNSUBCLASS') # gets all the information on the subclasses.
    if len(subclasses) > 0:
        for sub in subclasses:             # If there's a subclass for a given VNCLASS, 
            sc = {**sc, **supers(sub, ID)} # it'll point towards the class it's being dominated by.
    return sc
		
# EXTRACTION of GPS identification and EXAMPLES
def treeframes(t):
    ID = t.get('ID')  # gets the name of the verbnet class
    z = []            
    for frame in t.findall('FRAMES/FRAME'):
        description = re.sub(r"\s*[\s\.\-\ +\\\/\(\)]\s*", '_',  
        frame.find('DESCRIPTION').get('primary')) # primary description = identification of a GP
        if description in exclude:
            continue
        description = re.sub('PP', 'PP_{}', description) 
        preps = [p.get('value') or 
                p.find('SELRESTRS/SELRESTR').get('type').upper()                  
                for p in frame.findall('SYNTAX/PREP')+frame.findall('SYNTAX/LEX')] 
        preps = [sorted(p.split()) for p in preps] # manipulates data to insert the prep. in desc.                                
        examples = [e.text for e in frame.findall('EXAMPLES/EXAMPLE')] # get ex. for each desc.
        if len(preps)==1:
            description = description.format('_'.join(preps[0]))
        elif len(preps)==2:
            description = description.format('_'.join(preps[0]),
                                             '_'.join(preps[1]))
        elif len(preps)==3:
            description = description.format('_'.join(preps[0]), 
                                             '_'.join(preps[1]), 
                                             '_'.join(preps[2])) # inserting preps in descriptions
        z.append((description, examples))
        
    subclasses = t.findall('SUBCLASSES/VNSUBCLASS')  # gets the root of each subclasses
    subframes = [treeframes(subclass) for subclass in subclasses] # applies function to subclasses
    subframes = sum(subframes, []) # flatten list of lists
    return [(ID, z)] + subframes # returns list of (sub)class, GP-identification and example
		
# WRITING of the EXTRACTED INFORMATION in a DICT FILE
with open('lexicon.dict','w') as f: # we are going to write all of this block into lexicon.dict
    f.write('lexicon {\n')
    for file in [f for f in os.listdir('verbnet') if f[-4:] == '.xml']: # open VerbNet XMl files
        root = ET.parse('verbnet/'+file).getroot() # Applies the Python Element Tree module
        d = dict(treeframes(root)) # makes a dictionary out of the results of treeframes on a file
        sc = supers(root, 'verb') # applies supers function to each file
        for c in d.keys():
            f.write('"'+c+'"')
            if sc[c] == 'verb': # all non-dominated classes point to the default verb class
                f.write(': ' +sc[c] + ' {') 
            else:
                f.write(': ' +'"'+sc[c]+'"' + ' {') #dominated classes point towards their governor
            [f.write('\n  gp = { id=' + gp[0] + (max(len(gp[0]), 30)-len(gp[0]))
                     *' ' + ' dia=x } // ' + ' '.join(gp[1])) for gp in d[c]]
            f.write('\n}\n') # all GPs will have attributes: id and dia
    f.write('\n}')
\end{lstlisting}

\subsection{Création du dictionnaire des membres verbaux} \label{extracmembre}

Maintenant que nous avons extrait les classes verbales et l'architecture de celles-ci, il ne nous reste qu'à peupler le dictionnaire des lexèmes encodés dans la section \texttt{<MEMBERS>} de chaque classe de VerbNet. De cette manière notre \emph{lexicon} pourra effectivement couvrir une immense partie de l'anglais grâce au travail de %\cite{SchulerVerbnetBroadcoverageComprehensive2005}. À cette étape, nous procèderons aussi à la désambiguïsation des verbes. Schuler a distingué les différentes acceptions d'un même vocable en leurs assignant des classes verbales différents en fonction des différents sens d'un vocable.

\subsubsection{Récupérer les membres}
L'objectif de la fonction \emph{treemember} est de récupérer les membres assignés à chaque classe et sous-classe verbale encodés dans VerbNet.

Cette fonction récupère d'abord l'identifiant de la classe verbale puis les membres lui correspondant, et répète l'opération pour les sous-classes, puis la fonction retourne des paires de classes verbales et de membres. Par exemple, le script récupèrera deux fois le vocable, une fois de cette manière \lstinline| order : "get-13.5.1"|, puis, \lstinline| order : "order-60-1"|.

\subsubsection{Désambiguiser les membres}

L'objectif de ce bloc est de désambiguïser les différentes acceptions d'un même vocable. L'exemple que nous venons d'illuster pour le lexème \lex{order} démontre la nécessité de faire cette opération.

Pour remédier à la situation, nous avons extrait toutes les acceptions de mêmes vocables (qu'on a appelé des duplicats). Puis, nous les avons distingué en leur assignant un numéro (ex: \texttt{grill 1}, \texttt{grill\_2}, \texttt{grill\_3}). Finalement, nous avons identifié les vocables à une acception et nous les avons fusionné à la liste qui contenait tous les lexèmes désambiguïsés, ce qui résulte en l'entièreté des membres de VerbNet.

\subsubsection{Implémentation des membres dans un fichier .dict}

L'objectif de ce bloc est de créer le dictionnaire \emph{members} qui sera éventuellement intégré au \emph{lexicon} comme étant la section \texttt{VERBNET MEMBERS}.

Pour ce faire, nous classons d'abord les membres en ordre alphabétique, puis nous assignons, à chaque acception, la classe verbale qui lui correspond (ex: \lstinline|order_1 : "get-13.5.1"|).

\begin{lstlisting}[language=Python, caption = Ajout des membres de VerbNet, label=scriptmember]
#GET MEMBERS FROM VERBNET CLASSES
def treemember(t):
    ID = t.get('ID') # get ID of the VNCLASS
    members = [m.get('name') for m in t.findall('MEMBERS/MEMBER')] # get members 
    subclasses = t.findall('SUBCLASSES/VNSUBCLASS')
    submembers = []
    if len(subclasses) > 0: # if there's a subclass
        for sub in subclasses:
            submembers = submembers + treemember(sub) # get ID of the subclass and members
    return [(ID, members )] + submembers

# DISAMBIGUATE MEMBERS
files = [f for f in os.listdir('verbnet') if f[-4:] == '.xml']
members = dict(sum([treemember(ET.parse('verbnet/'+file).getroot())
 for file in files], [])) # VNCLASS : [member1, member2, etc. ]

values = sum(list(members.values()), []) # just the members of all classes

dups = {m:[ID for ID in members.keys() if m in members[ID]]
 for m in values if values.count(m)>1}  # get all the duplicates

lexemes = {d[0]+'_'+str(n+1):d[1][n]
 for d in dups.items() for n in range(len(d[1]))} # enumerate all duplicates: eat_1, eat_2

unique_member = {m:ID for ID in members.keys() 
 for m in values if m in members[ID] and values.count(m)==1} #  get all unique lexemes

unified_dict = {**unique_member, **lexemes} # fuse both dict. to get all members disambiguated

# WRITE MEMBERS IN A FILE
with open('members.dict','w') as f: # open a file
    f.write('members {\n')
    for key in sorted(unified_dict.keys()): # sort the members
        f.write(key) # write the members
        f.write(': '),
        f.write('"'+str(unified_dict[key])+'"') # point members towards ID of VNCLASS
        f.write('\n')
    f.write('\n}\n')
\end{lstlisting}

\section{Création du dictionnaire de patrons de régime}

Lors de la création de la section \texttt{VERBNET CLASSES}, nous avons extrait les classes verbales de VerbNet ainsi que les identifiants des patrons de régime qui sont encodés dans ces classes. Nous avons analysé le contenu des patrons de régime prélevés afin de trouver une manière d'encoder les propriétés lexicales de chaque patron de régime dans un dictionnaire prêt à cet effet.

L'analyse du contenu des patrons de régime nous a fait remarqué qu'il y avait certains comportements syntaxiques encodés dans les classes verbales dont nous voulions nous débarrasser. Il s'agissait en quelque sorte de faire un tri initial des \acp{GP} qui pourraient poser problèmes et des \acp{GP} que nous jugions inutiles. 

Nous avons d'abord exclu les constructions stylistiques comme \lstinline|NP_location_V_NP| qui réalise ce type de phrase \form{All through the mountains raged a fire.}. Puis, nous avons aussi exclu les \acp{GP} qui sélectionnent des constructions de type \emph{déplacement qu-}, car GenDR ne traite pas ces comportements pour l'instant. Il a aussi fallu enlever les constructions contenant des modificateurs de types adverbiaux ou adjectivaux, par exemple \lstinline|NP_V_ADVP_Middle_PP_into_to_with| qui s'exemplifie par la phrase \form{The computer connected \textbf{well} to the network.}. L'emploi d'un groupe adverbial dans un patron de régime n'a pas sa place selon nous, car il ne s'agit pas d' un actant syntaxique et il n'est certainement pas sélectionné par le verbe. Au contraire, c'est l'adverbe qui le modifie.

\draft{refaire ce paragraphe}Après avoir fait filtrer le tout, nous avons donc procédé à la création du dictionnaire de patron de régime. Nous en sommes venus à la conclusion que nous ne pouvions pas directement extraire les propriétés lexicales des \acp{GP} de VerbNet, bien que c'était l'objectif initial. Comme nous l'avons vu plus tôt, un patron de régime englobe (diathèse, combinatoire lexicale et syntaxique d'une lexie (dpos, prep et rel)), mais les cadres syntaxiques de VerbNet ne nous donnait pas complètement ce X (pas diathèse, mais rôles théamtiques, la dpos (NP pour dpos=N ou S\_INF/S\_ING pour dpos=V), les prépositions et les relations via les phrases exemples (implicitement)). Bref, on avait tout ce dont on avait besoin avec les identifiants de GP extraits pairé avec les phrases exemples. Cependant, les rôles thématiques ne remplacent en aucun cas la diathèse, c'était un morceau qui aura fallu encoder manuellement.

Grâce à ces informations, qui sont encodées dans la section \texttt{<FRAMES>} de VerbNet, mais que nous avons extraits et encodé, nous avons pu créer de toutes pièces un dictionnaire de patron de régime (\emph{gpcon}).

Pour mieux comprendre le code qui nous a permi de créer le \emph{gpcon}, nous décrirons brièvement les étapes importantes qui ont mené à sa complétion. Donc, on prenait la description qui nous donnait les: dpos, relation, et préposition lié à un actant. Puis on créait une entrée de gp à partir de ça. À l'aide d'une fonction, nous avons pu parcourir toutes les calsses verbales de VerbNet et trouver qu'il y avait 274 identifiants de GP unique. Nous savions donc combien de gp il nous fallait coder. Pour procédéer à la chose, nous avons donc pris chaque gp unique et en avons fait une entrée dans notre dictionnaire. Nous l'avons encodé manuellement puisqu'aucune partie de VerbNet ne nous permettait de l'importer directement puisque le cadre théorique dans lequel nous nous insérons et GenDR demande que les patrons de régime soient décrits d'une certaine façon. De plus, comme ni la diathèse, ni relations de surface sont explicités, il fallait le faire nous-même. Pour accélérer le processus nous avons défini des objets qui représentent des actants syntaxiques non-étiquettés, mais qui incorprorent les traits lexicaux qu'un tel actant syntaxique demanderait.

{définir les objets, définir une fonction pour prendre les objets et leur attribuer une étiquette syntaxique, puis créer les régimes grâce à ça}

\subsection{Propriétés des actants syntaxiques}

Dans le premier bloc, nous définissons des objets comme subj ou dir\_N qui correspondent à des propriétés syntaxiques qu'on attribue normalement à un actant syntaxique. Par exemple, l'objet subj contient les traits suivants \lstinline|subj = 'rel=subjective dpos=N'|, qui sont généralement attribués au premier actant syntaxique. Nous avons ainsi défini toutes les propriétés syntaxiques possibles des actants que nous avons relevé dans les \acp{GP} de VerbNet. Ainsi, ces objets correspondent à des raccourcis pour faciliter l'encodage des \acp{GP} individuels. Puisque nous avons codé chaque patron de régime manuellement, il nous fallait un raccourci pour décrire les propriétés d'un actant syntaxique. Par exemple, au lieu d'écrire au complet \lstinline|I = { rel = subjective dpos = N}| à chaque fois qu'un patron de régime a un tel actant syntaxique, nous n'avons qu'à spécifier l'objet défini dans cette section. 

Cela nous a permi de spécifier l'objet dans la description du régime au lieu de toutes les propriétés de l'actant syntaxique à chaque fois. Cela a grandement augmenté l'efficacité de l'encodage des 274 patrons de régime. Par exemple, \lstinline|NP_agent_V_NP| est décrit par les éléments: [subj, dir\_N] qui sont respectivement encodé en fonction de leur relation syntaxique de surface. Donc le système vérifie que contient l'item \texttt{subj} puis récupère les valeurs à l'intérieur \lstinline|subj = 'rel=subjective dpos=N'| et il répète l'opération jusqu'à ce que toutes les propriétés syntaxiques soient récupérées.

\subsection{Assignation des étiquettes syntaxiques profondes aux actants}

L'objectif de la combinaison des fonctions \emph{roman} et \emph{gp} est d'assigner les étiquettes des actants syntaxiques selon leur emplacement dans la description d'un \ac{GP}. Par exemple, dans \lstinline|NP_agent_V_NP: [subj, dir_N]|, puisque \texttt{dir\_N} est à la deuxième position, les propriétés qu'il renferme seront encodés dans un actant syntaxique \texttt{II}. En effet, si on se fie au produit final, nous avons dans le dictionnaire de patron de régime: \lstinline|NP_agent_V_NP { I={rel=subjective dpos=N} II={rel=dir_objective dpos=N} }|.  Cette fonction nous permet aussi de créer, par exemple, deux fois le même actant syntaxique pour un même \ac{GP}, afin de tenir compte du fait que certains actants sélectionnent deux prépositions: \lstinline| III={rel=oblique dpos=N prep=from}| et \lstinline|III={rel=oblique dpos=N prep="out of"}|. 

Ce qui est illustré par le \ac{GP} \lstinline|NP_asset_V_NP_PP_from_out_of| qui se décrit par les items suivants: \lstinline|[subj, dir_N, [from_N, out_of_N]]|

\subsubsection{Construction d'une entrée de \ac{GP}}


\subsubsection{Création du gpcon}
Finalement, nous créons le dictionnaire \emph{gpcon} en imposant la fonction \emph{gp} à toutes les descriptions de \acp{GP} pour que l'actant syntaxique (I,II,etc.) soit réalisé et pour qu'il renferme les propriétés syntaxiques qui lui sont associées.

\begin{lstlisting}[language=Python, caption = code pour gpcon.dict]

# 1 SYNTACTIC ACTANTS PROPERTIES
#subjective
subj = 'rel=subjective dpos=N'

#direct obj
dir_N = 'rel=dir_objective dpos=N'
dir_V_ING = 'rel=dir_objective dpos=V finiteness=GER'
dir_V_INF = 'rel=dir_objective dpos=V finiteness=INF'

#indirect obj
to_N = 'rel=indir_objective dpos=N prep=to'
indir_N = 'rel = indir_objective dpos = N'

#oblic
on_V = 'rel=oblique dpos=V prep=on'
to_obl_N = 'rel=oblique dpos=N prep=to' 
for_obl_N = 'rel=oblique dpos=N prep=for'
against_N = 'rel=oblique dpos=N prep=against'
...

# 2 ASSIGN SYNTACTIC LABEL to ACTANT
def roman(n):
    return ['I', 'II', 'III', 'IV', 'V', 'VI'][n-1] # transforms arabic numbers in roman numbers
def gp(name, real_actant):
    s = name + ' {\n'
    i=0
    for actant in real_actant:
        i = i+1                   # starts to enumerate at 1
        if type(actant) == list:  # if not actant but list, apply function to actants in list
            for y in actant:
                s = s + "   " + roman(i) + "={" + y + "}\n"
        else:
            s = s + "   " + roman(i) + "={" + actant + "}\n" # apply function to actant
    s = s + '}\n'
    return s 

# 3 CONSTRUCTION OF A GP ENTRY
descriptions = {
'NP_agent_V': [subj],
'NP_agent_V_NP': [subj, dir_N],
'NP_asset_V_NP_PP_from_out_of': [subj, dir_N, [from_N, out_of_N]],
...

# 4 GPCON CREATION
with open('gpcon.dict','w') as f: 
    f.write('gpcon {\n')
    for d in descriptions.keys():       # for each gps descriptions,
        f.write(gp(d, descriptions[d])) # write them with the correct syntactic label
    f.write('}')
\end{lstlisting}

\section{Scripts pour l'évaluation de GenDR}

Pour évaluer la qualité de l'implémentation de VerbNet à GenDR, nous avons extrait les phrases exemples qui figurent dans VerbNet dans le but d'en faire des inputs et de tenter de les générer. Autrement dit, nous avons pris les phrases exemples accompagnant chaque cadre syntaxique, et nous voulions créer des graphes sémantiques à partir de ces phrases, dans le but de les donner à GenDR pour qu'il tente de générer la phrase de départ.

\subsection{Extraction des exemples}

L'\textbf{extraction des exemples} se fait grâce à la fonction \emph{treeframe} que nous reprenons du script ayant servi à extraire les identifiants de \acp{GP} (voir figure~\ref{fig:archivn}). L'objectif est de parcourir les fichiers XML pour uniquement récupérer les phrases exemples contenues dans les classes verbales et sous-classes.

Le second bloc de code s'occupe de la \textbf{compilation des phrases} extraites dans le fichier \emph{phrases.txt}, en les divisant ligne par ligne. 

\begin{lstlisting}[language=Python, caption = Extraction des phrases exemples de VerbNet]
#EXAMPLES EXTRACTION
def treeframes(t):
    z = []
    for frame in t.findall('FRAMES/FRAME'): # for each syntactic frame
        description = re.sub(r"\s*[\s\.\-\ +\\\/\(\)]\s*", '_',
				frame.find('DESCRIPTION').get('primary'))
        if description in exclude:
            continue    
        examples = [e.text for e in frame.findall('EXAMPLES/EXAMPLE')] # get the examples
        z =  z + examples 
    subclasses = t.findall('SUBCLASSES/VNSUBCLASS')
    subframes = [treeframes(subclass) for subclass in subclasses] #repeat operation for subclasses
    subframes = sum(subframes, []) # flatten list of lists
    return z + subframes

#LIST OF SENTENCES IN THE FILE phrases.txt
liste=[]
with open('phrases.txt','w') as f:
    for file in [f for f in os.listdir('verbnet') if f[-4:] == '.xml']:
        root = ET.parse('verbnet/'+file).getroot()       
        d = (treeframes(root))  # Applies treeframes function to all of VerbNet files
        finale_liste = liste + d
        [f.write(x+'\n') for x in finale_liste] # returns line after each example

\end{lstlisting}

\subsection{Création des structures sémantiques}\label{sec:pythonstruc}

Ce script crée les graphes de bases qui nous permettront de faire les tests. Ces inputs en préparation sont dépourvus de n\oe{}uds et d'arcs. Ils ne contiennent que des éléments non-sémantiques qui faciliteront l'encodage des inputs.

Pour ce faire, nous ouvrons le fichier \emph{phrases.txt} qui contient chaque phrase exemple, ligne par ligne. Le script créera 978 structures sémantiques vides pour les 978 phrases extraites de VerbNet. Il identifiera chaque structure de 001 à 978. Les stuctures créées par ce script ne contiendront que le texte à représenter sémantiquement et les crochets \{ \} nécessaires pour encadrer le graphe. Finalement, le script créera 978 structures.

\begin{lstlisting}[language=Python, caption = Code pour créer les structures sémantiques vides, label=structurepython]
phrases = open('phrases.txt','r')

with open('structures.str','w') as f: # create a .str structure
    for(i,p) in enumerate(phrases):   # for each sentence
        with open('s'+str(i)+'.str','w') as g:
            structure = 'structure Sem S'+str(i)+' # name each structure by enumeration
						{\n S {text="'+p.strip()+'"\n\n main-> \n }\n}' # insert as texte the sentence
            f.write(structure)
            g.write(structure)
\end{lstlisting}


%!TEX root = ../memoire.tex

\chapter{Implémentation}\label{ch:implementation}

Au chapitre précédent, nous avions extraits des informations précieuses de VerbNet à l'aide de scripts Python. Nous avons ainsi fait usage de ces informations en les implémentant dans GenDR. Ce chapitre est séparé en trois parties. D'abord nous expliquons comment les dictionnaires fonctionnent après l'implémentation et comment ils communiquent entre eux. Puis, nous allons reconstruire la réalisation d'un arbre syntaxique de surface en expliquant comment les dictionnaires et règles de grammaires intéragissent dans cette nouvelle version de GenDR. Puis finalement, nous traiterons de l'évaluation du système et les modifications nécessaires pour un meilleur rendement.

\section{Implémentation des verbes et de leurs patrons de régime: lexicon.dict et gpcon.dict}

Au chapitre \ref{chapgendr}, nous avions démontré comment l'information lexicale s'encodait dans GenDR. Nous avions deux dictionnaires: un dictionnaire de sémantème et un dictionnaire de lexème. À la suite des informations extraites sur les patrons de régime, nous avons maintenant un troisième dictionnaire: un dictionnaire de patron de régime. L'implémentation de cette ressource a engendré un ajustement du lexicon pour tenir compte de la nouvelle architecture qui règne dorénavant entre elle et le dictionnaire de gp.

\subsection{Lexicon.dict 2.0}
Comme nous l'avons vu à la section \ref{sec:dictio}, GenDR traitait les verbes et leurs comportements syntaxiques directement dans le lexicon. Dorénavant, le lexicon est séparé en diverses sections. La première section \texttt{DEFAULT ATTRIBUTES} décrit les classes générales de GenDR. Elle contient les verbes, les noms, les adjectifs, les adverbes, les prépositions et les classes que nous avons vu précédemment qui traite les lexèmes qu'on ne veut pas encoder dans le dictionnaire: montant, date, lieu, noms propres, acronymes,etc. Les grandes classes possèdent très peu d'information syntaxique, car dorénavant cela va dans le gpcon. Elles contiennent néanmoins des informations très importantes. Notamment, la partie du discours, des traits morpho-syntaxiques et l'identifiant du patron de régime à utiliser par la classe. Seuls les verbes n'ont pas d'identifiant de patrons de régime dans cette section, puisqu'il s'agit de la classe de lexèmes qui a les patrons de régime les plus irréguliers. Ceux-ci sont encodés dans les classes verbales de VerbNet que nous avons extrait. En ce qui concerne les autres classes, nous leurs avons imposé un patron de régime par défaut qui permet pour l'instant de réaliser une grande quantité de construction syntaxiques. Vous pouvez en constater un exemple dans la figure~\ref{classedef} sous la classe \texttt{NOUNS}.

\begin{lstlisting}[language=XML, caption = Attributs par défaut des classes, label=classedef]
/*
=======================================================
                  DEFAULT ATTRIBUTES
=======================================================
*/

// ================= VERBS =================

// VERB
// ----

verb {
  dpos = V
  spos = verb
}

// ================= NOUNS =================

// NOUN
// ----
// Common nouns.

noun {
  dpos = N
  spos = noun
  countable = yes
  gp = { id=NP dia=1}
}
\end{lstlisting}

La prochaine section \texttt{VERBNET MEMBERS} contient les membres des classes verbales de VerbNet que nous avions extrait avec le script Python (voir figure \ref{scriptmember}). Sont listés tous les 6393 verbes ainsi que la classe de VerbNet (ou la sous-classe) qui leur correspond. C'est aussi dans cette section que la désambiguisation des verbes est explicitée. Tel que nous l'avons démontré à la section précédente, nous avions extraits les membres de classes de VerbNet, puis nous avons désambiuiser les formes identiques puisque certains verbes ont la même forme, mais des sens différents. Cette partie de la section en montre un exemple. On a désambiguiser la forme \lex{order} en répertoriant qu'elle apparaissait à deux reprises dans le corpus de VerbNet. Dans le premier cas elle pointait vers la classe \texttt{get-13.5.1} et il s'agit du sens \sem{passer une commande} tandis que le deuxième signfie \sem{donner un ordre} \texttt{order-60-1}.

\begin{lstlisting}[language=XML, caption = Partie membre du lexicon]
/*
 =======================================================
                      VERBNET MEMBERS
 =======================================================
*/
"open up" : "establish-55.5-1"
operate : "other_cos-45.4"
oppose : "amalgamate-22.2-3"
ordain : "appoint-29.1"
order_1 : "get-13.5.1"
order_2 : "order-60-1"
organize_1 : "create-26.4"
organize_2 : "establish-55.5-1"
organize_3 : "force-59-1"
originate : "establish-55.5-1"
ornament_1 : "butter-9.9"
ornament_2 : "fill-9.8"
ornament_3 : "illustrate-25.3"
\end{lstlisting}

La troisième section est \texttt{VERBNET CLASSES}. Cette section décrit les diverses classes de VerbNet en deux traits. D'abord, la diathèse est décrite différement que dans GenDR 1.0 . Ce trait décrit la correspondance des actants sémantiques et syntaxiques en précisant l'ordre. Par exemple, une diathèse où le premier actant sémantique est le premier actant syntaxique, mais le troisième actant sémantique est le deuxième actant syntaxique sera représentée ainsi:  dia=132 (ça implique I:1 II:3 III:2). Ces informations permettront à GenDR de faire la correspondance des actants entre la RSem et la RSyntP. Les informations contenues (restrictions, etc.) dans les actants syntaxiques sont encodées dans le dictionnaire de patron de régime. Puis finalement, chaque classe verbale est dotée d'un trait \texttt{id} qui dicte au système quel patron de régime utiliser pour cette classe verbale en fonction de la diathèse imposée.

Finalement, c'est par l'entremise des trois sections que nous venons de vous présenter que nous avons implémenté l'architecture de VerbNet dans notre système. Le mécanisme d'héritage des traits que nous avions exposé à la section \ref{sec:dictio} est réutilisé autrement. Les membres pointe vers les classes ou les sous-classes de VerbNet. Les sous-classes pointent vers les classes qui les domine, et les classes non-dominées pointe vers la classe 'verb' qui contient les attributs par défaut des verbes. Ce mécanisme d'héritage devrait pouvoir transmettre les paires de patrons de régime et de diathèse ainsi que les attributs par défaut. Si le système fonctionne bien, le mécanisme d'héritage nous permet de désaturer le dictionnaire et de calquer l'architecture de VerbNet correctement. Donc, les identifiants des gp ont une entrée dans le gpcon. C'est là qu'on décrit explicitement les comportements syntaxiques régis par un gp donné.

\begin{lstlisting}[language=XML, caption = Partie: Classes de VerbNet]
/*
=======================================================
                   VERBNET CLASSES
=======================================================
*/

"tell-37.2": verb {
  gp = { id=NP_V_NP  
	       dia=12 } // John informed me.
  gp = { id=NP_V_NP_PP_of_topic  
	       dia=123 } // John informed me of the situation. }
"tell-37.2-1": "tell-37.2" {
  gp = { id=NP_V_NP  
	       dia=12 } // Ellen told a story.
  gp = { id=NP_V_NP_PP_to_recipient 
		     dia=123 } // Ellen told a story to Helen.
  gp = { id=NP_V_NP_Dative_NP   
	       dia=132 } // Ellen told Helen a story. Ellen told me, 'Leave the room.'
  gp = { id=NP_V_NP
		     dia=13 } // Ellen told Helen.
  gp = { id=NP_V_NP_PP_about_topic
		     dia=132 } // Ellen told Helen about the situation.
}
\end{lstlisting}

Finalement le lexicon contient le reste du lexique: noms, adjectifs, adverbes, prépositions, déterminants,etc. Ces entrées proviennent de la version originale de GenDR \citep{lareau18} et elles ont été enrichies par le lexèmes qu'on retrouve dans les phrases exemples de VerbNet. Nous les avons rajouté manuellement. Bref, les entrées de cette section pointent vers leurs classes \texttt{NOUNS} ou \texttt{PREPOSITIONS} par défaut où elles héritent des attributs suivants: partie du discours, identification de gp, et diathèse.

\begin{lstlisting}[language=XML, caption = Partie: Unités lexicales non-verbales]
/*
=======================================================
               NON-VERBAL LEXICAL ENTRIES     
=======================================================
*/
accountant : noun
acorn : noun
acquaitance  : noun
across : preposition
\end{lstlisting}

\subsection{gpcon.dict}
Le gpcon est un dictionnaire de patron de régime qui contient 278 identifiants uniques de patrons de régime. Il store l'information associée à l'identification des gp. Nous avons décidé de mettre les patrons de régime à part pour alléger le lexicon. Effectivement, dans le cas contraire, on aurait dû expliciter les comportements syntaxiques de chaque verbe à l'intérieur même du dictionnaire. Considérant que la plupart des classes verbales ont plusieurs patrons de régime asssociés, le lexicon aurait été extrêmement saturé d'information. De plus, un grand nombre de patron de régime est partagé parmi les classes verbales avec le classement de VerbNet. Donc, nous réutilisons cette composante à notre avantage. D'ailleurs, cette manière de procéder est aussi utilisée par FORGe \citep{DBLP:conf/semeval/MilleCBW17, MilledemoFORGePompeu2017}.

Puis à l'intérieur on spécifie les propriétés syntaxiques. On spécifie les caractéristiques des actants syntaxiques. Puis les informations syntaxiques dans les actants sont utiles pour que le premier actant est contraint d'être un N par exemple, puis son deuxième un V. Puis de l'information de surface: la relation. Cela fait en sorte que si on prend le premier GP \lstinline! NP_agent_V { I={rel=subjective dpos=N} }!, quand on utilise le patron de régime identifié comme Np agent V on veut que son premier actant syntaxique soit de type nominal et que sa réalisation de surface est une relation subjective. Nous avons aussi instauré un mécanisme pour tenir compte du fait que certains patrons de régime permettent deux prépositions qui compétitionnent pour le même actant syntaxique. c'est le cas quand on regarde le gp \lstinline!NP_asset_V_NP_PP_from_out_of! qui a dans son régime \lstinline!III={rel=oblique dpos=N prep=from}! et \lstinline!III={rel=oblique dpos=N prep="out of"}!. Ainsi, cela permet de paraphraser encore plus et nous pouvons tenir compte du fait que VerbNet avait spécifié cela. 

Cependant, ce dictionnaire n'est pas sans failles. Nous nous sommes rendu compte qu'il existait des doublons dans notre dictionnaire. La cause de ces doublons prend vie dans le fait que VerbNet utilise les rôles thématiques pour identifier les actants syntaxiques. Cela permet que l'information contenue dans le gp Np agent V et NP attribute V est la même puisqu'ils ont les mêmes propriétés syntaxiques. La seule différence étant l'identification du NP avec un rôle thématique différent. Comme nous n'utilisons pas cette terminologie pour identifier les actants syntaxiques,  ce scénario a tendance à se répète. Cette situation n'est pas encombrante pour l'évaluation, mais le système gagnerait à régler ce problème pour en alléger le contenu.

\begin{lstlisting}[language=XML, caption = Gpcon]
NP_agent_V {
   I={rel=subjective dpos=N}
}
NP_agent_V_NP {
   I={rel=subjective dpos=N}
   II={rel=dir_objective dpos=N}
}
NP_asset_V_NP_PP_from_out_of {
   I={rel=subjective dpos=N}
   II={rel=dir_objective dpos=N}
   III={rel=oblique dpos=N prep=from}
   III={rel=oblique dpos=N prep="out of"}
}
NP_attribute_V {
   I={rel=subjective dpos=N}
}
\end{lstlisting}

\section{Implémentations de nouvelles règles de grammaire}
Nous avons ainsi terminé de décrire les dictionnaires. Il ne nous reste qu'à revisiter la grammaire de GenDR pour compléter le survol des modifications du réalisateur. Pour ce faire, nous présenterons un exemple de réalisation décrivant l'intéraction des dictionnaires et des nouvelles règles de grammaire. 

\subsection{Input}
Nous avons décidé de générer la phrase: \form{The teacher talked about history to the students.}. La figure \ref{text-input} représente la structure sémantique que nous avons donné en input au système. Le noe{}ud dominant est \sem{talk\_3} et il lie 3 actants sémantiques: \sem{teacher}, \sem{student} et \sem{history}. Chaque noe{}ud se fait attribuer les traits grammaticaux nécessaires (le temps, le nombre et la définitude) à la réalisation de la phrase visée.

\begin{lstlisting}[language=XML, caption=Input textuel, label=text-input]
structure Sem S {
  S:1{
    talk_3:1{
      tense=PAST 
      1-> teacher:1
      2-> student:1
			3-> history:1
    }
    teacher:1{number=SG definiteness=DEF}
    history:1{number=SG definiteness=NO}
    student:1{number=PL definiteness=DEF}
    main-> talk_3:1
  }
}
\end{lstlisting}

Cet input permet de générer neuf structures syntaxiques profondes. Elles correspondent aux phrases suivantes:
\begin{easylist}[enumerate]
  & \form{The teacher talked}
	& \form{The teacher talked to the students}
	& \form{he teacher talked with the students}
	& \form{the teacher talked to the students about history}
	& \form{The teacher talked with the students about history}
	& \form{the teacher talked}
	& \form{the teacher talked about history to the students}
	& \form{the teacher talked about history with the students}
	& \form{the teacher talked about history}
\end{easylist}

Ces neuf réalisations découlent des patrons de régime que permet le lexème \lex{talk\_3}. Effectivement, puisque celui-ci pointe vers la classe \texttt{"talk-37.5"}, il hérite des neufs patrons de régime encodés dans cette classe verbale. Toutes les constructions ont été réalisées parce que les patrons de régime satisfaisaient les contraintes demandées par l'input en \ref{text-input}. Celui-ci contenait 3 arguments: les actants sémantiques 1,2 et 3. Tous les patrons de régime de la classe \texttt{"talk-37.5"} ont des diathèses permettant de réaliser l'input. Effectivement, un patron de régime peut s'appliquer dès que tous les actants sémantiques s'y retrouvent ou si une partie des actants s'y retrouvent. Ce mécanisme provient d'une règle de grammaire que nous avons créé (nous y reviendrons plus tard). Nous avons choisi de représenter la réalisation de la 7e phrase (donc le 7e patron de régime).

\begin{lstlisting}[language=XML, caption=Traits \emph{gp} de la classe \texttt{talk-37.5}]

"talk-37.5": verb {
  gp = { id=NP_V                           dia=1 } // Susan talked.
  gp = { id=NP_V_PP_to_co_agent            dia=12 } // Susan talked to Rachel.
  gp = { id=NP_V_PP_with_co_agent          dia=12 } // Susan talked with Rachel.
  gp = { id=NP_V_PP_to_co_agent_PP_about_topic dia=123 } // Susan talked to Rachel about the problem.
  gp = { id=NP_V_PP_with_co_agent_PP_about_topic dia=123 } // Susan talked with Rachel about the problem.
  gp = { id=NP_V                           dia=12 } // Susan and Rachel talked.
  gp = { id=NP_V_PP_about_topic_PP_to_co_agent dia=132 } // Susan talked about the problem to Rachel.
  gp = { id=NP_V_PP_about_topic_PP_with_co_agent dia=132 } // Susan talked about the problem with Rachel.
  gp = { id=NP_V_PP_about_topic            dia=13 } // Susan talked about the problems of modern America.
}
\end{lstlisting}

\begin{lstlisting}[language=XML, caption=Informations sur le patron de régime sélectionné, label=gpexemple]

NP_V_PP_about_topic_PP_to_co_agent {
   I={rel=subjective dpos=N}
   II={rel=oblique dpos=N prep=about}
   III={rel=indir_objective dpos=N prep=to}
	}
\end{lstlisting}

\subsection{Création et lexicalisation de la racine}
D'abord, comme dans l'ancienne version de GenDR, la première règle appliquée est \emph{root\_standard}. Cela crée la racine de l'arbre et impose que la partie du discours doit être un verbe et que ce verbe sera fini (afin d'exclure la construction d'un arbre à partir d'un verbe à l'infinitif). La racine correspondra au noe{}ud dominant identifié dans l'input. S'ensuit de la lexicalisation de la racine par \lex{talk\_3} qui satisfait les contraintes du noe{}ud et qui est la supposément correspondance de \sem{talk\_3}. Cette lexicalisation se fait grâce à \emph{lex\_guess\_from\_lexicon} qui est une règle de secours (voir la section \ref{secours}). La figure \ref{deroulement0} expose l'application de la première règle.

\begin{figure}[htb]
	\centering
	\includegraphics[width=0.4\textwidth, trim = {0cm 0cm 0cm 0cm},clip]{ch6/figs/root.png}
	\caption{Création de la racine à partir du noe{}ud dominant}
	\label{deroulement0}
\end{figure}

\subsection{Sélection du patron de régime dans le lexicon}:
Ensuite, une fois que le noe{}ud dominant est lexicalisé, la règle \emph{actant\_gp\_selection} est déclenchée. Celle-ci permet à GenDR de récupérer les traits encodés dans gp. Puis, à l'intérieur de gp, il y a les traits \texttt{id} et \texttt{dia}. Ces traits sont donc récupérés par la règle et apposé sur le noe{}ud racine. La racine est maintenant contrainte d'utiliser le patron de régime x si la diathèse qu'elle a correspond aux mêmes actants demandés par la structure d'input.

\begin{figure}[htb]
	\centering
	\includegraphics[width=0.4\textwidth, trim = {0cm 0cm 0cm 0cm},clip]{ch6/figs/selectiongp.png}
	\caption{Application de la règle actant\_gp\_selection}
	\label{deroulement1}
\end{figure}

\subsection{Application de la règle actancielle: \emph{actant\_gp\_ijk}}
À l'étape précédente, le noe{}ud \lex{talk\_3} se fait imposer les restrictions suivantes: une paire identifiant de gp et diathèse. Ces traits sont essentiels à l'application des règles actancielles. La règle actant\_gp\_ijk est sélectionnée lorsque la diathèse précise qu'il y a trois actants sémantiques. Il faut aussi que les actants sémantiques que la diathèse précise se retrouve dans la structure sémantique donnée. Sinon, aucune règle actancielle n'est appliquée et la réalisation s'interrompt laissant une racine comme output. 

Ce mécanisme est nouveau puisque dans l'ancienne version de GenDR, le système analysait liaison actancielle individuellement, puis faisait correspondre cet liaison sémantique à une relation syntaxique. Cela se traduisait par la création d'un arc entre la racine et un noe{}ud vide et on y ajoutait des contraintes sur ce noe{}ud simultanément. La version actuelle de GenDR ne fonctionne plus ainsi. Une fois que le gp est sélectionné, on confirme à l'aide du trait dia quel règle actancielle il faudra choisir. Dans notre exemple, ce sera celle-ci puisqu'elle correspond à un gouverneur sélectionnant trois actants.

La règle actant ijk, crée 3 arcs en partance talk\_3 au bout desquels se trouvent des noe{}uds vides sans contraintes. Elle s'occupe aussi du passage des arcs sémantiques à syntaxique. Le patron de régime qui nous concerne est le suivant: \lstinline!gp = { id=NP_V_PP_about_topic_PP_to_co_agent dia=132 }!. On peut y lire que la diathèse précise que le 1 actant sémantique reste le 1er actant syntaxique, mais que le deuxième actant syntaxique est le 3e sémantique et ainsi de suite. Elle crée donc les arcs syntaxiques à partir de ces informations et ils sont vides.

\begin{figure}[htb]
	\centering
	\includegraphics[width=1\textwidth, trim = {0cm 0cm 0cm 0cm},clip]{ch6/figs/actant_gp_ijk.png}
	\caption{Application d'une règle actancielle: actant\_gp\_ijk}
	\label{deroulement2}
\end{figure}

\subsection{Application des contraintes sur les noe{}uds}
Dans l'ancienne version de GenDR, la règle actancielle contraignait aussi les noe{}uds nouvellement créés en syntaxe. Notre système a dû créé une règle séparée de la règle actancielle pour des raisons d'efficacité. Nous avons ainsi créé une règle qui contraints chaque noe{}ud nouvellement créé à partir des informations demandées par le patron de régime pour chaque actant syntaxique. Bref, la règle récupère les restrictions sur les noe{}uds dans le gpcon tel qu'on le voit à la figure \ref{gpexemple}. La règle s'applique donc 3 fois puisqu'il y a trois noe{}uds vides. On a maintenant trois noe{}uds contraints.

\subsection{Lexicalisation des noe{}uds contraints}
Ensuite, on répète une règle de lexicalisation. Dans ce cas il s'agit de \emph{lex\_standard} puisque tous les sémantèmes figurent dans le semanticon et le lexicon. La règle s'applique trois fois car il y a trois noe{}uds. Puis la lexicalisation réussi à chaque fois puisque ces lexèmes satisfont les contraintes imposées aux noe{}uds.
\begin{figure}[htb]
	\centering
	\includegraphics[width=1\textwidth, trim = {0cm 0cm 0cm 0cm},clip]{ch6/figs/lex.png}
	\caption{Applications d'une règle de lexicalisation: lex\_standard}
	\label{deroulement3}
\end{figure}

\subsection{Application de la règle \emph{actant\_gp\_selection}}
Finalement, la règle \emph{actant\_gp\_selection} s'applique encore, mais cette fois-ci pour les lexèmes \lex{teacher},\lex{student} et \lex{history}. Cette règle récupère leurs traits gp.id et gp.dia. Comme ce sont tous des noms communs, ils héritent du gp par défaut de la classe nominale id=NP dia=1 (nous n'avons pas plus d'information sur les patrons de régime des noms compte tenu que VerbNet se spécialisait dans les verbes, mais il existe d'autres ressources parmi celles que nous avions mentionnées qui pourrait combler cette lacune. C'est pourquoi nous n'avons que un gp pour les noms et qu'il est doté d'une diathèse simple permettant de faire la relation complément du nom). encore puisque ces nouveaux noe{}uds lexicalisés déclenchent l'application de la règle. Dès qu'un x a un gp, on va le repêcher, même si on s'en sert pas après. Si l'un d'entre eux avait eu un complément du nom, alors la sélection du gp aurait prouvé son utilité. C'est une application systématique.

Bref, l'application de toutes ces règles à notre input(figure \ref{input-text}) a permi son arborisation. Nous décrirons dans la section suivante le passage vers la structure syntaxique de surface.

\subsection{Lexicalisation de surface}
La première étape est de lexicaliser en surface les lexèmes profonds. Cette règle récupère aussi la partie du discours de surface de l'entrée lexicale. Le procédé est le même que nous avons vu au chapitre \ref{chapgendr}.

\subsection{Règles actancielles de surface}
Une fois que les lexèmes sont réalisés en surface, les règles actancielles de surface sont déclenchées. Trois règles actancielles de surface seront appliquées puisqu'il y a 3 arcs de dépendances (I, II et III) à réaliser. Concrètement, la règle récupère la valeur du trait \texttt{rel}. 

Donc, la règle synt\_subj est déclenchée et le système récupère l'information sur l'actant syntaxique qui possède le trait texttt{rel}. Comme nous avons choisi l'arborisation à partir du gp \emph{NP\_V\_PP\_to\_co\_agent\_PP\_about\_topic}, le système récupère l'information suivante: \lstinline! I={rel=subjective dpos=N}!. Le produit est le changement d'étiquette de la relation pour la valeur \texttt{subjective}.

Simultanément, la règle \emph{synt\_actant\_prep} est déclenchée une première fois pour faire la correspondance entre l'arc syntaxique qui lie \lex{talk\_3} et \lex{history}. La règle récupère ainsi le code suivant: \lstinline! II={rel=oblique dpos=N prep=about}!. Cela signifie que l'actant syntaxique II correspond à la relation \texttt{oblique} en RSyntS. Cette règle est déclenchée car l'un des actants syntaxiques s'expriment en syntaxe de surface à l'aide d'une préposition (une lexie fonctionnelle). Cela a pour incidence que le noe{}ud où profond de \lex{history} se scinde en deux afin que la préposition \lex{about} face le pont entre le verbe et l'objet indirect qu'il sélectionne. Ce phénomène est illustré par la figure~\ref{deroulement4}. 

Cette règle se déclenche une seconde fois pour traiter l'actant syntaxique III \lex{students}.

\subsection{Règles des déterminants}
Finalement, la règle det\_def réalise les déterminants qui doivent apparaître en syntaxe de surface. Ceux-ci correspondent aux traits que nous avions encodés dans l'input de départ. Seul \lex{teacher} et \lex{student} gouverneront des déterminants puisque leurs représentations sémantiques demandaient que les noe{}uds soit définis d'une certaine manière en syntaxe de surface. La règle de déterminant réalise \lex{the} lorsque c'est défini et \lex{a} lorsque le noe{}ud est marqué comme indéfini. Il s'agit d'une règle propre à l'anglais.
\begin{figure}[htb]
	\centering
	\includegraphics[width=0.5\textwidth, trim = {0cm 0cm 0cm 0cm},clip]{ch6/figs/ssynt.png}
	\caption{Applications des règles actancielles et réalisation des lexies fonctionnelles}
	\label{deroulement4}
\end{figure}
Cela met fin à notre chapitre implémentation. Nous passerons donc à la phase d'évaluation pour vérifier si notre système performe tel que nous l'avions prévu.
%!TEX root = ../memoire.tex
\chapter{Évaluation}\label{ch:eval}

Avant d'entrer dans le vif du sujet, il serait pertinent de faire un bref retour sur les méthodes d'évaluation en \ac{GAT}. \cite{ReiterInvestigationValidityMetrics2009} expliquent qu'il y existe trois types classiques d'évaluation. Ils nomment d'abord la méthode d'évaluation qui se base sur l'exécution d'une tâche en utilisant les textes génénés automatiquement. Ils nomment aussi la méthode d'évaluation humaine. Et finalement, ils parlent des méthodes métriques (automatiques). \cite{ReiterBuildingNaturalLanguage2000} s'étant penché plus tôt sur la question de la validité des méthodes d'évaluation automatiques, ils étaient en faveur d'une évaluation faite par des humains. Toutefois, environ une décennie plus tard, \cite{ReiterInvestigationValidityMetrics2009} remarquaient que les méthodes d'évaluation automatiques se faisaient de plus en plus populaires. Notamment, la méthode BLEU qui avait été développée, à la base, pour les systèmes de traductions automatiques. Nous ferons donc un bref survol de ces méthodes pour décider laquelle se prête le mieux à notre expérience.

BLEU a été créé à la base pour évaluer les rendements des traductions automatiques. Il s'agissait de comparer des outputs d'un système de traduction automatique à un ensemble de traductions humaine (servant de point de référence). Comme la traduction automatique et la génération automatique comportent toutes les deux l'aspect automatique, des chercheurs comme (Langkilde 2002; Habash 2004) ont estimé que la \ac{GAT} bénéficierait de cette méthode d'évaluation. Toutefois, lors de son passage à SemEval2017, FORGe a notamment été évalué par des méthodes métriques et \cite{DBLP:conf/semeval/MilleCBW17} ont fait une brève critique de cette méthode. Ils soulignent que BLEU évalue effectivement bien la couverture, mais comporte des lacunes pour analyser la qualité de chaque output. Leur système avait un score au dessus de la moyenne pour ce qui était de l'évaluation humaine, mais avait reçu un score plus faible pour selon la méthode BLEU. Ils expliquent ce décalage en mentionnant que FORGe mettait de l'avant la qualité de ses outputs par rapport à la quantité. De sorte que ce système filtre à deux reprises les constructions fautives ou potentiellement fautives. (Scoot et Moore, 2007) donnaient aussi quelques mises en garde de la méthode BLEU en précisant qu'elle n'évalue pas toujours correctement des propriétés linguistiques cruciales.

Les méthodes d'évaluation basées sur l'exécution d'une tâche à l'aide de textes générés automatiquement sont assez courantes selon \cite{ReiterInvestigationValidityMetrics2009}. Ces auteurs estiment qu'il s'agit de la méthode qui évalue le mieux le contenu des réalisations d'un système de \ac{GAT}. Toutefois, ils nous mettent en garde que c'est malheureusement la méthode la plus coûteuse en termes de temps et de ressources. En résumé, plus l'output est lisible et clair, plus hautes sont les chances que la tâche soit réalisée rapidement et correctement. Cette méthode n'est pas toujours facile à mettre en place car il faut trouver des participants prêt à donner de leur temps pour lire les rapports générés et effectuer une tâche correspondate. 

Finalement, il y a l'évaluation humaine, plus simple à faire que la méthode à base de tâche, mais plus lente que la méthode automatique. Toutefois, cela reste une méthode très populaire dans le domaine. Il s'agit de coter les outputs en fonction de leur performance à produire des phrases syntaxiquement et sémantiquement acceptables au bon jugement d'un évaluateur.

Considérant ces trois méthodes, nous devons en exclure deux: celle qui est basée sur une tâche et la méthode automatique. D'abord notre système ne réalise pas du texte dans un but précis. On n'a pas de tâche à effectuer pour tester la validité des réalisations. De plus, nous n'avions ni le temps, ni les ressources pour entreprendre ce type d'évaluation. Ensuite, nous ne pouvons pas utiliser la méthode automatique car notre système génère des arbres de dépendances de surface. Les systèmes utilisant la méthode automatique comparent des chaînes de caractères (des réalisations de surface où les textes sont linéarisées et morphologisés). Il nous est donc impossible de comparer nos arbres syntaxiques de surface avec du texte. Il ne reste qu'une méthode d'évaluation s'offrant à nous et c'est celle faite par des humains basée sur nos jugements.

\section{Mise en place de l'évaluation}
Pour procéder à l'évaluation de notre système, nous avons utilisé les outputs du script \ref{structurepython} (voir le chapitre \ref{ch:python}). Ceux-ci étaient des structures sémantiques vides dépourvues de prédicats et d'arguments. Il n'y avait que le code pour encadrer le graphe et le texte à reproduire sémantiquement. Nous avons comblé les 978 structures vides en y encodant les unités et relations sémantiques qui correspondaient à l'énoncé. La tâche est simple, nous passerons ces structures sémantiques en input à notre système et nous évaluerons les réalisations produites.

Comme nous avions une quantité limité de temps, nous avons décidé de prendre un échantillon des 978 strucutres sémantiques. Nous en avons choisi 75 aléatoirement. Parmi celles-ci, 25 ont servi à une partie développement précédent la phase d'évaluation comme telle. Ces 25 structures ont été passées au système afin de voir quels sont les problèmes immédiats que nous pouvons réglés sans vérifier si la qualité des arbres produits. Cette phase de développement nous a permi de constater qu'une bon nombre de nos inputs comportaient des problèmes. Nous avons donc noté le type de problème que les inputs comportaient afin de corriger le tir pour la partie évaluation. La partie de développement nous a aussi permi de constater que certains lexèmes appartenant à des PDD différentes apparaissaient en double dans notre dictionnaire. Autrement dit, le verbe \lex{work} et le nom \lex{work} ont la même forme, le système ne sait pas comment les différencier. Cela a donc une incidence sur la réalisation puisque le système construit l'arbre syntaxique à partir du premier lexème qu'il récupère. Donc si nous avions besoin du verbe dans l'arbre et que le système choisi le nom, la réalisation échouera nécessairement. C'est pourquoi nous avons procédé à un filtrage massif de ces cas. Nous les avons réglé en créant une entrée sémantique dans le semanticon qui contiendra les deux entrées lexicales: une version verbale et une version nominale. Elles seront distinguées ainsi dans le semanticon: \lstinline!work { lex = work_n  lex= work_2}!. Bref, nous n'avons pas analysé en profondeur la nature des générations à cette étape, nous voulions seulement répertorier et corriger les problèmes de surface.

Ensuite, nous avons passé au peigne fin chacun des inputs qui seront évalués. Nous nous sommes assurés que les problèmes d'entrées lexicales répertoriés dans la partie développement étaient corrigés en vue de l'évaluation. Finalement, nous avons pu procéder à la phase de tests. Nous avons testé les 50 structure sémantiques restantes. Pour ce faire, nous avons développé un script qui générait automatiquement toutes les représentations visuelles et textuelles des passages RSem-RSyntP et RSyntP-RSyntS. La partie visuelle permettait de regarder les différentes constructions d'arbres rapidement pour voir lesquelles étaient des réussites ou des échecs. La partie textuelle nous permettait de voir les informations sur les n\oe{}uds. Par exemple, quel était le patron de régime sélectionné pour cette arborisation ou alors, quelle était la diathèse sélectionnée, ou la partie du discours demandée par le n\oe{}ud, etc. Ces informations ne sont pas explicitées dans le format graphique de présentation des outputs. 
                              
\section{Rappel}

Nous avons évalué le rappel ainsi:\[\frac{\text{nombre de structures attendues générées}}{\text{nombre de structures attendues}}\]. Cette méthode d'évaluation produit un rappel de 62\%. Ce score se situe au-dessous de nos attentes. Mais il est expliqué par plusieurs facteurs cruciaux dont la majorité peut être corrigée en peu de temps. \draft{entre le 20 et le 30 je vais refaire les calculs avec les corrections pour voir qu'est-ce que ça aurait donné dans le cas échéant.}

D'abord, il y a les erreurs d'encodage qui ont su échapper aux mailles du filet lors de notre filtrage post-développement. Il y avait très peu d'erreur dans l'input sémantique. Nous en avons relevé un seule. Il s'agit de l'emploi d'une mauvaise forme d'un verbe. Nous avons pris grill\_1 et il aurait fallu prendre grill\_3. L'utilisation d'une mauvaise acception d'un verbe mène à l'impossibilité de récupérer le bon patron de régime. En termes d'erreurs humaines, certaines se sont glissées dans les dictionnaires à notre insu. Notamment, une diathèse manquante pour un patron de régime donné. Cela mène à l'impossibilité de réaliser le gp. Ou alors, l'oubli d'une préposition dans une actant syntaxique. Cela fait en sorte qu'en surface, la phrase n'est pas celle qu'on attendait.

Les classes par défaut ont aussi leur lot de problèmes. Effectivement, nous avons dû créer une classe \texttt{quote} pour les patrons de régime qui sélectionne des paroles (Ex: Helen told Ellen "leave the room"). Toutefois, une erreur s'est probablement glissée dans l'encodage de cette classe et le système n'a pas pu la réaliser lors de l'arborisation. Nous avons aussi eu un problème avec les classes qui sélectionnent certaines prépositions. Le système n'est pas capable de récupérer cette préposition dans la classe par défaut, ce qui empêché de réaliser la phrase souhaitée. 

Nous avons eu beacoup de problèmes de rappel lié aux manque de données de la part de VerbNet. Effectivement, il y eut quelques cas de verbes utilisés dans les phrases exemples de VerbNet qui ne figuraient pas dans leur propre dictionnaire verbal. Par le fait même, les patrons de régime associés à ces verbes ne sont pas encodés. Cela mène à l'impossibilité de générer l'arbre qu'on se serait attendu de voir. beg, believe. C'était souvent le cas lorsque la phrase exemple contenait deux verbes. Comme on n'a pas les informations requises pour process le second verbe (diathèse et gp) l'arbre que nous pensions généré ne sera jamais retrieve par GenDR. 

De plus, parmi les problèmes qui nous sont hérités de VerbNet, il y aussi eu quelques cas où la structure sémantique de l'input demandait un patron de régime Y de la part du prédicat, mais que ce patron de régime ne figurait pas parmi la liste de gps de la classe associée à ce verbe. Ou bien he backed out of going  (go\_2: n'avait pas le gp pour réalsier on the trip)

Il y aussi des problèmes due à des incompatibilités sémantique-syntaxe entre VerbNet et la TST. Effectivement, nous n'avons pas tenté de créer une struture sémantique pour qu'elle plaise au patron de régime donné par VerbNet. Cela fait en sorte que quelques des tests allaient nécessairement échouer. Par exemple, la phrase \form{Tom jumped the horse over the fence} est régie par le verbe jump\_2 qui est exprimé par le gp suivant: \lstinline!gp = { id=NP_V_NP_PP_SPATIAL_location dia=123 }!.Les trois actants sémantiques étant \sem{Tom} \sem{horse} et \sem{fence} selon VerbNet. Toutefois, nous considérons que dans ce scénario, le verbe \lex{jump} en surface exprimerait plutôt \sem{faire sauter}. Il existe une fonction lexicale permettant ce genre de situation, mais nous ne l'avons pas encodé dans cette version de VerbNet puisque nous ne touchons pas aux fonctions lexicales. Nous avons noté d'autre cas d'incompatiblités théoriques lors de la phase de développement. 

Nous avons relevé beaucoup de cas d'utilisation de verbe support en surface ('X took a flight' ). Le gp utilisé pour take a flight est bon en soi, mais la représentation sémantique de take est flight ne devrait pas faire appel au gp de take, mais plutôt au gp de flight qui encode les fonctions lexicales (verbe support). et on ne perdrait pas le sens. le \sem{fly} incorpore les lexicalisations \lex{fly} et  \lex{flight}. Took est réalisé comme verbe support en récupérant la fonction lexicale encodée sous \lex{flight}. Nous avons aussi relevé d'autres types de collocations dont X  slept the sleep of the dead. Dans ce cas \form{the sleep of the dead} est une expression figée qui signifie une intensité. Elle modifie le verbe sleep. VerbNet laisse sous-entendre que sleep sélectionne un objet direct qui lui sélectionne un complément du nom. Mais en fait, il ne s'agit que d'un verbe modifié par un intensificateur.

Finalement, le dernier problème qui a le plus affecté le rappel est la défectuosité du mécanisme d'héritage des gps entre les classes dominées et les classes dominantes. Il s'agit du mécanisme dont nous vantions les mérites au début du chapitre. Concrètement, il s'agit d'un moyen qui fait en sorte qu'une class x pointe vers une classe y et hérite des traits encodés dans la classe y. Mais il se trouve que le mécanisme d'héritage des traits fonctionne partiellement. 

Tel que présenté dans le début du chapitre, les unités lexicales pointent vers des classes verbales, une classe verbale peut pointer vers une autre classe verbale, et les classe verbales non-dominées pointent vers la classe par défaut \texttt{verb}. Celle-ci donne les traits dpos=V et spos=verb à tous les lexèmes pointant vers des classes verbales. L'héritage de ce trait est réussi à travers les classes dominées et dominantes car le lexème de surface contient les traits. C'est pourquoi nous disons que le mécanisme fonctionne partiellement. Mais les patrons de régime ne semblent pas se transmettre d'une classe dominante à une classe dominée. Il semblerait que les traits \texttt{id} et \texttt{dia} qui sont encodés dans le trait \texttt{gp} ne sont pas transmis. Ce problème semble provenir de MATE qui n'est pas capable de transmettre l'héritage d'un trait à l'intérieur d'un autre trait. Pour mieux illustrer le concept, nous prendrons le cas de \form{Steved tossed the ball from the corner to the garden.}. Le sémantème \sem{toss\_3} pointait vers la classe \texttt{throw-17.1-1} qui est une classe dominée par "throw-17.1" qui est domineé par "throw-17.1". Cependant le patron de régime dont nous avions besoin pour réaliser la phrase souhaitée se trouvait dans le régime de la classe mère "throw-17.1". GenDR n'a donc pas été capable d'aller chercher les traits id et dia. Mais il a été capable de réaliser la racine de l'arbre avec le lexème \lex{throw\_3} en héritant du trait dpos de la classe par défaut \texttt{verb}.

Pour pallier à ce mécanisme d'héritage deffectueux, nous pourrions directement implémanter tous les gps des classes dominantes dans les classes dominées. Ainsi on garde l'architecture pensée de VerbNet où les classes héritent des patrons des autres classes. Le défaut de cette solution est que notre dictionnaire sera saturé d'information puisqu'il existe énormément de sous-classe. Toutefois, cette solution est facilement adaptable puisque nous avons encore les scripts Python ayant servi à extraire les patrons de régime et à créer le lexicon. Nous n'avons qu'à modifier le script pour que chaque sous-classe hérite explicitement des gps des classe qui les domine. Évidemment, cela aurait aussi un impact direct sur la précision, puisqu'on a plus de chance de générer une erreur si plusieurs gps sont disponibles.

\section{Précision}

Nous avons noté la précision de cette manière:\[\frac{\text{nombre de structures correctes}}{\text{nombre de structures générées}}\]. Cela nous donne une précision de 77\%.

Les erreurs humaines que nous avions mentionnées pour le rappel ont généralement un impact sur la précision. Par exemple, en mettant la mauvaise désambiguisation d'un verbe dans l'input, alors les patrons de régime utilisés pour réaliser la phrase ne seront pas les bons. Cela peut générer de bonnes phrases, si par chance un bon gp s'y trouve. Mais on a vu que cela pouvait générer des phrases incongrues: \form{She grilled the steak on me} ou \form{She grilled the steak in me}.

Les erreurs de VerbNet affectent aussi la précision de l'output négativement. Lorsqu'un verbe utilisé dans l'exemple n'est pas répertorié par VerbNet. Le système utilise ses règles de secours de lexicalisation et va tout de même tenter de réaliser quelque chose. C'est le cas d'une phrase ayant le verbe \lex{do} pour \form{I begged her to do it}. Comme GenDR n'a pas hérité de l'entrée que VerbNet aurait dû avoir, le système a tenté diverses réalisations. Certaines d'entre elles échoueront, mais d'autres seront réalisées en surface. C'est le cas de \form{I begged her for the do.}. Dans ce cas, GenDR a supposé que c'était possiblement un nom. Et comme il existait un patron de régime de \lex{beg} permettant cette construction, alors le système a réalisée cette incongruité. Nous avons relevé d'autres cas similaires dans notre évaluation.

Les informations extraites de VerbNet nous ont aussi fait réaliser des phrases quasi-correctes mais bizarres. C'est une conséquence des patrons de régime qui permettent 2 à 4 prépositions différentes pour la réalisation d'une actant syntaxique en surface. Le système va donc générer deux arbres différents corrects. Chacun d'entre eux aura une préposition différente pour l'actant concerné, mais généralement seule l'une d'entre elle génère une phrase grammaticale. Ainsi, GenDR a généré les phrases \form{The doctor cured pat of pneumonia} et \form{The doctor cured pat out of pneumonia}. La deuxième est un arbre bien construit, mais dont la sémantique est fautive.

Un autre type d'erreur de précision provient de la manière dont GenDR génère les arbres. Tel que nous l'avons vu plus tôt, le système crée x nombres de racines pour un input où x est le nombre de patron de régime inclu pour une classe verbale. 

Ensuite, on sélectionne les gps et on vérifie que la diathèse permet l'appplication du gp sélectionné. C'est là que le problème prend forme. GenDR peut ainsi sélectionner un gp fautif qui respecte la diathèse et les contraintes sur le n\oe{}ud. Mais, il est possible que l'arborisation échoue parce que l'un des actants syntaxiques choisi contient une préposition qui rendra la phrase agrammaticale. Concrètement pour régler ce problème, il faudrait revoir l'application de nos règles. Par exemple la phrase \form{The street gushed}. Rien dans notre système nous empêchait de réaliser cette forme fautive.

\section{Synthèse de l'évaluation}

\draft{Est-ce que c'est pertinent que je parle de la f1 ? Elle est de 69\%}
\draft{Est-ce que j'ai besoin de faire un retour sur ce qui a été dit ? Ou bien j'en parle dans ma conclusion finale}

Ce qu'on retire de tout cela sont deux problèmes majeurs. Le mécanisme d'héritage qui fonctionne partiellement avec l'architecture présente qui nuit au rappel.

De plus, il faut aussi tenir compte du fait que nous avons testé sur 5\% des structures d'input. Mais nous pensons que c'est un 5\% significatif. Les mêmes erreurs revenaient souvent dans l'analyse des résultats.


%!TEX root = ../memoire.tex

\chapter*{Conclusion}
Un des enjeux en GAT c'est, de générer du texte le plus naturel possible. Pour ça y'existe différentes manières. Parmi ces manièrs y'a la méthode à base de règles. Cette méthode nécessite l'emploi de dictionnaire pour encoder les propriétés du lexique. Mais, si on veut couvrir large, il nous faut avoir accès aux différentes comportements des mots du langage. Certains comportements sont prévisibles, donc on peut les traiter facilement, mais les comportements syntaxiques des verbes sont très différents et imprévisibles. Il faut donc encoder ces données pour pouvoir les réaliser et réaliser du langage représentant la richesse des langues naturelles. Un verbe peut s'exprimer de différentes manières, sélectionner des arguments divers à l'aide de prépositions diverses, etc. 

une chance que des gens se sont penchés sur cette problématique et ont développé des ressources encodant ce type d'information. Pas juste en GAT, mais dans le TAL en général, ce type de ressource est extrêmement utile. Nous nous sommes dit que ce serait bon d'intégrer une telle ressource à GenDR, spécialiement pour traiter les verbes, puisque ce sont eux qui contrôlent la plupart des énoncés et c'est la prtie du discours qui démontre le plus de variation quant aux constructions syntaxiques permises par chaque lexème. L'objet de ce mémoire est de pourvoir GenDR d'une couverture linguistique beaucoup plus grande que celle qu'il a présentement en y intégrant VerbNet qui est une base de données lexicales décrivant les comportements syntaxiques de 6\,394 verbes.Nous voulions voir si l'implémentation d'une telle ressource était viable dans un réalisateur profond et comment le faire, et est-ce que ça fonctionne ?.

6 lignes par chapitre
Dans le premier chapitre, nous introduisons la \ac{GAT} et le pipeline classique qui la compose. Nous nous sommes arrêtés sur la réalisation linguistique qui est la dernière étape du pipeline. Ensuite nous avons souligné qu'il existait différents types d'approches pour effectuer la réalisation linguistique: à base de patrons, à base de règles et à base de statistiques. Ensuite, nous avons exploré les différents réalisateurs de surface et profonds. Notamment, nous avons parlé de SimpleNLG, JSrealB et RealPro qui sont des réalisateurs de surface. Puis, nous avons décrit des réalisateurs profonds: KMPL, SURGE, MARQUIS, FORGe.

Dans le deuxième chapitre, nous avons décrit en détails le réalisateur profond: GenDR, un héritier de MARQUIS. Nous avons décrit l'architecture qui le composait et le logiciel MATE (conçu pour la TST) qui offre un éditeur de graphes, de dictionnaire et de règles pour développer et tester une grammaire computationnelle. Ensuite nous avons expliqué quelques notions de base de la TST dont l'interface sémantique-syntaxe. Puisque GenDR opère au niveau de cette interface en mettant toutes ces forces à modéliser l'arborisation et la lexicalisation. Finalement, nous avons démontré le fonctionnement du réalisateur GenDR à l'aide d'un exemple décrivant l'intéraction des règles et dictionnaires pour réaliser du texte à partir d'un input sémantique. Nous avons aussi mentionné que GenDR contenait très peu d'information sur toutes les constructions possibles des verbes, mais imagine si on l'enrichie qu'est-ce que ça pourrait donner.

Dans le troisième chapitre, nous faisons un survol des ressources lexicales potentielles que nous envisageons pour intégrer à GenDR dans le but d'augmenter sa couverture et de lui permettre de réaliser des constructions syntaxiques variées en fonction des verbes. Nous décrivons notamment WordNet, FrameNet, XTAG, LCS, Comlex, Valex,et le VDE et finalement VerbNet, le grand gagnant. Cette ressource lexicale est basée sur les travaux de \cite{verb-classes.levin.1993}. Ce qui nous a le plus attiré de cette ressource est l'organisation des classes verbales ainsi que la large couverture de VerbNet plus de 6000 verbes, désambiguisés.

Dans le quatrième chapitre, nous avons procédé à l'extraction des données lexicales de VerbNet. Nous avons d'abord extrait les informations syntaxiques des classes verbales en conservant la hiérarchie pensée par VerbNet. Puis nous avons extrait les verbes associés à chaque classe verbale et nous les avons désambiguïsés avant de les rajouter à notre dictionnaire. Ensuite, nous avons procédé à la création du dictionnaire de patron de régime à partir des données extraites. Le tout a été fait par l'entremise de Python qui nous a permis de manipuler les fichiers de VerbNet qui sont encodés en XML.

Dans le cinquième chapitre, nous avons démontré comment nous avons adapté GenDR à l'utilisation d'un dictionnaire de patron de régime. Le nombre de verbe décrit par le réalisateur passait de passe de 500 à 6394, puis nous avons complètement changé l'utilisation des patrons de régime avec la venue du dictionnaire de \ac{GP}. Cela a permi à GenDR d'avoir une couverture beaucoup plus large avec les lexèmes et de couvrir une énorme quantité de constructions syntaxiques possibles en anglais. Par le fait même on règle le problème des gps multilpes et on agrandi la couverture de GenDR. 
	
Dans le sixième chapitre, on a d'abord créé les scripts nous permettant de générer les structures de bases pour évaluer GenDR. Le résultat nous donne 978 structures sémantiques dont nous avons manuellement encodé le contenu. 50 ont été choisies aléatoirement pour faire l'évaluation. On a évalué le rappel avec une technique similaire à BLEU. Ensuite on a évalué la précision avec une évaluation humaine classique pour voir si les phrases générées étaient grammaticales et quel était le pourcentage de celle-ci sur le nombre de générées. GenDR performe à 100 pourcent partout rappel et précision. VerbNet nous fait faire des erreurs de précisions (manque de gp, manque d'entrée et GenDR guess fait n'importe quoi, prépositions pour le mm actant, ça donne des phrases incohérentes) VerbNet excellent pour le rappel, mais y'a des incompatibilités sémantiques entre la TST et VerbNet.  

Le travail que nous avons fait apporte plusieurs contributions importantes à la recherche en \ac{GAT}. Nous avons démontré comment s'implémenterait une ressource lexicale comme VerbNet dans un réalisateur profond à base de règles. Nous avons par le fait même démontré qu'avec une ressource comme Python nous pouvons extraire et manipuler les données d'un dictionnaire pour les implémenter dans un réalisateur. Nous avons montré qu'il est possible de prendre une ressource fait dans un cadre X et la transcoder pour l'adapter à un autre cadre théorique sans que ce ne soit trop encombrant. C'est ainsi que nous avons créé un dictionnaire de patron de régime codé en \emph{Sens-Texte}. Cela pourrait être utile à d'autres réalisateurs qui voudraient s'inspirer de cette théorie. Ou bien pour des systèmes n'utilisant pas les rôles thématiques. Nous avons montré qu'il est possible de doter un réalisateur profond d'une immense couverture grâce à de telles ressources et que leur implémentation donne déjà de bons résultats sans même avoir été modifié. 80\% sans retouches, mais on pourrait l'adapter à nos besoins et aller chercher une meilleure précision. Une autre contribution de cette recherche est que nous avons par le fait même évaluer la capacité d'utiliser VerbNet précisément comme dictionnaire verbal pour un réalisateur profond en GAT. Les avantages et les inconvénients de cette ressource. Notamment, dans 20\% des phrases générées contenait des incongruités.

Finalement, ce projet de recherche ouvre la porte à GenDR à se doter d'autres ressources lexicales similaires pour améliorer se couverture. Notamment, nous pourrions implémenter les régimes des noms, grâce aux autres ressources (comme FrameNet). Nous pourrions aussi aller chercher les gps manquant dans d'autres ressources que nous avons répertoriés. De plus, comme il existe des VerbNet dans d'autres langues (francais, portugais et italien), on aurait avantage à aller les chercher, ce sera facile vu qu'on a déjà le frame d'importation. Et comme GenDr est multilingue ça vient rejoindre le but.





% Annexes
% Enlever le commentaire de \appendix pour faire vos annexes.
% Les annexes sont ensuites faites comme un chapitre normal : \chapter{nom_de_l'annexe}.
\appendix

%%%%%%%%%%%%%%%%%%%%%%%%%%%%%%%%%%%%%
%%   BIBLIOGRAPHIE                  %
%%%%%%%%%%%%%%%%%%%%%%%%%%%%%%%%%%%%%
  % Enlever les commentaires de la prochaine commande si vous préférez que le
  % chapitre s'appelle « Références » plutôt que « Bibliographie » (au choix selon le contexte).
%\let\bibname=\refname   

%% Lorsque vous serez prêt à faire afficher votre bibliographie
%% et vos références, enlevez les commandaires des commandes suivantes
%% et donnez le nom de votre fichier .bib à la commande \bibliography{..}
%% (consultez l'exemple au besoin).  Vous pouvez utiliser le style de votre
%% choix.  Le fichier francaissc.bst est inclus avec le gabarit.  Vous pouvez
%% utiliser ce style avec  \bibliographystyle{francaissc}
% 
\bibliographystyle{biblio/francaissc}		    % Le style de la bibliographie. Notons que les extensions ne sont pas données pour ces deux fichiers.
\bibliography{biblio/references}		    % La base de données contenant des entrées bibliographiques. Seules celles référencées dans le texte seront ajoutées \`a la bibliographie.


%\chapter{Le titre}
% Trouve un titre plus spécifique

\section{Section un de l'annexe A}

La premi\`ere annexe du document.

Pour plus de renseignements, consultez le site \url{http://www.fesp.umontreal.ca}.
\begin{table}[htb]
	\renewcommand{\arraystretch}{1.25}
	\newcommand{\dotrule}[1]{\parbox[t]{#1}{\dotfill}}
	\centering
	\caption[Titre alternatif pour la table des mati\`eres]{Liste des parties}
	\label{tab:parties}
	\begin{tabular}{p{0.6\textwidth}@{\hspace{0.15\textwidth}}p{0.15\textwidth}}
		\hline\hline & \\[-3mm]
  		Les couvertures conformes 											& obligatoires			\\
		Les pages de garde 													& obligatoires			\\
		La page de titre 													& obligatoire			\\
		Le résumé en français et les mots clés français						& obligatoires			\\
		Le résumé en anglais et les mots clés anglais 						& obligatoires			\\
		Le résumé de vulgarisation											& facultatif			\\
		La table des mati\`eres, la liste des tableaux, la liste des figures 	& obligatoires			\\
		La liste des sigles, la liste des abréviations						& obligatoires			\\
		La dédicace															& facultative			\\
		Les remerciements 													& facultatifs			\\
		L'avant-propos 														& facultatif			\\
		Le corps de l'ouvrage												& obligatoire			\\
		L'index analytique													& facultatif			\\
		Les sources documentaires 											& obligatoires			\\
		Les appendices (annexes) 											& facultatifs			\\
		Le curriculum vit\ae{}												& facultatif			\\
		Les documents spéciaux 												& facultatifs			\\
		[3mm] \hline\hline
	\end{tabular}
\end{table}
Pour plus de renseignements, consultez le site \url{http://www.fesp.umontreal.ca}.

%\chapter{Le titre2}

texte annexe B

\end{document}
