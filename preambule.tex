\usepackage[utf8]{inputenc}	% Fait dans le classe		
\usepackage[T1]{fontenc}

% La commande \sloppy peut avoir des effets étranges sur les
% lignes de certains paragraphes.  Dans ce cas, essayez \fussy
% qui suppresse les effets de \sloppy. (\fussy est le comportement par défaut.)
% On redéfinit \sloppy, pour tenter de réduire les comportements étranges.  
% Le seul changement apporté à la version originale est la valeur de \tolerance.
\def\sloppy{%
  \tolerance 500%  %9999 dans LaTeX ordinaire, mauvaise idée.
  \emergencystretch 3em%
  \hfuzz .5\csname p@\endcsname
  \vfuzz\hfuzz}
\sloppy  

\usepackage[usenames,dvipsnames]{xcolor}
\usepackage{courier}

\usepackage{graphicx,amsmath,amsfonts,amssymb,setspace,subfigure,color,icomma,dsfont}
% graphicx		Pour importer des images (PDF, JPG, PNG).
% amsmath		Écriture selon les normes de l'AMS.
% amsfonts		Fontes additionnelles de l'AMS.
% amssymb		Écriture des symbols de l'AMS.
% setspace		Permet de régler la distance interligne dans le document.
% subfigure		Simplifie l'inclusion de figures côtes-\`a-côtes.
% color			Pour l'utilisation de couleurs dans le texte.
% icomma		Reconnait la virgule comme caractère mathematique de facon intelligent
% dsfont		symboles mathématiques




\usepackage[pdfpagemode=UseNone,pdfstartview={XYZ null null null}]{hyperref}	% Cette extension permet l'insertion d'hyperliens dans votre document pdf.
 \definecolor{dark-red}{rgb}{0.4,0.15,0.15}					% Ici, trois couleurs sont définies et seront utilisées pour colorer les "hyperliens".
 \definecolor{dark-blue}{rgb}{0.15,0.15,0.4}
 \definecolor{medium-blue}{rgb}{0,0,0.5}
 \hypersetup{colorlinks,linkcolor={dark-red},citecolor={dark-blue},urlcolor={medium-blue}}
\usepackage{bookmark}  % Remédie à des petits problème de <hyperref> (important qu'il soit chargé après <hyperref>)
  % Enlever les commentaires et remplir cette section avec l'information pertinente.
  % Ceci ajoute des « méta-données » au pdf.  C'est optionnel, mais recommandé.
  % Vous pouvez voir ces méta-données en ouvrant un visionneur de pdf et en cherchant les
  % propriétés du pdf.  (Vous pouvez aussi tapez ' pdfinfo <nom-du-pdf> '  dans un terminal.)
  % Ces données sont utiles, par exemple, pour augmenter les chances qu'un algorithme de
  % recherche trouve votre document sur Internet, une fois diffusé.  Autrement dit, ceci
  % peut aider à augmenter la visibilité de votre travail.
%\hypersetup{
%    pdftitle = {Exemple d'une thèse}, 
%    pdfauthor = {Coadmin},
%    pdfsubject = {Exemple pour utiliser le gabarit du DMS},
%    pdfkeywords = {DMS, gabarit, exemple, thèse, mémoire, coadministrateur}
%}

% Numérotation des équations par section et numérotation des tableaux et figures par chapitre.
\numberwithin{equation}{section}
\numberwithin{table}{chapter}
\numberwithin{figure}{chapter}

% Définition des environnements utiles pour un mémoire scientifique.
% 1	%FR
\newtheorem{cor}{Corollaire}[section]
\newtheorem{deff}{Définition}[section]
\newtheorem{exemple}{Exemple}[section]
\newtheorem{lemme}{Lemme}[section]
\newtheorem{prop}{Proposition}[section]
\newtheorem{rem}{Remarque}[section]
\newtheorem{theo}{Théor\`eme}[section]

% Si vous préférez que les corollaires, definitions, théor\`emes, etc. soient numérotés par le même compteur, utilisez plutôt ce bloc de commandes : 
%2	% FR
%\newtheorem{corollary}{Corollaire}[section]
%\newtheorem{definition}[corollary]{Définition}
%\newtheorem{example}[corollary]{Exemple}
%\newtheorem{lemma}[corollary]{Lemme}
%\newtheorem{proposition}[corollary]{Proposition}
%\newtheorem{remark}[corollary]{Remarque}
%\newtheorem{theorem}[corollary]{Théor\`eme}

%Pour faire des énumérations : easylist
\usepackage[ampersand]{easylist}



\onehalfspacing				% Fixe la distance interligne \`a "1.5". Pour une interligne double, utilisez plutôt "\doublespacing".

\allowdisplaybreaks			% Cette commande autorise LaTeX \`a briser les blocs d'équations, permettant ainsi une couverture plus uniforme des pages.

%Commande pour numéroter les tableaux en chiffres romains (préfixe: le numéro du chapitre)
\renewcommand{\thetable}{\thechapter. \Roman{table}}

\setlength{\parskip}{1ex plus 3pt minus 1pt}

% Commandes pour linguistique
\usepackage{lng,lng-acro-fr}

% Commandes générales
\usepackage{gen}

% Feedback
\newcommand{\FL}[1]{\draft{\textbf{(FL)}}\footnote{\draft{#1}}}

%lstlistings for writing code in Python and XML
\usepackage{listings}
\usepackage{color}
\usepackage{setspace}
\definecolor{Code}{rgb}{0,0,0}
\definecolor{Decorators}{rgb}{0.5,0.5,0.5}
\definecolor{Numbers}{rgb}{0.5,0,0}
\definecolor{MatchingBrackets}{rgb}{0.25,0.5,0.5}
\definecolor{Keywords}{rgb}{0,0,1}
\definecolor{self}{rgb}{0,0,0}
\definecolor{Strings}{rgb}{0,0.63,0}
\definecolor{Comments}{rgb}{0.38, 0.25, 0.32}
\definecolor{Backquotes}{rgb}{0,0,0}
\definecolor{Classname}{rgb}{0,0,0}
\definecolor{FunctionName}{rgb}{0,0,0}
\definecolor{Operators}{rgb}{0,0,0}
\definecolor{Background}{rgb}{0.98,0.98,0.98}
\definecolor{gray}{rgb}{0.4,0.4,0.4}
\definecolor{darkblue}{rgb}{0.0,0.0,0.6}
\definecolor{cyan}{rgb}{0.0,0.6,0.6}
\lstdefinelanguage{Python}{
xleftmargin=1em,
framextopmargin=2em,
framexbottommargin=2em,
showspaces=false,
showtabs=false,
showstringspaces=false,
frame=l,
tabsize=4,
% Basic
basicstyle=\ttfamily\scriptsize\setstretch{1},
backgroundcolor=\color{Background},
% Comments
commentstyle=\color{Comments},
% Strings
stringstyle=\color{Strings},
% keywords
morekeywords={import,from,class,def,for,while,if,is,in,elif,else,not,and,or,print,break,continue,return,True,False,None,access,as,,del,except,exec,finally,global,import,lambda,pass,print,raise,try,assert},
keywordstyle={\color{Keywords}\bfseries},
% additional keywords
morekeywords={[2]@invariant,pylab,numpy,np,scipy},
keywordstyle={[2]\color{Decorators}\slshape},
emph={self},
emphstyle={\color{self}\slshape},
%
}
\linespread{1.3}

\lstdefinelanguage{XML}
{
  morestring=[b]",
  morestring=[s]{>}{<},
  morecomment=[s]{<?}{?>},
  stringstyle=\color{black},
  identifierstyle=\color{darkblue},
  keywordstyle=\color{cyan},
  morekeywords={xmlns,version,type}% list your attributes here
	numbers=left,
 numberstyle=\footnotesize,
 numbersep=1em,
 xleftmargin=1em,
 framextopmargin=2em,
 framexbottommargin=2em,
 showspaces=false,
 showtabs=false,
 showstringspaces=false,
 frame=l,
 tabsize=4,
% Basic
basicstyle=\ttfamily\scriptsize\setstretch{1},
backgroundcolor=\color{Background},
}

\newcommand{\mate}[4]{
  \begin{figure}[htb]
    (PG:) #2

    (PD:) #3

    (COND:) #4
  \caption{#1}
  \end{figure}
}