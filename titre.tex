%!TEX root = memoire.tex

% Voici les variables pour la création de votre page titre.

\title{Extraction des patrons de régime de VerbNet pour une implémentation dans un système de génération automatique de texte}
\author{Daniel Galarreta-Piquette}
\copyrightyear{2018}
\date{\today}									% Date de dépôt du document.
	% ces éléments ne doivent plus apparaittre selon les dierectives de la FESP
	% si toutefois vou souhaitez les inclure, il faudra aussi modifier le document dms.cls
% \president{Nom du président du jury}
% \directeur{Nom du directeur de recherche}
% \codirecteur{Nom du codirecteur}
% \membrejury{Nom du membre du jury} 
% \examinateur{Nom de l'examinateur externe}
% %\membresjury{ala, beta, gamma}
% %\plusmembresjury{psi, zeta, omega} 
% \repdoyen{Nom du représentant du doyen} 
\dateacceptation{Date d'acceptation}
\sujet{linguistique}							% Votre discipline de recherche, soit "mathématiques" ou "statistique".
%\orientation{mathématiques fondamentales}		% Cette commande est optionnelle. Les choix courants sont : "mathématiques fondamentales", "mathématiques de l'ingénieur" et "mathématiques appliquées".

\department{Département de traduction et linguistique}

% Fin des variables \`a définir. La commande "\maketitle" créera votre page titre.


\pagenumbering{roman}
\maketitle    
 
